\chapter{Arbejdsvilkår}\label{arbejdsvilkuxe5r}

Et skib kan være en farlig arbejdsplads! Risikoen for havari, kollision,
brand, grundstødning og sørøveroverfald er nogle af de ting, en sømand
må leve med. De mest fremtrædende søulykker161 er grundstødning162 og
kollision163. Ud over disse egentlige skibsulykker kan søfolkene blive
udsat for forskellige typer arbejdsulykker. De farligste situationer er,
når skibet bliver gjort klar til lastning og losning, hvor der er risiko
for at falde eller snuble eller blive ramt af en genstand i bevægelse.
En del af ulykkerne medfører død eller invaliditet. Det er udbredt med
høreskader blandt skibenes maskinbesætning på grund af det høje
støjniveau i maskinen. Maskinfolkene er også udsat for andre farer, som
Mads her fortæller om:

''Så nu sveder vi i maskinen igen, med temperaturer på 30+ og en
luftfugtighed på over 90 \% udenfor. Og når man står lige ved siden af
udstødningen fra hovedmotoren, er det lige som at gå ind i en knaldende
varm sauna, iført en bomuldskeddeldragt med lange ærmer. Og det gør det
ikke meget bedre, at vi skal lave hårdt fysisk arbejde samtidig. Så på
en dag kan der sagtens ryge 5-6 liter vand ned, og vi er nødt til at
tage salttabletter, fordi vi sveder så meget. Og de har det vist ikke
meget nemmere på dækket. I dag var er en thai der dehydrerede164, fordi
han ikke havde forstået, at han skulle drikke meget vand.''. Søfolkene
arbejder også med mange kemikalier om bord, f.eks. til rengøring og
tankrensning, hvilket kan give skader på lang sigt. Søfartsstyrelsen og
rederierne har dog gennem de seneste år sat fokus på søfolkenes
arbejdsmiljø og gør en stor indsats for at forbedre forholdene. Den nye
teknologi har gjort arbejdet om bord på et containerskib mere monotont
og stressende, og der kan være en sammenhæng mellem stress og
arbejdsulykker. Søfolks arbejdsmiljø er ofte præget af stress og uro,
hvilket kan påvirke deres evne til at sove og hvile ordentligt ud.

161

Søulykker er, når et skib oplever ulykker til søs. Grundstødning er, når
et skib ved et uheld sejler ind i noget under skibet. 163 Kollision er,
når to eller flere skibe støder ind i hinanden. 164 Dehydrere er, når en
person mangler vand. 162

49

I langfart kommer søfolkene ud for særligt belastende
arbejdssituationer, hvor der ofte er tale om sejlads i stærkt
trafikerede farvande med dårligt sigt, hyppige havneanløb, ophold i
havne om dagen og sejlads om natten, meget overtidsarbejde og deraf
følgende søvnmangel. Herudover kan det også rent psykisk være hårdt med
monotonien165 under de lange rejser, den stadige kontakt med de samme
mennesker, den daglige støj og skibets rystelser samt det at være væk
fra familie og venner. Så alt i alt kan det i dag stadig være et
farefuldt, men også meget spændende, erhverv, at være sømand.
