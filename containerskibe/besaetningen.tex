\chapter{Besætningen}\label{besuxe6tningen}

Et containerskibs besætning er inddelt i en helt speciel rangorden, hvor
officererne har grader, ligesom man har i militæret. Det er for at
sikre, at kommandovejene fungerer effektivt. Om bord på et containerskib
er skibsføreren den øverste myndighed, og så er der 1., 2., og 3.
styrmand, maskinchefen og 1. og 2. maskinmester, skibsassistenter,
elektriker, hovmesteren, stewarden154 og så vedligeholdelses- og
malerarbejdere. Søren Brask-Pedersen, som er under uddannelse til
maskinmester, fortæller her om hierarkiet om bord på skibet: ''Der er
selvfølgelig antallet af striber på skulderen, som betyder, hvem der
bestemmer, men jeg har aldrig været udsat for, at der ikke er blevet
hørt efter, hvis jeg har sagt noget, selv om jeg jo altid har været
lavest i hierarkiet155. Så på den måde bliver alle hørt, men det er
selvfølgelig ham med flest striber, som bestemmer, og det er så også ham
som tager ansvar, hvis der sker noget, så det er retfærdigt nok.''. Ud
over, at skibsføreren er chef for skibet, fungerer han som politimester
og dommer, fordi han er den højeste myndighed. Men hvad sker der, hvis
man f.eks. sover på vagten eller kommer for sent? ''I første omgang får
man en advarsel, men gentager det sig, bliver man sat af i næste havn.
Man vil ikke have folk, der ikke passer deres arbejde om bord. Jeg har
prøvet, da jeg kom om bord første gang, at en var blevet sat af i havnen
inden, det var en filippiner, han blev sat af, fordi han ikke passede
sit arbejde. Det bliver noteret i dagbogen, når der er blevet uddelt
advarsler, så man kan gå tilbage og se hvem der har fået hvad.''. Søren
Hoppe er skibsfører på SOFIE MÆRSK, og han mener, at en af hans
vigtigste opgaver er at sikre en god stemning blandt personalet. Men
ellers er skibsførerens arbejdsopgaver og ansvarsområder nedskrevet helt
nøjagtigt i henhold til den internationale lovgivning på området.
Faktisk er nærmest alt, der udføres på et containerskib, nøjagtigt
beskrevet. Hør her, hvordan en typisk arbejdsdag uden havneophold ser ud
for Søren Hoppe: Han står op kl. 7.15 og går op på broen156 og afløser
overstyrmanden fra kl. 7.30 -- 8.00, der så kan gå ned at få morgenmad.
Her får Søren så lidt føling med skibet, finder ud af, hvor de er henne,
og tjekker vejr og vind. Kl. 8.00 kommer 2. styrmanden så og afløser på
broen, og så går Søren ned til morgenmad. Herefter står den på
administrativt arbejde, såsom tjeklister og personaleadministration. A.
P. Møller - Mærsk har et computerprogram, der indeholder navnene på alle
deres søfolk. Når måneden er slut, taster Søren så alle oplysninger om
skibets personale ind i programmet. Oplysningerne kan handle om
overtidsarbejde, indkøb i kiosken eller salg af telefonkort. Søren skal
også bestille nye folk til skibet, for der er altid nogen, som skal af i
næste havn, og så skal der jo bestilles nye folk og flybilletter til
dem. Søren udfylder også søfartsbøger og hyrekontrakter, når der er
folk, som afmønstrer. Når der er nogle, som påmønstrer, skal der også
udfyldes forskellige papirer og tjekkes, om søfolkene er undersøgt af en
læge og har de rette kvalifikationer. Således er en stor del af Sørens
job at sørge for, at alle papirer er udfyldt i forhold til de
internationale love på området. Dette er meget vigtigt, for papirerne
bliver kontrolleret af havnemyndighederne, når skibet kommer i havn.

154

Stewarden er et besætningsmedlem, som ordner rengøring af værelser,
oprydning og hjælp i køkkenet mv. Hierarkiet er det samme som rangorden.
156 Broen er det øverste dæk på et skib, normalt på forreste
overbygning. Man navigerer og manøvrerer skibet fra broen. 155

46

Efter frokosten er det så tid til at koncentrere sig om kommunikationen
fra containerskibet og ud til den øvrige verden. Kommunikationen sker
via satellit157. For ude på havet er man ikke dækket ind af
telefonforbindelser og har heller ikke almindelig internetforbindelse.
Derfor har man et kommunikationsprogram, hvor skibsføreren samler alle
de meddelelser, der skal afsendes til den øvrige verden. Og en til flere
gange om dagen sørger han så for kontakt til en satellit, og sender
meddelelserne af sted. Samtidig modtager skibet meddelelser fra bl.a.
rederiet samt mails og nyheder på alverdens sprog, som så fordeles ud
til besætningens mailbokse, sådan at en filippiner modtager nyheder på
filippinsk, en polak nyheder på polsk osv. Efter at kommunikationen og
lidt andre forskellige opgaver er løst, er Sørens dag ved at være gået,
og det er tid til aftensmad og afslapning, før en ny dag starter\ldots{}
