\chapter{Verdenens omvæltning}

Tidligere var det også spændende for en sømand at tage varer og
souvenirs fra den store verden med til dem derhjemme, som måske aldrig
havde set sådanne ting før. I dag har globaliseringen betydet, at varer
fra den store verden bliver hurtigt og nemt transporteret hjem til os
med containerskibe. I disse dage tales der ofte om globalisering, men
hvad betyder globalisering egentlig? Ordet 'globalisering' er udtryk for
en udvikling, hvor virksomheder i stigende grad ser hele verden som
deres mest naturlige marked for at producere og sælge varer. Med andre
ord er globalisering et udtryk for, at verden er blevet mindre. Vi
handler stadig mere med hinanden over stadig større afstande. Og vi
rejser og samarbejder på tværs af kloden. Det er muligt, fordi den
teknologiske udvikling også har gjort transport og kommunikation både
effektiv, hurtig og billig. Men det betyder også, at danske virksomheder
møder skarp konkurrence fra stadig fjernere egne af verden.
Globaliseringen ændrer i disse år den måde, vi producerer og forbruger
på. Virksomhedernes produktion bliver i mindre og mindre omfang bundet
til bestemte steder, og produkterne sammenstykkes af produktion på mange
lokaliteter. Samtidig kræver de nye produktionsmønstre flytning af store
mængder råvarer som olie og korn, og det står containerskibene i høj
grad for. Søfart er i sin natur global og derfor en afgørende faktor for
globaliseringen. De større krav om sammenhæng og fleksibilitet i
transportsystemerne bevirker, at søfarten i stigende grad spiller sammen
med de øvrige transportformer, herunder særligt landevejstransport. De
store containerskibe sejler på alle verdens have og besøger mange
fremmedartede lande, som er kulturelt forskellige fra Danmark. Mads var
f.eks. ude at sejle, da Muhammed-krisen var på sit højeste. Det var et
farligt tidspunkt at være dansker i udlandet: ''Nu er vi langt om længe
nået til Europa, efter en lang og begivenhedsløs sejlads over Det
indiske Ocean, og en lettere nervepirrende tur gennem Suezkanalen, som
jo er flankeret af muslimske lande på begge sider (grunden til
nervøsiteten var jo al den ballade, der er fulgt i kølvandet på
muhammedsagen). Der skete dog ikke det helt store, men pga. af det
forhøjede sikkerhedsniveau, hejste vi ikke det danske flag -- som det
ellers er rutinen, når man sejler gennem Suez -- og vi måtte ikke gå i
land i Port Said, som er en havneby på middelhavssiden af Egypten.''.

165

Monotont er, når noget er ensformigt.

50

Hvor det tidligere var et privilegium for en sømand at komme ud at
sejle og se den store verden, har de fleste i dag mulighed for at rejse,
og vi får dagligt nyt fra hele verden gennem fjernsyn og internet.
Tidligere var det også spændende for en sømand at tage varer og
souvenirs fra den store verden med til dem derhjemme, som måske aldrig
havde set sådanne ting før. I dag har globaliseringen betydet, at varer
fra den store verden kommer til os, så f.eks. kan køre i japanske biler,
skrive på computere fra Kina og se på fjernsyn fra Korea. Dette kan lade
sig gøre, fordi det er billigt at få varerne transporteret hertil med
containerskib: ''Det er også fantastisk at tænke sig, at hvis du f.eks.
køber et fjernsyn og det koster 5000 kr., så er de 7-8 kr. af det er
transporten. Det er meget lidt, hvilket også betyder, at de fjerne dele
af verden sagtens kan konkurrere med os herhjemme, for det er ikke
transporten, der koster. Og det bliver kun billigere og billigere.''.
Stigningen i de mængder gods, som transporteres med containerskibe,
hænger i høj grad sammen med globaliseringen. Den endeløse række af
containerskibe, som bevæger sig hen over verdens have, og væksten i
antallet af nye containerkraner, er udtryk for en så afgørende ændring i
verdens økonomi, at det måske ikke er set siden den industrielle
revolution i begyndelsen af 1800-tallet. Den industrielle revolution var
en samfundsmæssig udviklingsproces, hvor landbrugssamfund blev forandret
til industriprægede bysamfund. Forarbejdelsen af råvarer skete tidligere
hovedsagelig som håndværk. Opfindelsen af især damp-, spinde- og
vævemaskinerne dannede fra 1700-tallet grundlag for en egentlig
industriel udvikling. Med brugen af elektricitet, olie og vandkraft tog
industrialiseringen yderligere fart efter 1870'erne. Mange af de
produkter, hvis fremstilling tidligere gav jobs til tusinder i både
store og små byer i Europa og USA, fremstilles i dag i Afrika, Asien og
Fjernøsten. Derfra sejler de så til Europa og USA. Og det ser ud til, at
denne udvikling fortsætter fremover, da et voksende antal virksomheder
flytter deres produktion til lande, hvor det er billigere at fremstille
varer. De danske, registrerede containerskibe har i dag en
bruttotonnage, som er halvanden gang højere end for 10 år siden, og det
er faktisk den højeste nogensinde. 5 \% af verdenstonnagen sejles nu i
containerskibe under dansk flag. Den 1. januar 2007 var de danske skibes
samlede bruttotonnage på 8,7 mio. BRT -- det højeste nogensinde. Danmark
har den syvende største handelsflåde i EU - kun overgået af Grækenland,
Malta, Cypern, Italien, Storbritannien og Tyskland. Pr. 1. januar 2007
er der i alt 4.986 dansk-registrerede containerskibe over 1000 BRT166.
Den store vækst inden for containerskibsfarten har også resulteret i
store ordrer på bygning af nye containerskibe. Der er således ordrer på
halvanden gang flere containerskibe, end der findes i dag. Stigningen i
omfanget af nye ordrer skyldes bl.a. vækst i verdenshandelen, herunder
især transporter til og fra Kina. Men det spiller også ind, at der i dag
stilles krav til skibenes sikkerhed og miljøvenlighed, som de ældre
skibe ikke kan leve op til. Skibsfarten og globaliseringen hænger nøje
sammen, og den fortsatte globalisering skaber nye muligheder. Derfor
bygger rederierne i disse år mange nye skibe, da globaliseringen gør, at
behovet for søtransport er stadigt stigende. Danmark er med helt i
front, for vi har en af de mest moderne flåder af containerskibe i
verden. Og dansk skibsfart er i dag blandt verdens mest
konkurrencedygtige. Dansk skibsfart transporterer 7-8 \% af tonnagen167
i verden og har op mod 10 \% af den samlede omsætning ved søfart på
verdensplan.

166 167

BRT er betegnelse for et skibs bruttotonnage, som er et rummål, der
omfatter samtlige af skibets lukkede rum.f Tonnage er et skibs
rumindhold, lasteevne eller vægt.

51
