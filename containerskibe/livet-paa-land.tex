\chapter{Livet på land}\label{livet-puxe5-land}

I de tidligere sejlskibsepoker var det for mange unge helt naturligt at
blive sømand, og hvis man var opvokset i et søfartssamfund, var der ofte
ikke så mange andre muligheder. I dag forholder det sig anderledes.
Mange af søfartssamfundene er forsvundet, og de unge kan vælge blandt et
hav af uddannelser. De fleste, som i dag vælger en maritim148
uddannelse, har ofte hverken baggrund i et søfartssamfund eller kendskab
til nogen, som er søfolk. En af dem er Søren Brask-Pedersen, som læser
til maskinmester: ''Jeg læste HTX oppe i Aalborg, og efter jeg var
færdig der, så skulle der jo ligesom ske noget nyt, og så fik jeg et
tilbud fra Mærsk om, at jeg kunne komme på Søfartsskole hernede (i
Svendborg, red.), og så ud at sejle bagefter, og så slog jeg til. Jeg
startede på søfartsskolen ude på Kogtved, og så da jeg var færdig der,
tog jeg på værkstedsskole for at få noget praktisk, og så kom jeg så ud
at sejle. Og så ville de jo have, at jeg skulle være skibsofficer inde
ved Mærsk, men det ville jeg ikke, så jeg sagde op, og så tog jeg ud at
sejle med Lauritsen for at få noget mere erfaring, så jeg kunne begynde
på maskinmesterskolen.''. Søren er ikke sikker på, om han vil være
sømand altid, men det gode ved en maskinmesteruddannelse er, at der også
er mulighed for at arbejde i land, hvilket jo f.eks. ikke er så nemt for
en styrmand. Der er i dag mangel på unge, som uddanner sig inden for det
maritime erhverv, så derfor gør rederierne en stor indsats 148

Maritim vedrører det, der har med havet eller søfarten at gøre.

43

for at få unge til at vælge en søfartsuddannelse. Containerskibene er
nemlig fyldt med moderne teknologi. Og en højteknologisk flåde kræver
søfolk, der er godt uddannede til at sejle de store skibe og håndtere
skibenes last. Al grundlæggende uddannelse af søfolk i Danmark foregår
på uddannelsesinstitutioner under Søfartsstyrelsen 149. De er offentligt
finansierede. Der findes 14 institutioner for søfartsuddannelser og to
skoleskibe. Hvert år afslutter ca. 250 unge en grundlæggende uddannelse,
mens ca. 450 afslutter en videregående uddannelse som skibsofficerer,
skibsførere, maskinmestre og skippere. Der findes et bredt udvalg af
uddannelsesmuligheder, hvis man har lyst til at stå til søs. Man kan
bl.a. uddanne sig til: • maskinmester • styrmand • skibsofficer •
skibsassistent • skibsmekaniker Eller man kan tage en ungdomsuddannelse
med maritimt præg: • HF-Søfart
