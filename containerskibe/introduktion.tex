\part{Containerskibe}\label{containerskibe}

Tidligere blev mange varer transporteret rundt i verden med skibe. Og
selv om mange måske ikke tror det, så er det stadigvæk sådan. I dag er
skibene dog blevet meget større og mere avancerede. De skibe, der i dag
transporterer varer rundt i verden, hedder containerskibe. De
transporterer faktisk omkring 90 \% af alt internationalt gods147. Et
containerskib sejler med store containere fyldt med alt mellem himmel og
jord såsom biler, frugter, kød, tøj, legetøj, fjernsyn, råstoffer og
meget mere. Containerskibe sejler i faste ruter over hele verdenen. Man
siger også, at de sejler i liniefart, hvilket er et udtryk for
regelmæssig trafik på bestemte havne. En virksomhed, som ejer og sejler
med skibe, kalder man for et rederi. Der arbejder ca. 30.000 mennesker
for de danske rederier, heraf næsten 13.000 i handelsflåden. Hele det
maritime erhverv, kaldet Det Blå Danmark, har ca. 100.000 ansatte. Kun
ca. 65 procent af de ansatte i handelsflåden er danskere, da
skibsassistener og dækspersonale ofte hyres i udlandet.
Containerskibsfarten bliver større for hver dag, der går, da produkter,
som tidligere blev transporteret på paller eller i løs vægt, i stadigt
højere grad transporteres i containere. Herudover produceres flere og
flere varer i Fjernøsten, for så at blive transporteret med
containerskibe til den vestlige verden. Containerskibet er egentlig ikke
nogen ny opfindelse, for allerede i 1955 kom amerikaneren Malcolm P.
Mclean, der ejede et stort vognmandsfirma, på en revolutionerende tanke:
Han begyndte at pakke alt gods fra sine kunder ned i store kasser af
samme størrelse, der blev kørt til nærmeste afskibningshavn og sat
ombord i et skib, der sejlede det til modtagerhavnen, hvor en anden
lastbil hentede det og kørte det ud til kunden. Mclean havde købt nogle
tankskibe fra 2. Verdenskrig, hvoraf Ideal X i 1955 blev bygget om til
verdens første containerskib. Herefter ændrede hele skibsfarten sig
meget, da skibene blev større og større og kunne sejle hurtigere, og der
var heller ikke længere brug for så mange søfolk til at sejle skibene.
Rederierne begyndte også mere og mere at arbejde i land og startede et
tæt samarbejde med havne, pakhuse og vognmandsforretninger. Derfor kan
rederierne i dag tilbyde deres kunder at transportere varer lige fra
producenten og til forbrugeren. Det første danske firma, der fik et
containerskib, var Østasiatisk Kompagni (ØK). Det skete i årene 1971-72,
hvor Nakskov Skibsværft leverede skibene Falstria og Meonia, mens
værftet B\&W leverede de mere end dobbelt så store Selandia og
Jutlandia. Herefter kom virksomheden A. P. Møller- Mærsk på banen, da
den i 1974 fik leveret sit første containerskib, kaldet Svendborg Mærsk.
I dag er A. P. MøllerMærsk verdens største containerrederi og ejer mere
end 500 containerskibe og omkring 1.400.000 containere verden over. A.
P. Møller - Mærsk ejer også verdens største containerskibe.
