\chapter{Fritid eller levetid?}\label{fritid-eller-levetid}

Tidligere forholdt det sig sådan, at søfolk var et særligt folkefærd,
som adskilte sig fra landkrabberne. Søfolkene var kendt for at komme ud
at opleve verden og for at hygge sig med sprut og damer, når de kom i
land. I dag minder livet om bord mere og mere om livet i land, og det er
forbudt at drikke alkohol på de fleste skibe -- man må heller ikke komme
beruset tilbage fra et ophold i land. Det er også blevet forbudt at have
damer om bord; men besøg hos prostituerede, når man er i land, findes da
stadig. Mads fortæller her en historie om en tur i land i Busan i
Sydkorea: ''En tur i land blev det også til, og da jeg var den eneste
aspirant, der havde tid og lyst til det, smuttede jeg ind sammen med
stewarden og stewardessen, 2 matroser, elektrikeren og motormanden, som
alle sammen er filippinere. Og jeg fandt hurtigt ud af, at filippinere
ikke er til at styre på 10 tønder land, når de først er kommet op i
omdrejninger. Så vi kom hurtigt ind til Texas Street, som er\ldots{} ja,
lad os bare sige at den ikke ligefrem er dydigheden selv. Jeg følte ikke
lige selv noget behov for at købe nogen\ldots{} lad os kalde det
ydelser, så stewardessen, elektrikeren (som ikke er til kvinder) og jeg
morede os blot med at se de andre styrte rundt med dollarsedlerne
siddende meget løst. Jeg havde dog en lidt ubehagelig situation på et
tidspunkt, for lige pludselig stod der en kampvogn af en russisk kvinde
foran mig, ca. 2 meter høj, og lige så bredskuldret som Arnold
Schwarzenegger, og som så absolut mente, og jeg skulle gå på hotel med
hende, og var meget insisterende i sin sag. Jeg fik heldigvis snoet mig
udenom ved at pege i den ene retning og se overrasket ud, hvorefter jeg
smuttede den anden vej, mens hun kiggede. Det var sgu en skræmmende
oplevelse, men også meget sjovt bagefter.''. Søfolkene arbejder det
meste af tiden, når de er ude at sejle, men der er da også lidt tid til
et socialt liv. Det sociale live foregår mest om aftenen, hvor søfolkene
ser film og sidder og snakker. Og når man så kommer i havn, går man
måske på en cafe eller på opdagelse. Der er altså muligheder for at
opleve store dele af verdenen og møde andre kulturer. Sømandskirken159
og Handelsflådens Velfærdsråd160 er to organer, som tager initiativ til
forskellige aktiviteter i forhold til søfolkenes sociale liv:
''Sømandskirken er ikke i alle havnebyer, men så kan man altid gå til en
norsk eller svensk sømandskirke. Sømandskirken er en god guide at have,
når man er i en by, man ikke kender, for så ved de altid hvad der er
værd at se, og hvad for nogle områder af byen, man ikke skal gå i, så vi
bruger det meget. Søfartens velfærdtjeneste sender bøger og film ud til
skibene og sørger for, at der ligger aviser, så vi får danske aviser om
bord. De optager også, hvad der har været af dokumentarfilm og sådan i
fjernsynet og sender ud. Vi får også nyheder fra Danmark på e-mail, og
det er et af dagens højde-punkter, når man kan læse, hvad der er sket
herhjemme. Så er man ikke helt overrasket, når man kommer hjem over,
hvad der er sket''.

159

Sømandskirke er en kirke, der betjener søfolk på deres eget sprog i
udenlandske havne. Handelsflådens Velfærdsråd er en privat, selvejende
institution, der er oprettet i 1948. Den skal varetage og fremme
initiativer vedrørende søfolks velfærd. Aktiviteterne omfatter bl.a.
idræts- og undervisningstilbud, nyhedsformidling, film og bøger. 160

48
