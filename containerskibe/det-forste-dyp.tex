\chapter{Det første dyp}\label{det-fuxf8rste-dyp}

I dag kommer de fleste unge ud at sejle med containerskib for første
gang som et led i deres søfartsuddannelse. Det er et stort skridt at
tage at forlade de faste hjemlige rammer for i stedet at skulle sejle ud
på de store verdenshave. Man skal sige farvel til familie og venner for
et stykke tid og i stedet bo på et stort skib med sine helt egne regler
og egen livsstil. Når man tager ud at sejle, er der mulighed for at
komme ud i verden og opleve andre ting, end man måske ellers ville komme
til, og det er mange tiltrukket af. Det fortæller Søren Hoppe om her:
''Min bedste kammerat under barndomsperioden, han hedder for øvrigt også
Søren, han havde en storebror, som sejlede, og han fortalte sådan nogle
spændende historier fra Østen og Grønland, så jeg fik lyst til at prøve
det og tænkte, at det kunne måske være meget godt at prøve. Der var ikke
rigtig andet, der trak. Der var jo oceaner af muligheder, og det var
bare at vælge, og det var så, hvad jeg have mest lyst til på det
tidspunkt.''. Før man kan komme ud at sejle, skal man gennemgå en
lægeundersøgelse150 og evt. have en søfartsbog151. Når lægeundersøgelsen
er bestået, er man klar til at sige farvel til landlivet. Søfartsbogen
er et dokument, der af nogle lande anvendes som legitimation ved ind- og
udrejse for søfolk i stedet for visum, det vil sige et dokument, der
giver tilladelse til at rejse ind i et land. Søfartsstyrelsen udsteder
kun søfartsbøger til danske statsborgere over 16 år. Alle danske søfolk,
der påmønstrer en stilling, som er omfattet af et skibs
sikkerhedsbesætning, skal have en søfartsbog. Søfolk under uddannelse er
dog ikke en del af et skibs sikkerhedsbesætning -- det vil sige, hvor
mange 149

Søfartsstyrelsen er en styrelse under Økonomi- og Erhvervsministeriet,
som varetager sager vedrørende søfartserhvervet, bl.a. sikkerhed,
arbejdsmiljø og erhvervspolitik. 150 Alle som arbejder om bord i et
dansk handelsskib skal regelmæssigt gennemgå en lægeundersøgelse. Søfolk
under 18 år undersøges årligt, mens søfolk over 18 år normalt skal
undersøges hvert andet år. I Danmark skal lægeundersøgelsens foretages
af en søfartslæge, som Søfartsstyrelsen udpeger. Ingen kan komme ud at
sejle uden et gyldigt sundhedsbevis. Det er det officielle bevis for, at
man er sundhedsmæssigt egnet. 151

44

besætningsmedlemmer der minimum skal være om bord på skibet.
Sikkerhedsbesætningen skal være oplyst for at sikre, at der er mand nok
om bord til at sejle skibet forsvarligt, og for at sikre sikkerheden for
menneskeliv på havet. Det er Søfartsstyrelsen, som laver den såkaldte
besætningsfastsættelse. En besætningsfastsættelse er et dokument, som
indeholder oplysninger om det krævede antal besætningsmedlemmer i et
skibs drifts- og/eller sikkerhedsbesætning og de enkelte
besætningsmedlemmers kvalifikationer. Det foregår for det meste ved, at
man tager et fly til en lufthavn i nærheden af, hvor det skib befinder
sig, som man skal påmønstre. Her hentes man så af en agent152, der fører
en til skibet. Her fortæller Mads om sin oplevelse: ''Hej alle sammen!
Her er så min første mail fra Ivar Lauritzen. Som de fleste af jer nok
ved, kom jeg først ombord i mandags, selvom jeg egentlig skulle have
været af sted sidste fredag. Men af uvisse årsager kom jeg altså først
af sted lidt senere. Så mandag middag, 8 timer efter afgangen fra
Svendborg, stod jeg så i Amsterdam lufthavn og ventede på rederiets
agent, der skulle køre mig til Rotterdam havn, hvor skibet lå.
Imidlertidig var der ikke noget med et skilt med mit navn eller noget i
den dur i syne, og jeg regnede med, at han bare var lidt forsinket. Men
da han stadig ikke var kommet efter 20 minutter, begyndte jeg at blive
en smule nervøs. For selv om jeg da ikke er totalt hjælpeløs, er det jo
aldrig fedt at være strandet i en ukendt lufthavn uden at vide, hvordan
man kommer hverken fra eller til. Men heldigvis nåede jeg lige at
opfange navnet Ivar Lauritzen i en samtale mellem to mænd, der kom
gående forbi. Det viste sig så at være agenten og en anden dansker, der
skulle ud på det samme skib som mig. Agenten havde bare fået at vide, at
han kun skulle hente én person, nemlig ham den anden. Nå! Men efter lidt
snakken frem og tilbage kom jeg da med. Så det var jo heldigt nok, må
man sige! Ellers havde jeg nok stadig bare stået i lufthavnen og ventet.
Da jeg kom til skibet, havde jeg været ombord i ca. 3 minutter og 17 et
halvt sekund, da jeg blev sat i arbejde. Der skulle tages stores153
ombord, så jeg skulle bare i gang med at hejse kasser fra kajen og op på
skibet og sætte det ned i maskinen eller rundt omkring på dækket. Så
mine første 4-5 timer ombord fik jeg ikke meget tid til at pakke ud og
kigge mig lidt omkring. Men det gjorde mig ikke det store, jeg skal jo
være her i 3-4 måneder, så jeg får masser af tid til at kigge på skibet.
Til dem, der er interesseret, så er skibet 165 meter langt, har plads
til ca. 1500 paller bananer, 900 brugte biler eller 216 containere. Den
er bygget i 1990, og blev dengang betragtet som prototypen på det nye
årtusinds skib. Så selvom hun er en dame på 17 år, er alting her ombord
stadig forholdsvis moderne. Men hun er stadig gammel nok til at være så
slidt, at der skal bruges en hel masse små tricks for at få ting til at
fungere ordentligt, hvilket i mine øjne bare bidrager til charmen ved at
være ombord. Så alt i alt befinder jeg mig allerede rigtig godt her
ombord, og besætningen er nogle folk, jeg kan rigtig godt med.''

152

Agent eller shippingmedarbejder, er en person, som planlægger
transporten af gods og varer til søs. For en agent gælder det om at
udnytte skibenes kapacitet bedst muligt. Agenten sørger for
papirarbejdet i forbindelse med transporten og tager sig af de praktiske
forhold i forbindelse med havneanløb, lastning og losning. 153 Stores er
et udtryk for varer.

45
