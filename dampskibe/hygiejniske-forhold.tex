\chapter{Hygiejniske forhold}\label{hygiejniske-forhold}

''Forandringen fra sejl til damp virkede ligefrem tiltalende.
\word{Frivagten} var frivagt, enten det så drejede sig om godt eller
dårligt vejr.  Rejsens varighed kunne nogenlunde beregnes på
forhånd\ldots{} \word{provianten} var friskere og vand\word{rationerne} sjældnere''.

Søloven angav nærmere regler for mandskabets bespisning.
Kostreglementerne indeholdt arten og mængden af de fødevarer, søfolkene
kunne gøre krav på. Som regel serveredes tre hovedmåltider, hertil varm
eller kold kaffe, te eller vand. Datidens teknik gjorde det vanskeligt at
holde madvarerne friske, så hovedbestanddelen i kosten var brød og saltet
kød. 

Dampskibene var ikke så afhængige af vejr og især vind for at sejle.
Dette bevirkede, at det f.eks. var meget lettere at beregne, hvor længe
en tur ville vare og dermed også, hvor meget proviant man skulle have
med. Kosten i dampskibene blev regnet for meget bedre end i sejlskibene.
Dampskibenes indretning gjorde tilberedningen af maden nemmere. Den tid,
da skibets yngste besætningsmedlem tilberedte maden i et lille skur
midtskibs, var ved at være forbi. Der blev ansat
\wordd{hovmestre}{Hovmester er den /verste af kokkene om bord.}, kokke og
køkken-personale til udelukkende at klare det arbejde. 

Hovmester er den øverste af kokkene om bord. 88
Beskøjter er hårde kiks, der er langtidsholdbare.

\begin{quote}
    
    ''Jeg befandt mig ellers godt om bord i et dampskib. Kosten var meget
    bedre, end hvad jeg var vant til fra sejlskibene og jeg mønstrede med
    ØKs pionerskib S/S SIAM Det var det første skib, ØK satte i fart. Jeg
    gjorde to rejser med dette skib. Der var stor forskel på
    kostforplejningen. I NORMANNIA måtte vi gnave
    \wordd{beskøjter}{Beskøjter er hårde kiks, der er langtidsholdbare.},
    men i SIAM fik vi hver dag frisk og blødt brød, og der var ikke noget
    der hed \word{ration}. I SIAM kunne vi få lige så meget, vi havde
    brug for.  I NORMANNIA fik vi udleveret ration om lørdagen, medens vi
    i SIAM kunne forsyne os hver dag hos hovmesteren. Man kunne her om
    bord mærke, at ØK havde til sinds at forbedre den
    \wordd{menige}{Menig er et besætningsmedlem, der ikke er officer.}
    sømands kår, og tiden har jo vist, at ikke alene med kosten, men også
    med ordentlige opholdsrum og moderne installationer som baderum og
    spisemesse.''

\end{quote}

Selv om det var lettere at planlægge rejsen og dermed behovet for
madvarer,
kunne man ikke gardere sig mod besætningsmedlemmer, der for egen vinding
svindlede med \word{provianten}. En styrmand havde et horn i siden på kaptajnen
og brugte \word{provianten} som et middel til at prøve at få kaptajnen fyret:

\begin{quote}
    
    ''Sammen med en rigelig kulbeholdning tog vi grøntsager, frugt og kød
    om bord og afgik nogle timer senere til Marseille. Efterhånden som vi
    nærmede os Gibraltar, forstod jeg, at der var et eller andet i gære.
    Det foregik skjult, og man skiltes, så snart jeg nærmede mig. Jeg
    kaldte første-styrmand ned i salonen en dag og spurgte: Hvad er De
    ude efter?  Han svarede, at vi snart var uden mel, kaffe, the, sukker
    og lignende, og at jeg ikke havde sørget for at komplettere disse
    varer i Las Palmas.  Hvorfor havde De skibets førstestyrmand, ikke
    varskoet mig i Las Palmas?  For øvrigt forstår jeg ikke, vi skal have
    \word{proviant}, mel, konserves, smør, margarine og så videre til en
    halv måned. Hvor er det blevet af?  At løbe kort for \word{proviant},
    når der har været mulighed for at komplettere, er en alvorlig sag for
    en fører''. 

\end{quote}

Det viste sig ved senere forhør, at der var blevet stjålet og solgt af
provianten, og styrmanden havde vidst besked. Indkvarteringsforholdene
var derimod ofte elendige.  Mandskabet blev ofte bespist og sov i lukafer
i stævnen: 

\begin{quote}
    
    ''\word{Lukaferne} er jo i langt de fleste tilfælde alt andet end svarende
    til moderne hygiejne- og hyggefordringer. De små rum, de iskolde
    vægge, de gammeldags klodsede, upraktiske køjer, de så ofte smækvåde
    gulve, snart fugtig, snart kold luft, snart uhyggelig overophedning.
    Selv en så simpel og for en \wordd{fyrbøder}{Fyrbøder er en person,
    der er ansat i maskinen til at passe fyret og skovle kul på fyret
    under kedlerne. En fyrbøder er ikke faguddannet, ofte har han starter
    sin karriere som kullemper.} nødvendig ting som et badekar
    findes kun sjældent. Et sove- og et opholds- og spiserum hører for
    ikke farende ind under de simpleste selvfølgeligheder. Man ser at
    lige nybyggede dampere har dårligere mandskabsindretninger end ældre
    byggede.''

\end{quote}


\begin{quote}
    
    ''Vi atten \word{fyrbødere} boede og spiste i det samme rum. Skabe var der
    ingen af, men mellem køjerne, der var anbragt i to etager langs
    rummets sider og på tværs helt forude, var der knager hvorpå vi kunne
    hænge vort tøj. Midt i rummet var der et langt spisebord, og hele
    dette arrangement medførte, at de folk, der skulle ud af køjerne ved
    vagtskifte og til måltiderne, brugte bordet som trappe. Det var jo
    ikke ligefrem appetitvækkende, når man sad og spiste, og der kom et
    par bare og ikke helt rene fødder ned ved siden af bliktallerkenen,
    men det var en uundgåelig ting, som ingen fandt på at gøre vrøvl
    over.''

\end{quote}


Officererne havde ganske andre forhold.
De havde kahytter, der var placeret enten agters eller midtskibs. I de
mindre lastdampere logerede kaptajnen ofte i ensom majestæt - eventuelt
med et mindre rum ved siden af til styrmanden. 

\begin{quote}
    
    ''I skærende kontrast til mandskabslukaferne forude stod kaptajnens
    kahyt. Udstyret med bord og stole, køje og \wordd{chatol}{Chatol er
    et skuffemøbel.}, lys, luft og
    måske nogle dekorative indslag i form af billeder, porcelæn eller en
    grøn plante var kahytten ikke uden lighed med stadsstuen i mangt et
    skipperhjem.''

\end{quote}


Men i passagerdamperne var kaptajnen og de øvrige officers kahytter ofte
i forbindelse med passagerkahytterne.

 Fyrbøder er en
person, der er ansat i maskinen til at passe fyret og skovle kul på
fyret under kedlerne. En fyrbøder er ikke faguddannet, ofte har han
startet sin karriere som kullemper. 91 Chatol er et skuffemøbel. 90
