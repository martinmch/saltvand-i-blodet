\documentclass{book}

\newcommand{\word}[2]{ \blue #1 #2 } 

\usepackage{xcolor}
\begin{document}

\part{Dampskibsepoken}

%\begin{intro}
En damper - eller et dampskib - er et fragtskib, der bliver drevet frem af en
dampmaskine og er bygget af jern eller stål. I den allerførste tid blev der også
bygget dampskibe af træ.
%\end{intro}

%![Nanna af Svendborg](images/sejlskibe_indledning-nanna.jpg)

Dampfragtskibe begyndte at gøre sig gældende fra omkring 1875 til 1900. Ganske
vist var der før den tid så småt begyndt at komme dampere i indenrigs- og
nærfarten, men det var hovedagelig til passager- og postruter og faste
fragtruter. I 1870 var der således kun 4 danske dampskibe i
\word{trampfart}{Trampfart er, når et skib sejler i fri fart mellem forskellige
havne, hvofra og hvortil der skal transporteres varer. I modsætning hertil er
der skibe, der er placeret i liniefart med gods mellem to eller flere bestemte
havne.}

%![Besætning på et sejlskib. Drengen holder skibshunden.](images/sejlskibe_indledning-skibsh.jpg)

Sejlskibe var heller ikke helt så punktlige, og så kunne det tage lang tid at **laste**[^1] og
**losse**[^2] skibet.  Det kunne også være temmelig risikabelt at sejle med disse sejlskibe.
Mange sømænd forsvandt sporløst, hvis et skib for eksempel ramte et isbjerg eller
kæntrede, når lasten forskubbede sig i hård søgang. Alligevel kunne det betale sig at
søsætte nye sejlskibe. Hver gang skipperen eller kaptajnen afleverede en ladning sikkert
i havn, fik skibets ejere oftest et fint overskud, selvom priserne naturligvis svingede,
og nogle år var bedre end andre. Mange redere brugte overskuddet til at bygge nye skibe.
På den måde tjente de flere penge - og så kunne de købe endnu flere skibe. Nogle områder
og byer i Danmark specialiserede sig i sejlskibssøfarten. Den var ganske vist risikabel,
men gav også mange mennesker brød på bordet hver dag. Et af disse områder lå i Det
sydfynske Øhav.

[^1]: **Laste** er, når varer eller gods tages om bord på et skib.
[^2]: **Losse** er, naar varer eller gods tages fra borde paa et skib.


\chapter{Livet på land}

 _"Vi legede med små træskibe og planker om sommeren og vadede rundt og skubbede til dem,
hvis det ikke gik stærkt nok."_

%![Tømmerstabler på Faaborg Havn. ](images/sejlskibe_tema-1-toemmer.jpg)

På havnen var der altid noget at se, når de store sejlskibe kom ind. Hestevogne hentede de
mange varer, som skibene havde med. Og søfolkene fortalte historier fra fremmede lande.
Havnen var spændende og satte fantasien i gang. Ville man ud at opleve verden, måtte man
sejle af sted, for der var ingen andre muligheder. Sejlskibene satte derfor gang i drømmen
om den store verden. Voksede børnene op i en søfartsby fjollet eller i en **havneby**[^3], var det ikke
usædvanligt, at de legede på havnen, selv om det ikke altid var ufarligt.

%![Marstal var en af de byer, der i 1700-tallet blev en betydningsfuld søfartsby.](images/sejlskibe_tema-1-marstal.jpg)

\sFram{showBlue}{
    \subsection{Søfartsby}

    En søfartsby var en by, der lå ved vandet. I byen levede befolkningen mest
    af søfarten. I Danmark lå mange af disse søfartsbyer på øer, hvor sejlads
    under alle omstændigheder var vigtig. I søfartsbyerne var der en stor del
    af den mandlige befolkning, som sejlede. De kom ofte hjem om vinteren,
    inden stormene blev for kraftige, og isen lukkede farvandene. Om vinteren
    var familien derfor samlet, og man fik ordnet de praktiske ting. Det var
    gerne i denne periode, at giftermål blev indgået og pengene gjort op. Der
    blev festet og levet, inden isen brød op, og mændene igen skulle ud på
    bølgerne. I nogle søfartsbyer blev der udviklet forskellige skikke, der
    kunne være meget anderledes end dem, man kendte fra andre byer.

    Når mændene var ude at sejle, tog kvinderne sig af de økonomiske sager og
    alt det, der havde med hjemmet at gøre. At være sømandskone var noget helt
    specielt på linje med det at være sømand.

    Nogle rigtige og vigtige søfartsbyer herhjemme var Marstal på Ærø, Troense
    på Tåsinge, Dragør på Amager, Nordby og Sønderho på Fanø. Der var også
    mange sejlskibe og søfolk i byer som Faaborg, Ærøskøbing, Svendborg og
    Rudkøbing. Tilsammen udgjorde det sydfynske øhav et center for sejl-skibe,
    hvor der i sejlskibstiden var omkring 1/3 af alle danske sejlskibe.
}

_”Undertiden legede vi på tømmerstablerne på Faaborg Trælast. Vi klatrede op og sprang fra
den ene til den anden. Når så **havnefogeden**[^4] kom, fik vi øretæver, hvis han fik fat i nogen.
Ellers var vi meget på havnen. Jeg måtte ikke løbe til havnen uden at få lov. Men jeg løb
sgu derned alligevel og så vankede der stryg, hvis far opdagede det. Det var alle skibene
i havnen der trak, og vi kom i snak med skibsdrengene, der kunne fortælle historier og
vise uartige billeder. Der var ikke ret mange drenge, der holdt sig dernede fra.”_

 _”Vi drillede havnefogeden. Det var gamle Noack. Når havnefogeden var efter os, så kravlede
vi ned mellem pælene og ind på **glaciset**[^5], så kunne han sgutte få fat i os. Vi lavede mange
numre, og så kom Noack hen til den gamle og sagde: De satans knægte!”_

Den daglige færden på havnen gav mulighed for at hjælpe mændene på skibene. Hvis man var
rigtig heldig, kunne man få lov at kravle op i skibenes master.

 _”Der kom mange 3-mastere til **Bagenkop**[^6] med kul. Så ville vi jo gerne op at hjælpe til,
sejlene skulle tørres osv. Vi fik lov på den betingelse, at vi skulle hente hver 10 spande
vand om dagen og komme vandfad på dækket. Hvis vi så hentede 10 spande vand om dagen, så
måtte vi kravle op i riggen.”_

\sFram{showBlue}{
    \subsection{Riggen}

Rigningen er den del af sejlskibet, som sejlene hejses op i. Rigningen er bygget op omkring skibets master og består af en masse tovværk. At kravle til tops i rigningen var et farligt og besværligt arbejde og derfor ikke altid populært blandt søfolkene.
}

%![Riggen på et sejlskib.](images/sejlskibe_tema-1-riggen.jpg)

 _”Vi drenge vi lå jo og kravlede i rigningen på de skibe, der lagde op om vinteren. Vi
skulle se hvem der kunne komme hurtigst op og lægge sig på maven oppe på fløjknappen. Ja
Gud fri mig vel $\ldots$ at der ikke skete en ulykke! Vi var som aber dengang. Det var noget alle
drenge gjorde.”_

Man fik et naturligt forhold til at sejle, hvis man voksede op ved vandet.

_”Vi var dårligt nok kommet ned på havnen, så lå vi nede i en jolle og roede jo. Ja der var jo altid nogen
større med, som havde forstand på det, indtil vi andre havde fået sat det rigtigt sammen,
så kom det. Det var både med sejljoller og rojoller.”_

Nogle blev indført i jollelivet af nødvendighed. Ikke alle steder var der råd til at lade
børnene bruge tid i skolen.

 _”Vi boede i et fiskerleje nord for **Assens**[^7] – Aborg Strand. Min far havde temmelig gode
kundskaber, og han søgte skolevæsenet om, at vi ikke kunne slippe for at gå i skole, for
vi var nødt til at være hjemme for at hjælpe dem, imod at han underviste os, og vi mødte
hvert år til eksamen sammen med de andre børn og kunne klare os. Når vi lå derude og
slæbte ålevåd om natten, så lå vi og drev sådan som omstændighederne var i tre kvarter til
en time – så sad vi rigtignok nede i logaret og terpede, så stak den gamle lige hovedet op
af kappen en gang imellem for at se om der var noget usædvanligt. For ellers så sad han og
terpede med os dernede. Om det så var tysk, så kunne vi det.”_

\sFram{showBlue}{
    \subsection{Jollen}

En jolle er et af de mindste fartøjer, man kan sejle i. Jollerne bliver enten
drevet frem af sejl eller årer. Man brugte jollerne til mange forskellige
mindre opgaver. Skulle man ud til et skib, der lå for anker, brugte man jollen
til at ro ud til det. Joller var også det foretrukne transportmiddel til
kortere ture mellem øer eller måske i havne og lignende. Større joller havde
fiskerne glæde af, når de skulle røgte garn eller ruser. Joller blev desuden
brugt som redningsbåde, selv om de ikke var særligt egnede til det. Indtil
1960’erne var det bådebyggerne, som lavede de fleste joller. De blev bygget i
hånden og var lavet i træ. Senere blev jollerne bygget af glasfiber, og
en del af disse var masseproducerede. I dag er de fleste joller forsynet med
motor, og man ser kun sjældent en jolle, der bliver roet.
}

I søfartssamfundene lå det ligefrem i luften, at søen ventede forude.

 _"Selvfølgelig færdedes vi børn ved havnen, så snart der var lejlighed til det, og tumlede
os i joller, og dengang var der jo en hel del sejlskibe hjemmehørende her i **Ærøskøbing**[^8]
og det var jo en begivenhed at komme ned at se, hvis de kom hjem om efteråret og
havde været ude siden marts måned. Så kom de gerne hjem med en ladning kul eller
**ballastet**[^9] hjem for at lægge op, og så var vi jo ved jollen, når de var fortøjet
og roede omkring i den. Og selvfølgelig fik vi lyst til søen. I **Marstal**[^10] var det
jo sådan, at hvis en dreng ikke ville til søs, når han var konfirmeret, så var det
jo ikke nogen rigtig dreng. De få, der kom i håndværkslære eller sådan noget, de
var ikke rigtig anset, og der var noget af det samme her, men ikke så udtalt."_

%![Børn ved havnen. ](images/sejlskibe_tema-1-boern.jpg)

Nogle piger syntes også, at havnen var et spændende sted. Men omgivelserne var
ikke altid klar til at lade piger løbe rundt på havnen.

 _”Jeg har været en ti til tolv år, da jeg var sluppet ned på havnen og ivrigt
iagttog en skonnert blive losset. Jeg havde hørt i skolen, at der var nogle tyrkere
med om bord, og dem ville jeg gerne se, hvordan de så ud. De drejede det tunge
lossespil med de tunge kurve på. Jeg stod bare i cirka 15 meters afstand og kiggede
på dette. Det har ikke varet længere end et kvarter eller tyve minutter. Længere
tid havde jeg ikke til min rådighed. Jeg havde en gammel dame at gå byærinder for
efter skoletid. Om aftenen mødte en tante op. Oh, ve, nogen havde bedt hende gå
til min mor og fortælle, at jeg løb på havnen og var ude efter fremmede søfolk.
Jeg havde en fornuftig mor, som bad mig om at blive derfra, så hun kunne blive fri
for sådanne sladderhistorier, men jeg kom aldrig på havnen alene mere.”_

%![Et lossespil i brug.](images/sejlskibe_tema-1-lossespil.jpg)

\sFram{showBlue}{
    \subsection{Lossespil}
Lossespil var en mekanisme, som gjorde det nemmere at laste og losse skibet - altså at putte varer ned i skibet eller at tage varerne op fra skibet. Når man drejede håndtagene på lossespillet, trak man samtidig i et stykke tov, der var forbundet med lossehjulet. På lossehjulet var der en krog. På den kunne man fastgøre kurve eller kasser fyldt med gods. Det tog en uge at laste et skib med et hånd-drevet lossespil. Efterhånden blev lossespillene motoriserede med små damp- eller benzinmotorer. I dag bruger man ikke lossespil mere, men faste krananlæg eller mobilkraner.
}

[^3]: **Havneby** er en by med en større havn
[^4]: **Havnefoged** er en mand, der er ansat til at styre havnen.
[^5]: **Glacis** er en stenvæg, der går skråt ud i vandet.
[^6]: **Bagenkop** er et fiskerleje på sydspidsen af Langeland.
[^7]: **Assens** er en by på Fyn.
[^8]: **Ærøskøbing** er en by på Ærø.
[^9]: **Ballast** er sten eller sand, som man puttede i bunden af et tomt skib,
      så det ikke væltede.
[^10]: **Marstal** er en søfartsby på øen Ærø, der ligger syd for Fyn.

\chapter{Far-vel}

 _"Jeg var lige konfirmeret og så skulle jeg ud med den gamle Petersen, det var vores nabo.
 Han boede ved siden af far og mor, og det var vist aftalt længe før jeg blev
 konfirmeret.”_

%![Konfirmander fra 1914.](images/sejlskibe_tema-2-konfirmand.jpg)

I **sejlskibstiden**[^11] havde man ikke mulighed for at vælge mellem uddannelser og beskæftigelse
som i dag. Drengene kom meget ofte til at lave det samme som faderen, mens pigerne
skulle lære at holde hus ved at tjene på gårdene eller i større husholdninger. Her
skulle de være, til de var **giftemodne**[^12]. Konfirmationen var et afgørende skel. Det
omgivende samfund anså herefter én for at være klar til arbejdslivet. Når man blev
konfirmeret, gik der ikke længe, før leg blev til alvor. Familierne var børnerige og
havde som regel ikke ret mange penge. Når barnet blev sendt af sted, var der en mund
mindre at mætte. Hvis man var dreng i et søfartssamfund, eller hvis ens familie var
beskæftiget i **søfartserhvervet**[^13], så var der gode chancer for, at man selv skulle til søs.
Tradition spillede en stor rolle, og mange gange var det allerede aftalt på forhånd,
hvilket skib den unge skulle sejle med. Nogle gange kunne det gå hurtigt, og før
konfirmanden vidste af det, kunne han være på vej til skibet.

 _”Konfirmationsfesten skulle være om onsdagen, da vi skulle til alters, og **da var jo lagt
stort i kakkelovnen**[^14]. De skulle jo være der alle sammen, når vi kom hjem fra den
kirkehistorie der, og skulle spise, og det gik fint med det, og så der klokken 5 om
eftermiddagen, da kom der sgu et telegram at jeg skulle rejse klokken 5. Jeg havde fået
**hyre**[^15].”_

Arbejdsaftaler med skipperen eller ejeren af skibet var tit indgået på forhånd, men før
man kunne komme om bord, skulle man lige ses an.

 _”Jeg gik med min bedstemor derned, og hun spurgte, om han havde brug for en dreng. Ja,
det har vi, sagde han, er det ham der? Ja, det var det jo. Ja vi kan godt prøve at tage
ham med. Og så skulle jeg have 10 kr. om måneden, og hvis jeg så duede til noget skulle
jeg have 15 kr.”._

I søfartssamfund blev konfirmationen afholdt tidligere, så drengene kunne nå at sejle ud
med skibene, lige så snart **_isen_** begyndte at smelte i slutningen af marts eller i
begyndelsen af april:

%![En mergelspiger til splejsning af reb, og åbning/lukning af sjækler.](images/sejlskibe_tema-2-isen.jpg)

\sFram{showBlue}{
    \subsection{Isen}
I 1800-tallet og første halvdel af 1900-tallet var klimaet koldere end i dag. Det betød, at de indre farvande ofte frøs til med is. Isen gjorde det umuligt at sejle med sejlskibe. Farvandene begyndte at ise til i løbet af januar, og i nogle år skulle man helt hen til april, før man kunne genoptage sejladsen. Oftest kunne sejlskibene dog sejle ud i løbet af marts måned.}

 _”Vi havde altid konfirmation 14 dage før andre havde. Så sejlede vi til **Marstal**. De
skibe der havde været hjemme og overvintre og blev repareret om vinteren, de skulle
gerne ud senest i april og så skulle konfirmationen være overstået først. Det var
**hyrebasser**[^16], det kaldte vi dem dengang, altså privatfolk, der **forhyrede**[^17] søfolk. De havde
gerne sådan en lille søudrustningsforretning ved siden af med olietøj og søstøvler og
forskellige knive, hvad en sømand skulle have, merlespir og sejlhandsker og sådan.”_

%![En mergelspiger til splejsning af reb, og åbning/lukning af sjækler.](images/sejlskibe_tema-2-merlespige.png)

%![Sejlhandsker til at beskytte hænderne mod slid.](images/sejlskibe_tema-2-sejlhandsk.png)

\sFram{showBlue}{
    \subsection{Merlespiger}
Merlespigeret var en aflang spids lavet af metal eller træ. Merlespiret var en naturlig del af sømandens udstyr. Merlespiger blev brugt til at splejse tovværk, det vil sige at sætte forskellige stykker tovværk sammen, så man kunne få den rigtige længde. Man splejsede også tovværk, hvis man skulle indsætte et øje – altså et hul eller en ring, man kunne trække noget igennem eller hægte noget fast i. Splejsning kunne også bruges til at forstærke tovet med.
    \subsection{Sejlhandsker}
Sejlhandskerne var lavet af læder og beskyttede hånden mod slid. Man brugte handsken, når man skulle sy et nyt sejl eller reparere sejl, der var gået i stykker. På sejlhandsken var der indsyet en brik, som støttede nålen. I land var det at være sejlmager et erhverv, mens det på skibene oftest var de ældre matroser, der fungerede som sejlmagere.
}

Udrustningen blev der mange gange sørget for hjemmefra. Køjesækken og skibskisten var
vigtige. Skibskisten blev gerne pakket af moderen. Og konfirmanden kunne regne med at få
en god og fornuftig udrustning med sig med tøj til alt slags vejr, hvis familien var
vant til at beskæftige sig med arbejdet på søen. Selve mønstringen om bord foregik på
den måde, at den nye dreng fik anvist en køje forude - og så blev han straks kastet ud i
arbejdet. Allerede på det tidspunkt gik sølivets barske realiteter op for den unge, og
nogle af barndommens drømme er nok hurtigt fordampet op i den blå luft.

%![To sømænd med deres køjesække over skulderen.](images/sejlskibe_tema-2-koejesaek.jpg)

\sFram{showBlue}{
	\subsection{Køjesæk}
Køjesækken var en lærredssæk med ringe slået i foroven, så den kunne snøres sammen. Sommetider malede ejermanden sit navn på sækken, så han lettere kunne kende den. Andre malede billeder på sækken – f. eks. to krydsede flag. Køjesækken blev brugt til at opbevare sømandens dyne, hoved-pude, dynebetræk og undertøj. Det var det, man kaldte for ”køjetøjet”. Ikke alle søfolk havde råd til en dyne, så de måtte nøjes med et tæppe.
}

[^11]: **Sejlskibstid** er dengang, sejlskibene sejlede.
[^12]: **Giftemoden** er, når man er gammel nok til at blive gift.
[^13]: **Søfartserhverv** er de jobs, som har forbindelse med skibe.
[^14]: **"Lagt i kakkelovnen"** er et udtryk for optræk til ballade.
[^15]: **Hyre** betyder løn.
[^16]: **Hyrebasser** er folk, der ledte efter mænd og drenge, som kunne få job
       på skibene
[^17]: **Forhyrede** er ansatte, som gør tjeneste på et handelsskib mod
  betaling.

\chapter{Høj og lav}

 _”Der var mange gange, at skipperen eller styrmanden sagde til drengene: ”Nu ka' I bare
 vente jer til vi kommer i søen, så skal vi nok sørge for jer!” Se det var jo en trussel,
 at så skulle de have tæsk, og så var det vel årsagen mange gange til, at de simpelthen
 bare stak af for at blive fri for skibet.”_

%![Skipper på et sejlskib. ](images/sejlskibe_tema-3-skipper.jpg)

Livet på sejlskibene var underlagt en lang række love og regler, skrevne såvel som
uskrevne. Mange af disse regler var nok nødvendige, mens nogle i vores øjne kan virke
barske og urimelige. Et sejlskib langt fra havn var som et afsondret lille samfund, og i
tilfælde af krise måtte der ikke herske tvivl om, hvem der gjorde hvad. Derfor havde
**skipperen**[^18] eller kaptajnen altid det endelige ord, og man skulle altid adlyde ham. Det var
også skipperen, der var øverste myndighed, altså en slags dommer om bord. Det var ham, som
kunne straffe folk ved at trække dem i løn eller sætte dem i land i den nærmeste havn. Det
gav skipperen et stort ansvar, men også mulighed for at tyrannisere sine folk, når skibet
var på havet, hvis det var det, han ville. Omvendt kunne han også belønne mandskabet med
ekstra fritid i havn, ekstra kost eller måske en lønforhøjelse.

Når man stod til søs med et sejlskib, måtte man blot håbe, at skipperen var en rimelig
mand, og at han og besætningen behandlede én godt. Men det var ofte ikke tilfældet. Selv
om det var ulovligt, blev fysisk afstraffelse som tæv eller spark ofte brugt over for
skibsdrengene, der jo var nye i faget og lavede fejl en gang imellem. Man kunne ikke
stille meget op, hvis man kom om bord på et ”uheldigt skib”. Man var ifølge loven
forpligtet til at sejle med i minimum to år, og der var ikke mulighed for at sige jobbet
op. Stak man af fra skibet, når det kom i havn, blev man kaldt for ”rømningsmand”. Og hvis
man var det, måtte myndighederne i land arrestere en, og så måtte man betale en ret stor
bøde eller i værste fald gå i fængsel.

 _”Jeg mønstrede om bord i en galease, der lå i Frihavnen i København og skulle være
bedstemand. Det var et lille skib – indenrigs. Da jeg havde været der i to dage, så
pakkede jeg sgu min sæk og så gik jeg i land. Jeg havde fået 15 kr. i forskud til en ny
halmmadras. Så lagde jeg resten af pengene på madrassen og til ham den anden dreng, der
var om bord, sagde jeg: ”Nu ka' du give ham dem der og sige jeg er gået!” Så var jeg jo
rømningsmand. – Men så kom der jo bud fra **Sø- og Handelsretten**[^19], at jeg var rømningsmand og
jeg skulle straffes. Og så fik jeg en bøde på 12 kroner, og ham det frække bæst der i
**Holmensgade**[^20] 1. udskrivningskreds han mente jo, jeg var et forfærdeligt menneske, at jeg
sådan kunne rømme fra skibet. Han skrev ned i min søfartsbog, at jeg rømte. Så tænkte jeg:
”Tak skal du ha', nu får du aldrig nogen hyre mere!” Nej, jeg var en storforbryder!”._

%![Her ses besætningen i fuld gang på et sejlskib. Styrmanden kigger på til venstre i billedet. ](images/sejlskibe_tema-3-arbejdsfor.jpg)

Om bord på skibet var folk ordet efter rang. Officererne var selvfølgelig øverst,
men mellem resten af mandskabet var der nogle helt klare, uskrevne regler for, hvordan man
skulle omgås hinanden. Der var ingen plads til sarte følelser eller fine fornemmelser.
Omgangstonen var rå og kontant, men sammenholdet kunne være stærkt.

De ældste matroser havde mest at skulle have sagt. Den ældste fik den bedste køje og det
behageligste arbejde, men måtte til gengæld gå forrest, når det virkelig gjaldt. Det kunne
være vanskeligt arbejde i **riggen**[^21] eller måske som søfolkenes talsmand over for kaptajnen,
hvis der var noget, de var utilfredse med. Systemet fortsatte hele vejen ned til
skibsdrengen eller kokkedrengen, der ikke havde ret meget at sige. Til gengæld var det
almindeligt, at man skulle hjælpe hinanden med arbejdet og hverdagens problemer. Den mere
erfarne skulle være parat til at hjælpe den mindre erfarne med arbejdet.

%![En skibsdreng om bord på et sejlskib.](images/sejlskibe_tema-3-skibsdreng.jpg)

\sFram{showBlue}{
	\subsection{Skibsdreng}

Nederst i hele skibssamfundet stod drengene. Det var de nykonfirmerede knægte, der udmønstrede som skibsdrenge. Skibsdrengens opgave var at gå til hånde ved alt arbejde, at gøre rent forude og agterude, vaske og spule. I logaret eller lukafet, der var et lille opholdsrum ude foran under dækket, skulle han hjælpe til med at hente maden i kabyssen, tage af bordet og vaske op. Det var ikke kun styrmænd og skonnertskippere, som uddelte øretæver. Det gjorde matroser også. Jo, de mest udsatte var de yngste om bord – herunder skibsdrengen.

\emph{”Sommetider da fik jeg jo en øretæve. Det var en matros, der sagde: ”Når vi spiser, og du kan se, vi mangler noget, så skal det ikke siges til dig, at du skal gå hen og hente det, hvis der er mere!” Men jeg sad jo alligevel og lurede – Gud ved om de kan spise mere. Men så rejste han sig bare op, og så smak han mig en. Så kunne jeg nok huske det.”}
}

 _”Ja, det er sådan, altså når vi er i søen, at hvis der er et job vi mener vi kan klare
bedre end en anden, at vi ta'r det. Det er somme tider, det har jeg da set, at de
[skibsdrengene] har stået og rystet for at skulle gå til vejrs, når sejlet rigtig har
slået til. Så har jeg sagt somme tider: ”Bliv du bare nede, det skal jeg nok klare selv.”
Det har jeg gjort flere gange. Og jeg ved en jeg sejlede med, senere han blev skipper, da
jeg kom til at sejle med ham, så sagde han: ”Jeg kan sgu huske da jeg var en dreng, du var
flink ved mig og lod mig blive nede.” Men jeg vidste jo, hvor farligt det var for
nybegyndere at komme op i sådan et sejl.”_

%![En kokkedreng om bord på et sejlskib.](images/sejlskibe_tema-3-kokkedreng.jpg)

\sFram{showBlue}{
	\subsection{Kokkedreng}

I skibe, der sejlede med op til 6 mands besætning, havde man hyret en kokkedreng til arbejdet i kabyssen. Han blev som regel kaldt ”kok”. Til kokkedreng valgte skipperne gerne den mindste og svageste dreng. En sømand, som var lille af vækst, kunne have vanskeligt ved at udmønstre som letmatros eller matros, men som kok gik det bedre an.

Stillingen som kokkedreng var dog ikke attraktiv for de unge, blandt andet fordi kokken skulle være flere steder på en gang. Dels skulle han sørge for kabyssen, dels skulle han deltage i alt andet forefaldende arbejde. Heldigvis havde mange mødre givet sønnen et hurtigt kursus i madlavning, inden han skulle til søs som kokkedreng. Det var desuden almindeligt, at skipperen eller styrmanden gav drengen gode råd og vejledning i arbejdet, fordi besætningen ellers kunne risikere at få uspiselig mad.
}

Folkene holdt øje med hinanden i dårligt vejr og var parat til at give en hånd, når det
kneb. Det var en helt naturlig sag, og der var ingen, der sagde tak for hjælpen – man
forventede, at alle ville gøre det samme. En selvfølge var det, at man deltes om
**provianten**[^22]. Der var ingen, der kunne tage noget til side til sig selv uden at være en
dårlig kammerat. Derfor købte man heller aldrig ekstra proviant med. Man hjalp også
hinanden i fritiden, og at låne lidt penge til en kammerat var der ikke noget i vejen for.

 ”Hvis der var noget man kunne hjælpe med gjorde man det. Jeg har da f.eks. vasket tøj for
en og syet og stoppet for en anden – en svensker, og jeg blev selv engang hjulpet med
nogle rubler oppe i **Skt. Petersborg**[^23], dengang kunne jeg ikke få noget hos skipperen. Der
var en der gav mig lidt. Han tænkte det var synd, at jeg ikke skulle have en ting med hjem
til mine forældre. Man kunne købe de her æg med mange små inden i og tobaksdåser. Det var
russisk lakarbejde.”

\sFram{showBlue}{
	\subsection{Proviant}
Proviant var et andet ord for den mad og drikke, som skibet skulle have med sig, når det stod ud på rejse. De skibe, som udrustede sig til årets togt eller måske flere års togt, havde om foråret travlt med at ordne provianteringen. Så vidt muligt provianterede man i sin hjemby. Her kendte man priserne, og folk vidste, at det ikke kunne betale sig at snyde en lokal kunde. Tit var det sådan, at et par eller flere af handelsfolkene havde andel i skibet, og så gik man naturligvis først til dem og handlede.

Købmanden fik sin bestilling, og det samme gjorde slagteren og bageren. Alt det indkøbte blev omhyggeligt noteret i regnskabet. Man ville ikke bruge for mange penge på kosten. Efterhånden kom det så om bord, æg og margarine, rugbrød, kød og flæsk, fisk, sild, grøntsager, mælk, gryn, kartofler osv. Man havde ikke køleskab, så kødet blev saltet i tønder. Var man længe til søs, blev nogle af madvarerne ofte fordærvede. Man provianterede derfor også undervejs på rejsen, hvis man havde mulighed for det.
}

%![Provianten om bord på skibet kunne variere meget.](images/sejlskibe_tema-3-proviant.jpg)

Gensidig tillid var forudsætningen for, at man kunne hjælpe og stole på hinanden som
skibskammerater. Hvis nogen brød denne tillid, reagerede de andre. Tyveri var groft brud
på kammeratskabet – ja, selv mistanke om tyveri kunne ophidse gemytterne. At låse sin
**skibskiste**[^chest] var det samme som at beskylde kammeraterne for at være tyvagtige. Den slags
tolererede man ikke.

 ”Man måtte ikke låse sin skibskiste. Det var det samme som at vi mistænkte kammeraterne
for at stjæle. Jeg havde en lås i kisten, og jeg kunne ikke tage nøglen ud uden at jeg
skulle låse den, og den måtte ikke sidde i for de stødte benet imod. ”Hvad skal
jeg gøre?” spurgte jeg. ”Ja, du kan brække låsen af!” sagde de. Det måtte jeg så
gøre”.

Bonde var et skældsord i et sejlskib. Et bondeskib var et skib, hvor tingene blev
grebet forkert an, og hvor kammeratskabet ikke fungerede. I de fleste skibe
fungerede kammeratskabet godt. Det betød ikke, at folk ikke kunne blive uvenner,
men konflikterne blev løst efter reglerne. En af reglerne kunne være slagsmål,
helst overvåget af andre af besætningen. Når kampen var slut, skulle de to
kamphaner være gode venner igen. Der var ikke plads til langvarigt had på et
sejlskib, fordi man var meget afhængige af hinanden.

[^18]: **Skipperen** er lederen af mandskabet på et skib.
[^19]: **Sø- og Handelsretten** er en domstol for sager, der har med skibe
  at gøre.
[^20]: **Holmensgade** er en gade i København.
[^21]: **Riggen** er alt det, der sidder på masterne.
[^22]: **Proviant** er maden om bord på skibet.
[^23]: **Skt. Petersborg** er en by i Rusland
[^chest]: **Skibskister** er en kiste til opbevaring af private ejendele som
  tøj mm.

\chapter{Lev vel?}

 _”Efterårsdage blev vi aldrig vasket i søen. Heller ikke om vinteren. Det var alt for
 koldt at tage noget tøj af og stå der og vaske sig. Det kunne man ikke. Sommerdage der
 på Østersøen kunne vi jo tage udenbordsvand.”_

![Frisk fisk var et velkomment supplement til kosten. Her ses en
friskfanget haj et sted i Atlanterhavet på skonnerten Kodan. ](images/sejlskibe_tema-4-haj.jpg)

Hverdagen på et sejlskib kunne være hård. Arbejdet var krævende, og de hygiejniske forhold
ikke for gode. Man boede trangt, kunne ofte ikke komme i bad, og kosten var sparsom.
Madlavningen foregik i et lille aflukke **(kabyssen)**[^24] placeret midt på dækket, hvor der lige
var plads til at vende sig. Maden blev lavet på et lille komfur, hvor røgen let kunne slå
ind og fylde rummet. Det var heller ikke altid nemt at lave mad, når skibet gyngede
kraftigt i en storm eller en forkert sø. Madlavningen på de mindre sejlskibe blev som
regel altid foretaget af yngste mand ombord. På de mindste skibe kunne det være forbundet
med hårdt arbejde. Foruden madlavningen skulle kokken også hjælpe til med alt det andet om
bord. Om vinteren kunne det dog have sine fordele at være i kabyssen.

 _”Kabyssen var det eneste sted der var varme. Der var ikke varme nogen steder om vinteren
undtagen kabyssen. Det var det eneste sted vi kunne få tørret et par strømper.”._

Kosten var meget forskellig fra skib til skib. Meget afhang af, om kokken kunne lave god
mad ud af den proviant, man havde med i skibene. På nogle skibe spiste man godt, mens der
på andre blev sparet på kosten. Når først skibet var afsejlet, kunne man ikke gøre så
meget.

”Der var meget forskel på det. Ja – uha. Nogle var hele **sultekasser**[^25] og andre var
nogenlunde. De sparede på kosten for at give rederiet mere. Det var jo om at få overskud
på de skibe. Jo mere de sparede på kosten jo mere tjente de jo for pokker. Vi kunne godt
klage, men når vi ikke havde ud-provianteret mere og lå i søen, så var man nødt til at
indordne sig.”._

Frisk kød var en sjældenhed. De store skibe, der krydsede oceanerne, havde dog ofte
levende grise om bord. Der var også mulighed for at supplere kosten med fiskefangst. Fik
man harpuneret en haj eller en delfin, kunne man være heldig at få helt frisk kød. Dårlig
eller for lidt kost kunne føre til mangelsygdomme og svække immunforsvaret. Og det kunne
være farligt, når man kom til byer, hvor der var udbrudt epidemi. **Kolera**[^26], **dysenteri**[^27], **tyfus**[^28]
og lignende skavanker var kendte fænomener blandt søfolkene. Alle frygtede at blive syge.
Tænkte man sig ikke om, kunne det betyde, at man måske døde.

_”I Petersborg stod vi og **handede**[^29] sten. Så havde vi godt nok fået at vide, at der var
kolera. Vi måtte endelig ikke gå i vandet. Vi havde jo jollen liggende i vandet agterud,
og når så vi havde stået og handet næsten op fra lasten og op i land fra 6 morgen til 6
aften og svedt, og det var egentlig hårdt arbejde, så kunne vi jo ikke nære os om aftenen,
andet end at vi skulle lige ned og skylle os en bitte, og der var ingen, der så det, når
vi gik ned i jollen og så blev vasket en bitte. Men det resulterede alligevel i, at vi
havde en tysker om bord fra Königsberg, og han fik kolera og blev lagt i land og han døde
dèr. Vi skulle jo have holdt os væk, men vi andre, vi gik da fri.”._

En anden plage var **søsygen**[^30], som naturligvis var en del af sømandslivet. De fleste søfolk
havde på et eller andet tidspunkt oplevet søsyge. For nogle var det en konstant plage,
mens søsygen for andre var et engangsfænomen. En slem søsyge kunne tage modet fra de
fleste.

_”Om sommeren lastede vi kridt til Kotka, og det blev dårligt vejr. Vi lå underdrejede oppe
i bugten ved Øland. Jeg mener vi lå der i halvandet døgn, og drengen var så søsyg, så
ganske forfærdelig søsyg, og han lå og rullede rundt i vandet. Det var frygteligt at se.
Vi fik så meget vand over, at vi næsten ikke kunne komme ned i forlogaret. Så sagde jeg
til skipperen: Nu lægger jeg ham ned i min **køje**[^31]. Ja, han ville ikke have ham derned og
brække… Jamen han har ikke noget at brække af, det kan du godt tro. Han kom så med der. Så
lå han dernede en dags tid. Han var aldrig søsyg mere.”._

Bedre blev det heller ikke af, at toiletforholdene var meget primitive.

_”Ja, se vi havde en lille tønde, f.eks. styrmand og skipper de havde jo henne i hytten
agter, der havde de jo en spand, og den skulle kokken jo holde, sørge for at tømme hver
anden dag. Men mandskabet forude, de havde jo bare en rund tønde med to stropper i, og når
vi så lå i havn, så skulle de sørge for at krybe i læ et eller andet sted, særlig hvis der
gik damer på kajen. Der var i hvert fald tønde, og så fyldte vi en halv spand vand i og så
ud over siden med det. Men altså, vi havde ikke andet end den åbne tønde.”._

Hvis sejlskibet var af ældre dato, kunne det være plaget af småkryb. Den erfarne sømand
vidste, at det var bedst at holde sig væk fra skibe med skadedyr, hvis man kunne.

_”Det var meget almindeligt dengang med sådan utøj om bord. I **barkentinen**[^32] \textsc{Haabet} her af
byen var der mange væggelus. Jeg var mønstret i den og skulle med den ud at sejle. Så var
de andre kommet om bord, men jeg sov jo hjemme. Først lå de i køjerne, så lå de på deres
**kistebænke**[^33], og så gik de i land. Så sagde jeg til kaptajn Andersen: Ja, nok er jeg
mønstret her, men jeg er nu ikke **mønstret**[^34] til at være sammen med sådan en besætning som
den der, så jeg vil i land. Åhr, det kunne jeg altså ikke komme. Ja, det vil jeg altså,
for jeg vil ikke sejle sammen med alle de væggelus. Noterne mellem plankerne de var helt
røde af bare lus. Men der var masser af skibe, der havde væggelus dengang, og man
kunne gerne se det ved at lyset brændte på de skibe om natten for at holde lusene
lidt i ave. Så kom jeg ikke med den.”._

_”Kakerlakker det var jo nogle bæster. Når vi skulle sove så rendte de jo og kneb
os, men det var ikke farligt på nogen måde. Men se der i sejlskibene, selv om der
var kakerlakker og væggelus, så blev det jo ikke røget ud som f.eks. i damperne.
De skulle have bevis. Det måtte ikke blive over 6 mdr. gammelt, og hvis der var
utøj om bord, så skulle skibet svovles hver 6. måned. Så blev vi jaget i land
nogle timer.”._

[^24]: **Kabyseen** er køkkenet om bord på et skib.
[^25]: **Sultekasser** er skibe, hvor man fik for lidt at spise.
[^26]: **Kolera** er en smitsom mave-tarm-sygdom, som ofte dukker op i
  forbindelse med urent drikkevand.
[^27]: **Dysenteri** er en smitsom tarmbetændelse med blodig diarré. Opst som
  følge af dårlig hygiejne i troperne.
[^28]: **Tyfus** er en art af blodforgiftning og den farligste af alle
  salmonelleinfektionernen. Opstår som følge af forurenet mad og drikke i
  troperne.
[^29]: **Handede** betyder 'rakte', eller 'videregav'.
[^30]: **Søsyge** opstår som følge af ubalance i balancesansen. Kendetegn er
  kvalme, depressiv tilstand, opkast og mavesmerter.
[^31]: **Køjer** er senge om bord på et skib.
[^32]: **Barkentine** er en skibstype med mindst tre master.
[^33]: **Kistebænke** er kister, man kunne sidde på.
[^34]: **Mønstre** er at gå om bord og få hyre på et skib; det er altså
  begyndelsen på jobbet.

\chapter{Sjov og alvor}

 _”På frivagterne om bord i søen da sov vi jo den meste tid. Somme tider spillede vi kort.
 Noget skulle tiden jo gå med.”_

Fritid til søs var der ikke meget af i sejlskibene. I gennemsnit havde man 12 timers
arbejde og 12 timers fri i døgnet. Med et almindeligt søvnbehov på 7-8 timer var der kun
4-5 timer i døgnet til at spise, klæde sig på, drikke te eller kaffe, personlig hygiejne
samt toiletbesøg. For det meste blev fritiden sovet væk – der var ikke tid og kræfter til
andet.

 _”**Frivagterne**[^35]? Puh-ha. Der røg vi lige på hovedet i køjen og så sov vi lige som en sten.”_

Søfolkene var på mange områder afskåret fra omverdenen. Der eksisterede ingen form for
radio, ingen nyhedsinformation, ingen underholdning – ud over hvad de selv kunne finde på.
Når skibet var i søen, betød det mindre. For det meste nåede man kun lige at stikke
hovederne sammen ganske kort, inden der skulle soves. Og tit var der kun tid til at læse
eller genlæse breve hjemmefra. Egentlig fritid om bord kunne man stort set kun opleve på
de store **fuldriggede sejlskibe**[^36], der med passat-vindene i ryggen krydsede de store oceaner.

 _”Som regel der i de danske skonnerter når de havde frivagt, og de havde været oppe og
havde puklet i den tid, så var de sgu gerne så udmattede, så de sov altså fra det hele. Så
var det en anden ting i de store sejlskibe i **passaten**[^37], hvor det var så fint vejr, der var
mange der brugte deres frivagt til at lave skibsmodeller – helmodeller eller halvmodeller
og andre sådan forskellige ting. Ligesådan flaskeskibe blev der lavet. Jeg kan huske flere
gange jeg var på Charlie Brown, en meget kendt beværtning i London, hvor alle verdens
søfolk kom når de havde mønstret af, og der var alting deroppe – der var krokodiller og
assagaier og alt mellem himmel og jord og skibsmodeller, og der var det så tit, når
søfolkene ikke havde mere, så tog de sgu deres skibsmodel under armen og gik op og sagde,
hvad kan jeg få for den? Jah, du kan få **ti shilling**[^38], og så kan du indløse den. Og så fik
de ti shilling og drak dem op, og den blev aldrig indløst, så de fik jo nogle souvenirs –
folks arbejde der.”_

 _”I skonnerterne var der ikke noget fritid. Men i de store skibe. Jeg kan huske engang vi
lavede sådan en kludebold. Dagen før havde der været åbent i **slopkisten**[^39], og så var der en
af mine kammerater, en matros der havde købt et par sko, og dem havde han på dagen efter,
da vi spillede fodbold. Vi løb rundt på stordækket og det havde jo ikke meget med fodbold
at gøre. Det var nærmest sådan tjatteri. Og så sparkede han kraftigt, så denne her sko røg
af og op i rigningen og **udenbords**[^40] regnede han med. Der røg din sko udenbords, var der
nogen der råbte. Ad helvede til, råbte han, så blev han så gal, så tog han den anden af og
smed ud. Men så var der en der sagde – jamen her sidder jo din sko – den sad fast i
rigningen. Så blev han endnu mere tosset.”_

![Der spises et måltid mad til lidt musik mens skibet er i
havn.](images/sejlskibe_tema-5-spisning.jpg)

Det hændte, at der blev spillet musik og sunget, men som regel ventede man til skibet kom i havn. Det var i
havnene, sømændene kunne slå sig løs, mens skibet ventede fragter eller skulle repareres.
Når skibet nærmede sig en havn efter mange dage til søs, begyndte landgangsfeberen.
Snakken tog til om alle de ting, man skulle nå, inden sejladsen blev genoptaget. Nogle
brugte straks opholdet i havnen til at få et velkomment kosttilskud.

 _”Når vi kom til en dansk havn, så røg vi op og købte en hel lagkage. Den kostede 2 kr. –
en flødeskumslagkage. Så åd vi den. I England købte vi plumbudding, og i Spanien købte vi
engang nogle fine kager, og der var noget i, og det var sgu klipfisk, så vi åd dem altså
ikke. Vi havde lige losset klipfisk og så få en kage med klipfisk!”_

Sømændene var afskåret fra et almindeligt seksualliv. I uger og måneder lå de ude til søs
uden at have kontakt med kvinder. De unge var afskåret fra at søge den naturlige kontakt
til jævnaldrene unge piger, og de voksne mænd måtte undvære deres kæresters kærlighed. I
det mandssamfund, som skibene udgjorde, blev kvinder for nogle en besættelse, og besøg i
havn kunne ikke råde bod på den manglende kontakt med kvinder. Disse havnebesøg blev for
det meste et spørgsmål om hurtigt at tilfredsstille nogle seksuelle drifter – det vil
sige, at man købte sig til et samleje med en totalt fremmed og uvedkommende pige, for hvem
det kun gjaldt om at få det overstået. Allerede mens skibet var på vej ind, havde man om
bord besluttet sig for, om man skulle på bordel eller horekasse, som man også kaldte det.
En ung sømand, som tøvede med at slutte sig til selskabet, kunne blive til grin blandt de
andre.

 _”Jah, når søfolkene de gik i land – hvis der var bordeller, så gik de ind på dem og ellers
på havnerestaurationer, hvor de kunne træffe piger. Det var gerne de ældre som tog
føringen. Vil I med derhen? Skal vi det i aften? Og sådan. Nogle ville jo med og andre
ville ikke. Der var ingen tvang, men sådan en ung mand som ikke ville med på bordel kunne
de godt lide at drille lidt – åh – sådan en barnerøv osv. Så hændte det jo, at han til
sidst tænkte $\ldots$ det skal de skisme ikke tro $\ldots$ og gik med.”_

Frygten for kønssygdomme afholdt dog nogle fra at gå med.

 _”Der var en fisker med oppe fra **Hanstholm**[^41]. Vi kom til **Malaga**[^42] og skulle **losse**[^43] fisken, og så
var han gået op til nogle piger, og så gik han og ragede sig en gonoré til, og jeg så
hvordan han skabte sig og jamrede sig, og han blev lagt i land på et hospital og kom slet
ikke med skibet, vi måtte sejle fra ham. Da tænkte jeg ved mig selv: For det første kom
jeg fra et godt barndomshjem, og for det andet, tænkte jeg, du skal aldrig komme til sådan
en pige, for der er alligevel risiko, når jeg havde set ham, hvordan han led og så blev
sejlet agterud dernede i Malaga, så tænkte jeg ved mig selv: nej, det skal du holde dig
fra!!” _

Nogle prioriterede i stedet det at komme ud at se noget nyt. Det kunne være tyrefægtning i
Spanien, en tur i teatret i London eller måske noget helt tredje.

_”Jeg kan huske, da vi lå i Norge, da var vi ude at se på Holmenkollen – nej, det med
værtshuse, det har jeg ikke haft lyst til sådan.”_

[^35]: **Frivagter** er fritid om bord.
[^36]: **Fuldriggede sejlskibe** er skibe med råsejl (firkantede sejl) på alle
  master.
[^37]: **Passatvinde** er jævne vinde, der blæser hen over verdenshavene.
[^38]: **Shilling** er en engelsk møntenhed. Svarede til 4 kroner, og det er ca.
  90 kroner i dag.
[^39]: **Slopkisten** er kaptajnens lille private butik om bord.
[^40]: **Udenbords** vil sige vandet, der omgiver skibet.
[^41]: **Hanstholm** er en fiskerby i Nordvestjylland.
[^42]: **Malaga** er en havneby i Sydspanien.
[^43]: **Losset** er, når varerne er kommet ud af skibet.

\chapter{Pas På !}

 _"\textsc{Primo} forsvandt i **Nordatlanten**[^44] lige efter, at jeg var gået fra den. Det var
 en 3-mastet skonnert. Jeg kom i land på **Københavns red.**[^45] Så rejsen efter, da forsvandt
 skibet. Ja, de fandt den senere, men de fandt aldrig besætningen.”_

![Mange sejlskibe blev med vilje sejlet ind på land, hvis mandskabet frygtede forlis. {3}](images/sejlskibe_tema-6-forlis.jpg)

Når man satte sine ben i et sejlskib, udsatte man sig også for en vis risiko. Man blev
ganske vist rig på oplevelser, men kunne også i værste fald miste livet. Farerne lurede.
Sygdom, lemlæstelse eller død var noget, de fleste søfolk stiftede bekendtskab med på et
eller andet tidspunkt i deres karriere. De hørte om skibe, der forsvandt, mødte kolleger,
som døde af feber, eller oplevede måske selv en ulykke, som de mirakuløst overlevede. Det
var ikke ualmindeligt, at gamle matroser havde oplevet indtil flere **forlis**[^46], som de med gru
kunne fortælle om nede i logarets mørke. Grumme historier var en uundgåelig del af
sølivet, og det var ikke uden grund, at søfartssamfund havde en større andel af enker end
andre steder. Et sejlskib var ganske enkelt en af de farligste arbejdspladser, der
fandtes. Det frygtede råb: ”Mand overbord!” betød i dårligt vejr for det meste et farvel
til en skibskammerat, der blev overladt til den ensomme druknedød. Voldsomme oplevelser
kunne føre til syner eller en stærk tro på større magter. Troen på Gud kunne være stærk,
men afhang også af det hjem, som sømanden kom fra. Nogle var mere troende end andre.

![Vrag af sejlskibet Castor, der strandede ved Casablanca i 1913
](images/sejlskibe_tema-6-vrag.jpg)

 _”Jeg stod henne ved
skipperen, han stod ved roret og havde **sejsing**[^47] omkring livet, og så siger han, ”hold dig
ved”, siger han, ”for der kommer en grim **sø**[^48]”, og så kiggede han ud til siden, men vi
skulle kigge opad, da kom den rullende oppe over os, lige som sådan en elektrisk pæl der
er i højden, og så faldt den jo i øvrigt, og jeg tog fat under **ruffet**[^49] sådan og jeg har
sørme taget sådan, at mine negle de var gået ind i træet. Men der var ikke noget at gøre,
jeg blev væltet ud, og jeg havde som ærlig talt troet, at jeg var druknet, det troede jeg.
Alt var jo mørkt, når man kom under sådan en masse vand, men det har jo været mit held
mange gange, at jeg har været udsat for det og er sluppet godt fra det._

 _Ja, jeg blev faktisk smidt lige på hovedet ind på dækket, og så var jeg på vej ud igen,
men så tog skipperen mig så. Og da kunne jeg høre, det er altså sådan noget, man får i
øjeblikket, jeg hørte en koncert af den skønneste musik, som du kan tænke dig, og i min
underbevidsthed, dèr var jeg klar over, at jeg tænkte, jeg vidste ikke rigtig, hvor jeg
var henne, jeg var slet ikke klar over, om jeg i det hele taget levede, men jeg hørte
denne her musik, og så lige pludselig hører jeg nogen, der siger: ”Lever du, er du vågen”,
og så var det skipperen, der stod og ruskede i mig. Og så vågnede jeg op, og så siger han
”Fejler du noget”. ”Nej, jeg gør ikke”, siger jeg så, da var jeg ved at være rask. Så
siger han, ”Så gå hen og hjælp til med at hale det sejl ned”._

![Strandet skib](images/sejlskibe_tema-6-strandet-s.jpg)

 _Og det var det sejl, at søen havde splintret alle trådene der. Der var jo
tusinde tråde og det var det, at stormen fløjtede i, og det var det, der lavede musikken
jo. Og så står jeg derhenne sammen med mine kammerater, og jeg så med det samme, vi
mangler en mand, men jeg kunne ikke sige, hvad han hed. Det kunne jeg ikke._

 _Men, - så gik jeg hen og sagde til skipperen ”Vi har mistet en mand”, og med det samme han
sagde ”Det er Hans”, da var jeg klar over det. Nej, der var ikke noget at gøre ved det.
Ja, så sagde skipperen, jeg skulle kravle et stykke op i rigningen og kigge, og skønt der
lå planker og brædder her og der, kunne jeg ikke undgå at se han lå og fægtede med armene,
men – ak gud fader, der var ikke noget at gøre $\ldots$ (bevæget). Nej, ok nej, - vi var selv så
hjælpeløse, selv om vi havde villet, ku' vi ikke ha' drejet den, eller fået den til at
køre rundt. Det ku' vi ikke. Så havde vi skullet bruge flere kilometer. Det kunne vi ikke.
Det sagde han jo også til skipperen, han bad jo en bøn for ham, men..”._

Blandt besætningerne var sygdom altid frygtet. Noget af det værst tænkelige for en sømand
var, hvis han blev sat i land på grund af sin sygdom på et hospital og derefter
agterudsejlet – det vil sige, at skibet sejlede sin vej med den raske besætning. Skete det
i troperne, var man nærmest prisgivet. Risiko for følgesygdomme kombineret med mangelfuld
pleje gjorde, at særlig mange døde på de kanter. Nogle gange var sømanden også afhængig af
kontante midler. Ingen penge – ingen behandling. Om bord var der ikke meget, man kunne
stille op. Skipperen og styrmanden havde ganske vist lidt lægelig viden med fra
navigationsskolen, men når det kom til stykket, kunne de ikke stille meget op.
Småskavanker kunne dog klares med lidt opfindsomhed.

 _”Joh, jeg havde tandpine. Hold kæft! Jeg kunne springe overbord. Ja, jeg havde det hele
vejen over Atlanterhavet, jo. Jeg havde ikke mere forstand på tandpine end som der var
honning i en skruptudse. Hvergang jeg kom op på dækket så var der aldrig noget i vejen.
Når man skulle ind og sove og mente, nu har jeg det dejligt, så lige så snart jeg lagde
hovedet på hovedpuden, så havde jeg tandpine med det samme. Jeg havde det sådan, at jeg
var ligeglad med hvad jeg gjorde! Til sidst så glødede jeg en **sejlnål**[^50]! Kokken han sagde:
”Må jeg stikke den ned?” Gu’ må du ej, den vil jeg sgu selv hav lov at stikke ned.” Jeg
fik lige fat i den – han skulle kigge mig i munden om det var rigtigt – så jagede jeg til.
Men jeg tror nok, jeg sprang helt ”op over fokkeråen”! Hold da kæft hvor gjorde det ondt!
Så var den tandpine væk. Jeg havde brændt nerven over._

 _Men da havde jeg så også gået med det i over en måned. Og jeg stoppede papir i ørerne, og
jeg var lige ved at stoppe **Social Demokraten**[^51] op i r\o ven. Ja, hvad jeg ikke gjorde altså, Alt
muligt prøvede jeg! Jeg stoppede sølvpapir ind i ørerne og stoppede det alt for langt ind,
så jeg næsten ikke kunne få det ud igen. Hver nat kl. 12, så kom det. Så havde skipper
fået fat i en **passer**[^52] og havde lavet sådan et par kroge og så snuppede han tanden til
sidst. Men han var mindst 8 dage om at få fat i den, den blev siddende.”._

[^44]: **Nordatlanten** er havet mellem Norga, England, USA, Canada og
  Grønland.
[^45]: **Københavns red** er det sted uden for havnen, hvor skibene lå for
  anker.
[^46]: **Forlis** er, når et skib snak eller på anden måde gik i stykker.
[^47]: **Sejsing** er tovværk, man bandt om sejlene.
[^48]: **Sø** er en eller flere bølger.
[^49]: **Ruffet** er dækhuset, der benyttes til kahytsrum.
[^50]: **Sejlnål** er en nål til at sy sejlene med.
[^51]: **Social Demokraten** er en avis.
[^52]: **Passer** er et instrument til at måle afstande på et søkort.

\chapter{Den store verden}

 _"Jeg havde mange sjældne **konkylier**[^53] nede fra **Vestindien**[^54]. Og så havde jeg et af de her
 jernspyd der var med modhage og pyntet sådan som et tyrefægterne bruger. Jeg havde da
 også en harpun. Men det hele forsvandt. Mine brødre havde gået og solgt det for at få
 penge.”._

![Sømanden havde mulighed for at se verden og møde fremmede
folkeslag. Rejsen til fremmed verdensdele begyndte ofte på en større
europ\ae isk havn som her i Antwerpen i
Belgien.](images/sejlskibe_tema-7-antwerpen.jpg)

De søfolk der fik mulighed for at komme med de større skibe, som sejlede til andre
verdensdele, mødte folk fra helt andre kulturer end den nordeuropæiske, som de selv var en
del af. Dengang havde man et noget andet menneskesyn end i dag, og fordomme om andre
kulturer var almindeligt. Dog kunne kortvarige bekendtskaber opstå spontant alligevel, og
hvis der var tale om længerevarende havneophold kunne der knyttes kontakt til de lokale.
Lettest var det, hvis der var tale om andre nordeuropæiske eller nordamerikanske havne.
Kom et sejlskib til en lille havneby, kunne besætningen være så heldig at blive inviteret
med til fest.

 _”Vi kom op til New Foundland i december måned. Det var helt sjovt at komme op og losse kul
på sådan en lille **plads**[^55]. Der boede 300 mennesker der. Første dag, vi var kommet ind, så
var der fest. De havde sådan et **forsamlingshus**[^56] ($\ldots$) og når kulskibet kom den ene gang om
året, så var der fest, og det vil sige, at vi jo skulle op og spise til aften, og der kom
så de lokale folk og så spillede de jo. **Vi stak helt af fra selskabet**[^57], for vi havde jo
vores sko på og lignede slet ikke de andre. Men det gik godt og det var sjovt.”_

![Det var spændende at møde fremmede
folkeslag](images/sejlskibe_tema-7-fremmede.jpg)

Sværere kunne det være for de danske søfolk at
begå sig, når de sejlede til fjerne egne. De folk, som sømændene mødte, var ofte blandt de
fattigste. Det kunne være havnearbejdere, prostituerede eller måske tiggere, der var
henvist til et kummerligt liv i rendestenen. Sprog- og kulturbarrierer gjorde
kommunikationen overordentlig vanskelig. Alligevel hændte det, at man fik fælles
oplevelser med den lokale befolkning på godt og ondt.

![Fremmede himmelstrøg, som her en by i Venezuela, satte fantasien i sving, for hvad mon der
ventede i land?](images/sejlskibe_tema-7-venezuela.jpg)

 _”This, Charles, Hjulemand og jeg omme i floden, som løber ud et lille
stykke herfra, og vaske tøj og hente vand. Fint vand at vaske i. Vandet har vi nemt ved at
få fat i. Vi vender bare båden, så den bliver næsten fuld af vand, for vi skal jo også
være der. Jeg ved ikke hvor mange mil oppe i landet floden kommer fra, men hele vejen ud
bader folk i den og vasker tøj og smider affald ud i den. Der er ikke meget vand i den nu.
Når vi ligger på maven, stikker ballerne ovenfor. De sorte piger her har ikke badedragt
på, så de må ligge på maven hele tiden. Vi maver rundt mellem hinanden i al ærbarhed. Det
er meget fint drikkevand. Det kan næsten gå ombord selv, så fyldt af alskens ting er det.
Selv hajerne kan lide det. De kommer helt ind på **det grunde vand**[^59], småhajerne altså, for at
få en god mundfuld af det dejlige vand. Jeg kan gætte mig til, hvor sprælsk det vil blive,
når vi har sejlet rundt i varmen med det i nogle uger. Når vi så kommer ombord, hiver vi
det op med pøse og hælder det i vandfadene. Vi har to store stående foran **halvdækket agter**[^58]
samt et stort træfad surret på dækket samt en stor beholder nede i forskibet._

Traf negeren fra Cape Vincent, vor gode ven. Han gav os kokosnødder. Dem er der nok af her
i træerne. Fik dem i frøkapslen, men han viste os, hvordan vi skulle få dem ud af den. Vi
drikker en del kokosmælk, som skulle være sundt. Det er i hvert fald rent. Om aftenen var
alle mand i land til fest og fik limonade, men vi gik ret tidligt om bord igen. Det var en
fin måneskinsaften, så Aksel og jeg startede igen. Nu ligger der en del negerhytter
placeret rundt omkring i sandet på vejen op til byen. I en ret stor, som der ikke var
sider i, var der dækket et stort bord med forskelligt. Der skulle nok være fest, men der
var ingen mennesker, så Aksel og jeg satte os ned og ventede. Der kom ingen, så vi smagte
lidt på varerne, så gik vi, for Sveske blev bange for, at vi skulle få tærsk, hvis de
opdagede os.

Om aftenen til fest på pladsen, ja det er altså en stor plads med træer og buske. I midten
et monument og en åben plads med stole, som vi kan leje, men det opdagede vi for sent. Det
var en aften, alle stolene var optaget, så rejste et par sig vel for at strække benene
lidt, og straks snuppede vi stolene. De kom tilbage og blev ved at kredse om os, men sagde
ingenting, så vi blev siddende. Senere fortalte Vincent os, at vi skulle give penge for
stolene, så det var en flov historie. Hvad mon pigerne tænkte om os. Inde på denne lille
åbne plads går alle de hvide, altså noblessen, som vi altså mener vi hører til, idet vi
også opholder sig der. Uden om går så alle de sorte, eller dem som ikke er hvide nok, og
alleryderst går så skravlet.” .

Indtrykkene fra den store verden tog mange med sig hjem i form af forskellige souvenirs,
der kunne foræres til familien. De hjembragte sager gjorde som regel altid lykke.

”Nede sydpå havde de alle de sydfrugter, appelsiner og vindruer. Det var luksus dengang.
Jeg købte en stor trækasse med figner. Den havde jeg med hjem til mine brødre og min
søster. De kendte jo ikke andet end de tørre. Den gjorde lykke den kasse der. Min søster
var 1 år ældre end mig, og hun kunne altid bruge parfume og sådan noget. Det var spændende
med udenlandske ting, især fra Frankrig. Der var ikke så meget ved ting fra England.”

[^53]: **Konkylie** er en art sneglehus dannet af rovsnegl, der lever i vandet
  og spiser ådsler, mindre fisk og muslinger.
[^54]: **Vestindien** er en øgruppe i det Caribiske Hav mellem Syd+ og
  Nordamerika.
[^55]: **Plads** er en mindre havn
[^56]: **Forsamlingshus** er et hus, hvor mindre lokalsamfund kan mødes til
  foredrag, fester eller lignende.
[^57]: **Vi stak af fra selskabet** betyder herÆ Vi lignede ikke de andre.
[^58]: **Halvdækket agter** er et delvist ovedækket område bagerst i
  skibet.
[^59]: **Det grunde vand** er et udtryk for lavvandet område.
\end{document}
