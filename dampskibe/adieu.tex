\chapter{Adieu}\label{adieu}

\begin{quote}
    ''Jeg måtte sige farvel til \wordd{Ørstedværket}{
    Ørstedværket er et stort kraftværk i København, der producerer
    elektricitet.}, og slutfasen, et nyt og
herligt afsnit i min uddannelse begyndte. Det var fint vejr på min
første rejse, men alligevel ude i Atlanterhavet var der store dønninger,
der fik skibet til at duve og rulle svagt. Kort sagt, jeg blev søsyg og
det stod på hele turen over til New York.''
\end{quote}

For nogle var søen en naturlig arbejdsplads at vælge, mens den for andre
nærmest blev en nødudgang. At flere og flere rejste fra landet til byerne
føre naturligvis til, at befolkningspresset på storbyerne voksede. Det
medførte for mange vanskelige levevilkår, hvis der kom perioder med
arbejdsløshed, eller hvis lønnen var lav. Mange søfolk til dampskibene
blev rekrutteret i byens fattige kvarterer. Man søgte arbejde til søs,
fordi der ikke var andre muligheder, eller fordi man var træt af
omgivelserne og den situation, man befandt sig i. 

\begin{quote}
''I tiden 6. oktober
    til 3. november 1919 gik jeg hver dag på \wordd{DFDSs}{DFDS er Det
    Forenede Dampskibsselskab. Var et af de største rederier i Danmark.}
    forhyringskontor ved Kvæsthusbroen. Det var mellem klokken 10.00 og
    14.00. Der skete sgu aldrig mig noget. Der kom mange drenge, som blev
    kaldt ind på kontoret, kom ud igen, og jeg så dem aldrig mere. Da jeg
    havde siddet der en 14 dages tid, og der stadig ikke var bud efter
    mig, blev jeg sgu gal i hovedet, gik hen og bankede på døren, lukkede
    op og gik ind. Hvem har sendt bud efter dig? lød det fra en
    herrestemme bag et stort skrivebord, og jeg svarede: Det er der ingen
    der har gjort, men nu er jeg kommet her i 14 dage og har set mange
    drenge komme herind, og komme ud igen, hvorefter jeg ikke har set dem
    mere, hvorfor jeg mente, de havde fået \word{hyre}, så jeg mente, det måtte
    være min tur nu! Du er for lille. Du kan ikke nå \wordd{kopperne i
    dækket}{Kopperne i dækket betyder ,at kopperne var gemt i dybe skabe,
    så de ikke smadrede, hvis skibet gyngede for meget.},
    sagde han. Kopper i dækket, tænkte jeg. Så kan jeg vel tage en stol
    at stå på! svarede jeg. Ej, ej, sagde han, en lille vågen er bedre
    end en stor og doven, og så fik jeg hyre som messedreng på ''NIELS
    EBBESEN'', København-Randers og retur, og jeg skulle møde mandag
    morgen kl. 06.00 på ''NIELS EBBESEN'' medbringende tøj, madras og
    sengetøj, vær så god, og det var mandag den 3. november 1919.

    Nu var det med at komme op til den af \word{DFDS} befalede læge og blive
    totalt undersøgt for både legemlige og åndelige sygdomme. Han syntes
    nok, at jeg var lidt spinkel, men ellers sund nok, så jeg passerede hans
    stærke brilleglas. Jeg fik bedstefaders værktøjskasse, en stor firkantet
    kasse med låg og beslag. Nu var Otto kommet hjem, så han stod op og
    hjalp mig til båden med kassen med mere. Da jeg for sidste gang gik ud
    af døren i nr. 32, sagde fruen til min far: Ja, nu kan du jo se, om han
    kan være hos andre mennesker! Egentlig ville jeg have været i
    snedkerlære, men da snedkermestrene i København ikke havde deres
    lærlinge boende, og jeg hverken kunne eller ville bo hjemme, ja så var
    den hurtigste udvej at sende mig til søs, og -- ja -- det var egentlig
    godt. Jeg trængte til både mad og øretæver og fik begge dele, men
    absolut ikke flere end de andre drenge ombord.''
\end{quote}

I damperne var der brug for en helt ny type søfolk. Det var folk, der
kunne håndtere og vedligeholde de teknisk komplicerede dampkedler.
Matroserne fik derfor selskab af fyrbødere og maskinister, der havde
deres egen fagforening og arbejdede under andre vilkår med meget faste
regler for arbejdstider med mere. Men det gik de fleste steder
nogenlunde. Lærepladser for smede var de fleste steder et lige så barsk
sted at blive udlært som på et skib.  Det var derfor ikke svært for
smedene at falde ind i jargonen, selv om tilværelsen blev mere opdelt om
bord. Fyrbøderne snakkede ikke så meget med ''dem forude'', matroserne,
ganske enkelt fordi de to grupper ikke opholdt sig de samme steder på
skibet i dagligdagen. Det aftvang en naturlig respekt, hvis man var god
til sit håndværk. Ikke alle smede var dog lige søstærke, men alt kunne jo
læres\ldots{}

\begin{quote}
''Der var jo ikke arbejdsløshedskasser dengang. Enhver måtte klare sig
    selv. Der var jo \wordd{fattigforsorgen}{70}, men den undgik enhver ærekær
    borger. Jeg gik en måned uden arbejde og var hver dag på tæerne fra
    morgenstunden, men alle vegne den samme besked, desværre er der ikke
    brug for Dem. Det var en trøstesløs vandring fra sted til sted. Min
    moders søster havde et stort pensionat i Svendsgade. Hende hjalp jeg
    med at hente varer hjem og gjorde anden nytte, så der fik jeg min
    gode mad, så jeg betalte kun for logiet hjemme.
    
    En dag jeg gik på Kongens Nytorv, kom en tidligere svend hos min
    læremester mig i møde.  Han sejlede nu som første mester med s/s
    ''JEANNE'', rederiet Heimdal. Han og jeg havde arbejdet meget sammen,
    og han spurgte mig, hvorledes det gik. Snavs, sagde jeg. Kunne du
    tænke dig at komme ud at sejle? Det kunne jeg godt, for den
    drivertilværelser huede mig ikke. Han gav mig et kort, hvorefter jeg
    henvendte mig til forhyringsagent Jepsen i Nyhavn. Jeg fik en
    søfartsbog udleveret og blev påmønstret som \wordd{lemper}{71}, men
    jeg skulle gå dagvagt i maskinen fra kl. 6 til 6, og kun når det kneb
    med kullene, når de lå langt tilbage i bunkerne skulle jeg hjælpe til
    der.
    
    Min moder blev meget overrasket, da jeg kom hjem og fortalte, at jeg
    skulle ud at sejle. Jeg fik en dyne og en pude, og så sagde jeg
    farvel og tak og drog på langtur med s/s ''JEANNE''. Den var bygget i
    England og mandskabs\wordd{lukaferne}{72} var i stævnen. Det var ret
    ubehageligt, når man skulle midtskibs i dårligt vejr. Jeg arbejdede
    mest i maskinen med reparationer og vedligeholdelse.  Der lærte jeg
    meget. Vi sejlede først til Skutskär, en lasteplads nær
    \wordd{Gävle}{73}, hvor
    vi lastede træ til \wordd{Rouen}{74} i Frankrig. Oprejsen gik godt med fint
    vejr, men på rejsen igennem Østersøen til \wordd{Kielerkanalen}{75} havde vi
    meget hårdt vejr. Jeg var meget søsyg, og først da vi kom til
    \wordd{Holtenau}{Missing} ved \word{Kielerkanalen} søndag morgen, var skibet i ro igen, og
    kunne jeg have gået i land, da vi lå i slusen, havde aldrig nogen
    fået mig om bord i noget skib mere, men jeg måtte fortsætte, og
    efterhånden kom man over de børnesygdomme også. Da jeg mest opholdt
    mig i maskinen, kom jeg ikke så meget i berøring med gutterne forude.
    Det var brave mennesker, og de var ikke \wordd{brovtne}{76} over, at jeg havde
    en del lettere tjeneste end dem. De respekterede mig som fagmand, og
    jeg blev ommønstret til donkeymand, så
    ved land passede jeg \wordd{donkeykedlen}{Donkeykedel er en
    dampkedel, der leverer damp til et skibs dækspil (en slags kran)
    eller ankerspil (maskine der trak ankeret op). Donkeymanden er ham
    der passer kedlen.}, medens fyrbøderne havde rengøring
    af hovedkedlerne og maskinerne, når vi lå i land.
    
    Vi havde senere fået en \word{lemper}, en ung fynbo, som fyrbøderne gjorde
    megen grin med. Dengang var det således, at hver fyrbøder havde en
    ration af 2 flasker brændevin pr. uge i fast ration. Senere blev det
    ændret, så vi fik kakao udleveret i stedet for snaps. Den omtalte
    unge fynbo skulle jeg vække en dag i søen, inden han skulle på vagt
    klokken 8 morgen. Han blev \wordd{purret ud}{Purret ud betyder
    vækket.} klokken 7.30, da han skulle
    spise frokost, før han tørnede til. Jeg vækkede ham med de ord: Op at
    stå, du skal ud og skaffe. Det er jo udtryk for spisning om bord. Han
    svarede helt grædefærdigt på sit fynske mål: Hvad skal jeg skaffe, i
    går skaffede jeg to ankre brændevin, jeg kan da ikke blive ved med at
    skaffe! Det morede os alle, hans udbrud.  Han var jo ikke klar over,
    at jeg mente `frokost'.''
\end{quote}
    
    Fattigforsorgen er sparsom økonomisk hjælp fra det offentlige. Den blev
    anset for nedværdigende at få. Lemper er en, der skovlede kul ind i
    dampkedlerne. 72 Lukaf er et opholdsrum for mandskabet om bord. 73 Gävle
    er en svensk kystby ca. 150 km nord for Stockholm. 74 Rouen er en fransk
    kystby 50 km nordvest for Paris. 75 Kielerkanalen er en tysk kanal, der
    forbinder Østersøen med Nordsøen. 76 Brovtne betyder sure. 71
    
