\chapter{Advarsel}

Når først skibet er på rejse, er besætningen på mange måder overladt til
selv at klare mange situationer, som man på land er vant til at have
andre til at hjælpe med. Søfolk skal være forberedt på det
uforudsigelige som stormvejr, fare for forlis, arbejdsulykker, sygdom og
lignende. Selv når skibet lå i havn, lurede farerne. I Sydamerika var
kokkeeleven i land i følgeskab med en \wordd{matros}{Matros er en søman,
der har sejlet mindst tre år som dæksmand.} og en \word{fyrbøder}: 

\begin{quote}
    
    ``Vi fik hver et stort glas øl. Hvad der var i mit glas øl, aner jeg
    ikke, men jeg nåede ikke at drikke det hele, før jeg var væk. Totalt
    bedøvet. Hvad der skete i tiden fra kl. 21.00 til 24.00, aner jeg
    ikke noget om, for jeg vågnede først, da Crone smed mig op i min egen
    køje om bord. Han fortalte mig senere, at værten havde kommet noget i
    mit glas, jeg skulle have været \word{Shanghajet}{Shanghajet vil sige
    at blive bortført og tvunget til at være sømand.}. Værten vidste,
    jeg var kokkeelev, og en sådan gut manglede han netop til kokken, men
    takket være Crone gik det ikke denne gang''

\end{quote}

\begin{quote}
    
    ''Befolkningens fjendskab var åbenbar. Fra kommandobroen, hvor jeg
    gerne opholdt mig om dagen, for den lå jo højt over dækket og fangede
    derfor ethvert lille vindpust, så jeg, at nogle sorte sprang ned i
    druknehullet101 ved luge 1. Lige før var andenstyrmanden gået derned,
    Dernede på dækket lå en kvajl102 af svær fortøjringstrosse103, og
    pludselig så jeg styrmandens ben sprælle over kvajlen. Et par sorte
    baksede øjensynligt med ham. Jeg kom så hurtigt, som jeg kunne ned ad
    broen og hen på fordækket, og da de sorte så mig, tog pokker ved dem.
    De slap styrmanden og sprang op ad trappen på den modsatte side og
    ned ad stillingen til land, mens jeg knaldede med min skyder.
    Andenstyrmanden var grundigt stoppet fast i kvajlen og måtte hales
    op.''. 
    
\end{quote}


Kaptajnen var, ud over at være chef på skibet, både politimyndighed og
domstol. Han måtte finde en løsning, når et problem opstod, og ofte kunne
den virke både utraditionel og umenneskelig. Da et besætningsmedlem blev
psykisk syg, så kaptajnen ingen anden mulighed end at låse ham inde, til
de ankom til næste havn. Dette skete både for mandens egen og resten af
besætningens skyld: 

\begin{quote}
    
    ''Vi havde i Brathurst104 fået en ung neger om bord, en ganske
    primitiv ung mand, der var let at skræmme. Han mønstrede som
    kullemper, og han passede sit arbejde godt. Pludselig forsvandt han,
    og det blev fortalt mig, at fyrbøderne, som han boede sammen med
    forude, havde fortalt ham, at befolkningen på Sicilien var
    menneskeædere ligesom hans forfædre. Han måtte derfor passe godt på
    når vi kom i havn. Det var dum kådhed105 og ikke ondskab, og de var
    kede af det. I hvert fald var kullemperen borte, sprunget over bord
    troede vi, men en morgen fortalte Tommy, at han havde set lemperen
    liste ud af kabyssen lige før det blev lyst. Vi søgte overalt og til
    sidst fandt vi ham under en af kedlerne.  Han ville ikke komme ud, og
    da tredjemester mavede sig ind til ham, huggede han ham i skulderen
    med en kniv og mester måtte skyndsomt kravle ud igen. 
    
    Vi fik ham ud, og nu låsede vi ham inde i ''jernskabet'' forude. Det
    var et snævert rum med et lille koøje mod søsiden og lukket med en
    tyk jerndør ud mod druknehullet. Da vi havde børnene om bord, måtte
    jeg vide, hvor vi havde ham, den stakkel. Han var øjensynlig
    sindssyg. Vi passede ham med mad og andre fornødenheder, mens vi
    sejlede til næste havn. I de første to lossehavne måtte vi holde ham
    indelukket i jernskabet og det var han tilfreds med. Så kunne
    byboerne jo ikke få fat i ham, sagde han på sit ubehjælpsomme
    engelsk. Jeg forsøgte i begge disse havne at få ham på en anstalt,
    men der var ingen dansk konsul106, og de stedlige myndigheder ville
    ikke have ham. Først da vi kom til Palermo, hvor der var en konsul,
    der var dansk, lykkedes det at få ham fra borde. Stakkels lille mand.
    Han døde året efter på en anstalt.''

\end{quote}

Afstanden til lægehjælp kunne være lang og smertefuld: 

\begin{quote}
    
    ''Gammel sømand, som jeg var, så jeg godt, at der kom en sø, der
    ville bryde over skibet, og den gik også op over kommandobroen, og på
    mindre en et sekund blev jeg slået rundt med rattet, som jeg fik
    armen ind under. Til sidst satte armen sig i klemme mellem to knager,
    og der var så megen kraft på, at den ene knage gik af rattet. Men
    armen brækkede lige neden for albuen.  Den anden greb fat ved
    håndleddet, som blev drejet helt ned mod pulsåren. Det varede kun 2-3
    minutter, før jeg blev bragt ned i kaptajnens kahyt. Armen var vredet
    en halv omgang rundt og jeg måtte ligge i denne stilling i 4 døgn. Vi
    passerede da København og hejste flag, så lodsbåden kom på siden. Jeg
    blev bragt ind til Toldboden og skulle indlægges på det gamle
    Frederiks hospital i Amaliegade.'' 

\end{quote}


Somme tider kunne det være nødvendigt at bruge ufine metoder for at
forbygge sygdomme:

Druknehul er et lavtliggende dæk med lasteluger, blev på ældre dampskibe
ofte overskyllet i hårdt vejr. Kvajl er noget tovværk (en trosse), der
er skudt op (lagt sammen) i en rundkreds. At kvajle op betyder at rinke
op: lægge tovværk i ring. 103 Trosse er et reb til at fortøje et skib
med. 104 Brathurst ligger ved Gambiafloden i Gambia. 105 Kådhed vil sige
at være overfrisk. 106 Konsul er en stats repræsentant i udlandet. 102

\begin{quote}
    
    ``Vi havde tre \word{malaria}patienter blandt besætningen. Syge
    søfolk er vanskelige. De hader enhver form for piller og for øvrigt
    også al anden form for medicin, dog med undtagelse af \word{engelsk
    salt},
    og en kraftig dosis af dette klarede for øvrigt som regel så godt som
    enhver form for sygdom, måske med undtagelse af et brækket ben. Maven
    renset godt ud og manden er rask. 
    
    \word{Malaria}{Malarie er en infektionssygdom, der optræder især i
    varme lande. Smitte spredes med myg. Sygdommen behandles med kinin. I
    dag findes forbyggende medicin.} kræver imidlertid også
    \word{kinin}{Kinin er et lægemiddel til behandling af malaria.}, selv
    om den første behandling er `salt', men selv om den syge artigt
    putter pillen i munden og drikker vand til, så havner pillen under
    madrassen. Den bliver spyttet ud, så snart styrmanden er gået. 
    
    Det var ynkeligt at se på disse mænd. De var ingen ting til, før vi
    var oppe i køligere klima.  Jeg spekulerede over sagen og fandt så på
    at give besætningen flydende kinin som forebyggende middel. Når vi på
    vej sydover passerede de Kanariske Øer, stillede hele besætningen på
    halvdækket klokken 8 om morgen, og hver mand fik en snaps akvavit med
    en teskefuld kinindråber, og den gled altid ned. Jeg fortsatte hermed
    under hele opholdet på kysten, til vi igen passerede eller anløb de
    Kanariske Øer.''. 

\end{quote}


På de store passagerskibe blev der indrettet små hospitalsafdelinger med
lægefagligt personale, så man kunne klare de mest almindelige sygdomme og
små operationer. 

Risikoen for forlis var uforandret. Selv om dampskibene var nemmere at
sejle, skete der påsejlinger af både rev og isbjerge. I 1904 forliste et
dansk dampskib med udvandrere.  Passagerskibet NORGE sejlede 28. juni
1904 på et skær i Atlanterhavet, og ca. 600 af skibets 800 passagerer og
besætningsmedlemmer druknede, bl.a. fordi der ikke var nok redningsbåde.
Et andet mere berømt forlis skete 8 år senere, da TITANIC sejlede på et
isbjerg.
