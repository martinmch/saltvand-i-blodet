\chapter{En landkrappe}\label{en-landkrappe}

\begin{quote}
''Jeg besluttede nu at prøve dampskibene, men en dampskibsmatros blev i
de dage ikke regnet for rigtig sømand.''
\end{quote}

I 1920'erne var tiden for
sejlskibene ved at rinde ud. En økonomisk krise, som fulgte efter
afslutningen af 1. Verdenskrig, betød, at stort set alle sejlskibe blev
lagt op. Det vil sige, at de lå i havnene uden mandskab eller last om
bord. Ejerne ventede på bedre tider, men de kom aldrig. I stedet søgte
flere og flere arbejde på dampskibene, der nogle steder lige akkurat
kunne holdes kørende økonomisk. Men arbejdsløsheden var stor, og mange
søfolk gik i land og ventede på at komme ud at sejle igen. For at få
arbejde måtte man være beslutsom: 

\begin{quote}
''Jeg var på Styrmandsforeningens61
kontor og meldte mig i foreningen. Gik derfra med en negativ indstilling
til fremtiden. Der var ingen håb om hyre derigennem. Jeg husker ikke,
hvor mange ledige styrmænd foreningen havde gående, men det var mange.
Forretningsføreren anbefalede mig, at rejse hjem til Marstal, der var
måske større chance 60

Trampfart er, når et skib sejler i fri fart mellem forskellige havne,
hvorfra og hvortil der skal transporteres varer. I modsætning hertil er
der skibe, der er placeret i linjefart med gods mellem to eller flere
bestemte havne. 61 Styrmandsforeningen er en fagforening for styrmænd.

16

for at få en hyre med et motorskib, eller et af de få sejlskibe, der
endnu var tilbage, chancen var sikkert større der, end den var i
København. Det var ikke just lyse udsigter. Nå, jeg sagde farvel til
Hekla, kom i land ved Toldboden og var indstillet på at løbe på trapper
til de forskellige rederier resten af dagen, jeg måtte jo i alle
tilfælde have en ansøgning liggende, hvis der pludselig skulle vise sig
noget. Da jeg kom gående op i Bredgade mod Kongens Nytorv på vej til
rederiet A. P. Møllers kontor, dette var vel det, der var mest kendt i
Marstal. Der var mange styrmænd herfra, der blev ansat netop der.
Pludselig fik jeg øje på et blank-pudset skilt, hvorpå der stod
Dampskibsselskabet Torm, Bredgade 6. Det var et selskab jeg ikke kendte
meget til eller havde hørt om. Jeg havde nogle gange set et mindre
sejlskib med et T i skorstenen passere holmen for at gå op i
gasværkshavnen, men ellers kendte jeg intet til dette selskab. Det var
mest mærskselskaberne, vi hørte om i Marstal. Ja, du skal nok igennem
alle rederierne, tænkte jeg, så hvorfor ikke øve dig lidt med at begynde
her? Jeg kom ind på kontoret og fremsagde mit ærinde, der blev rystet på
hovedet og peget på en hylde med en stor stak papirer. Det er bare
ansøgninger, blev der sagt. Jeg spurgte om rederiet havde trykte
formularer til ansøgning, eller om man bare kunne sende en ind med kopi
af ens papirer, så måtte den vel komme til at ligge øverst. Han spurgte
med et lille smil, om han måtte se mine papirer. Jeg havde ret gode
eksamener, havde også en anbefaling fra min tid i BONAVISTA (et
sejlskib). Han kiggede papirerne igennem og sagde: et øjeblik! og
forsvandt ind i et af kontorerne. Et øjeblik efter kom han tilbage uden
mine papirer, lukkede lemmen op, idet han sagde, at direktør Kampen
gerne ville hilse på mig. Jeg kom så ind til selve chefen, som sad og
kiggede på mine papirer. Vil De gerne ind i vores rederi, var hans
første spørgsmål. Ja tak, svarede jeg. Jeg ville nu hellere have været
ind i et noget større rederi, et der havde større skibe. Torm havde vel
dengang en 16-18 skibe. Den største lastede omkring 3000 tons og de
mindste omkring 16-1800 tons. Men det var ikke tider at vrage, masser af
styrmænd gik ledige. Så måtte man være glad og tilfreds for et sådant
tilbud. Ja, men så er det i orden, svarede han og rakte mig mine
papirer. Opgiv Deres adresse til hr. Jensen ude i forkontoret, rejs hjem
og hold en lille ferie og De vil høre fra os i løbet af ca. 14 dage. Jeg
gav hr. Jensen min adresse. Med et lille smil ønskede han mig god ferie,
og jeg rejste hjem. Den 20. oktober mønstrede jeg som andenstyrmand om
bord på dampskibet SIGNE af København. Jeg kom om bord på Helsingør
red62, og det var første gang jeg satte mine fødder på dækket af et
dampskib.'' 
\end{quote}

Forholdene på et dampskib var på mange områder helt
anderledes end på sejlskibene. Man fik som regel en bedre løn, og der
var ofte bedre arbejdsvilkår. Men dampskibene var også genstand for
faglige kampe, og det lå i luften, at man som matros måtte deltage
aktivt. Det kunne dog være en dyr fornøjelse og i sidste ende føre til
arbejdsløshed på ny. 

\begin{quote}
''Nu var jeg helbefaren matros 22 år gammel, og jeg
blev stoppet til orlogs63, og jeg søgte arbejde i Frihavnen og lærte at
betjene et dampspil. Jeg fik undervisning af dem, jeg skulle arbejde
sammen med, og jeg opdagede snart, at det var betydeligt lettere med
mekanik end at stå i et krøbbelspil64 og lægge alle kræfterne i et
håndsving hele dagen for at få kul eller anden last op af rummet. Jeg
tjente 4 kroner den dag, og jeg havde aldrig før tjent så mange penge på
en dag og så let. Jeg tænkte, skal du gå ombord på en damper, og jeg tog
min beslutning.

62

Red er en ankerplads uden for en havn, flodmunding og lignende. Stoppet
til orlogs vil sige, at personen har aftjent sin værnepligt i militæret.
64 Krøbbelspil er et ankerspil (med vandret spiltromle) 63

17

Sømændenes Forbund65 var næsten lige trådt ud i livet, og jeg blev
medlem. Den danske sømand vågnede i forståelse af, hvor urimelig lille
hans betaling var, for det arbejde, han udførte. Matroshyren var 50
kroner om måneden. Nu krævede vi den sat op til 60. Det ville rederne
ikke gå med til, og den første danske sømandsstrejke blev etableret. Et
knusende nederlag blev resultatet. Der var ingen strejkefond66, så hver
enkelt måtte bære byrderne, og bevægelsen blev fuldstændig sprængt.
Efter dette måtte man for at få hyre underskrive en erklæring på, at man
ikke var tilsluttet nogen fagbevægelse. Rederne indførte karakterbog,
som tilkendegav, hvordan opførsel og duelighed var beskaffen, men det
mudrede blandt søfolkene. Der gik nogle år, og så samledes vi et halvt
hundrede stykker i den lille kælder i Toldbodgade, hvor kølen til det
nuværende forbund blev lagt takket være dygtige mænds oplysningsarbejde.
Den næste strejke vandt vi, men måtte underskrive en otteårig
overenskomst.''
\end{quote}
