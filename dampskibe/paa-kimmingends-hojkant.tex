\chapter{På kimmingens højkant}\label{puxe5-kimmingens-huxf8jkant}

Søfolk kommer ud i den store verden og møder andre kulturer og tit andre
politiske systemer. Den unge ''Messepetter'' Richard Lauritzen var
mønstret på damperen ASTRID i København i 1934 (se Kapitel 10 - Høj og
Lav). Den første udenlandske havn, de anløb med skibet, var Danzig109 i
Polen. Det var ikke så langt væk, men dengang var der ting og sager i
gære, der adskilte Polen fra det fredelige Danmark: ''2. styrmanden gav
mig et frimærke til et brev hjem. Han advarede mig mod at gå i land
alene. Der er tit uroligheder i byen, sagde han. Den er jo delt mellem
tyskere og polakker, og det går skidt mellem dem, siden nazisterne har
taget magten i Tyskland. Kokken Jens Pedersen, som jeg nu var på fornavn
med, gav mig samme råd. -- Danzig har været en god by, men nu er den
tyske befolkning tosset, fordi Hitler vil befri den for det polske
tyrani. Det rene sludder. Men du kommer ikke i biografen eller andre
steder i denne by uden at blive bombarderet med nazisternes modbydelige
propaganda110! Jeg ville ikke trodse de velmente råd. Men jeg måtte føle
den fremmede jord under mine fødder. En aften gik jeg i land, bare for
en tur på kajen under den funklende stjernehimmel.'' Polen -- ikke så
langt væk, men dog alligevel så anderledes. Messedrengen kom snart
længere væk og ud på det dybe vand og så den fjerne store verden: ''Vi
var i Casablanca111. Vi var endelig i havn, og jeg havde set de første
arabere i mit liv. Et spændende folkefærd, som jeg ikke havde lært spor
om i skolen. Men var de alle sammen så farlige, som 2. styrmanden havde
fortalt? Det var svært at tro. Med solens frembrud var åbenbart et
signal for havnens brogede mængde af handelsmænd, for pludselig hørte vi
et farligt spektakel af råb og skrig og klaprende fødder omme på
landsiden. -- Det er alle ''prakkerne112'', der stormer os, sagde
hovmesteren. -- Så er det slut med freden. I den babyloniske palaver113
syntes jeg at høre nogle danske gloser, og det fik mig til at råbe ind i
larmen: Skildpadder! Har I skildpadder? Virkningen var ganske
overraskende for mig. Mit kraftige udbrud bragte flokken til tavshed, og
mange løftede armene mod himlen som i fortvivlelse og rystede på
hovedet. Ingen af dem havde skildpadder på varelisten. Det var en
skuffelse, for jeg ville ikke købe noget som helst andet. Flere prøvede
dog at friste mig med deres ting og sager, det var små fine tæpper,
nydelige skindpuder og tasker, perlesmykker og lignende, men det nyttede
ikke. -- Skildpadde, sagde jeg. Det ville jeg købe, gentog jeg. Flokken
var ved at opgive mig, men så greb en gammel rynket ''Gandhi'' min hånd
og sagde. - My friend -- I will bring you a very nice turtle here in the
afternoon. I promise; in the afternoon! Den gamle ``Gandhi'' kom senere
hen langs kajen og gentog sit tilbud: - You want turles \ldots{}. I will
bring you a very nice one in the afternoon! -- Yes, yes! Råbte jeg og
vinkede -- I understand you very good! -- Det var ikke perfekt engelsk,
men vi forstod hinanden. Og den gamle marokkaner holdt også sit løfte.
Samme aften skrev jeg et langt brev hjem.''.

109

Danzig er en by i Polen. Propaganda er politisk reklame. 111 Casablanca
er en by i Marokko (på atlanterhavskysten). 112 Prakkerne er
handelsmænd, markedssælgere, der prøver på at prakke folk en salgsvare
på - det vil sige, at de prøver at overtale dem til at købe. 113
Babylonisk palaver vil sige, af folk taler flere sprog i munden på
hinanden -- og at samtalen virker indholdsløs. 110

28

Skibet gik herefter til Dakar for at få lastearbejdere om bord, idet de
skulle laste jordnødder i Kaolack i Senegal. Richard havde læst om
Afrika og om de kendte opdagelsesrejsende Livingstone og Stanley, og nu
lå han selv oppe på en afrikansk flod: ''Det var da helt utroligt, at
jeg nu også befandt mig langt inde i det sorte Afrika. Jeg var i
Kaolack, et sted i urskoven med en primitiv lasteplads og en gammel
negerlandsby, der lignede illustrationerne i bogen om Livingstone og
Stanley. Hertil var jeg kommet ved en sejlads på 45 sømil ad en flod med
grumme krokodiller og vældige flodheste, og gennem en tæt urskov, der
vrimlede med skrigende aber og farvestrålende fugle. Undervejs havde vi
to nætter ligget for anker på floden, og da havde jeg lyttet til
junglens mange mærkelige lyde og hørte også flere gange de store
Senegalløver brøle i natten. Ja, jeg var i Afrika, og det var ikke mere
end godt tres år efter mødet mellem Livingstone og Stanley. Det var
næsten ikke til at fatte. Men virkeligt var det. ASTRIDs besætning talte
nitten mand af den hvide slags, men vi blev næsten borte for hinanden i
det mylder af kulsorte afrikanere, vi havde fået om bord i Dakar. Det
var et arbejdshold på tres mand, meget uens af alder og fysik. De skulle
besørge lastningen af os, da der ingen arbejdskraft fandtes i Kaolack.
Disse mange arbejdere var fremmede på lastepladsen og havde derfor
ophold om bord hos os under hele turen frem og tilbage til Dakar, hvor
de havde hjemme.'' Messedrengen Richard mødte således fremmede kulturer
og anderledes mennesker på nært hold ude i den store verden, som man
mødte som sømand (charterrejser og ryksækturisme lå dengang stadigvæk
lysår væk for almindelige mennesker), og det var ikke lige skræmmende
det hele, han mødte. Her et lille uddrag af brev hjem til forældrene:
''Bare I kunde se alt det, som nu foregaar om bord i ASTRID, det er en
stor Oplevelse. Da jeg for lidt siden sad og spillede paa Mandolinen,
kom en Neger og stak Hovedet inden for og lo. Han er en pæn Fyr, og jeg
synes godt om ham. Han hedder Thomas, siger han. Nu ville han vise mig
et par gamle Sko, som han havde faaet af den hjertevarme Hovmester. Hvor
var han dog glad for de Sko.'' Richard har således gode og spændende
oplevelser baseret på hans samtidige breve til hjemmet. Officererne,
herunder hans boss om bord, hovmesteren, var flinke folk. Ofte får man
ellers det indtryk, når vi læser romaner, at officererne til søs var
grumme folk i gamle dage. Men der var altså også fordragelige forhold og
gode folk i gamle dage, og det var jo rart for en lille knægt, der drog
ud i den store verden. Selvfølgelig var tonen ofte bramfri114, og nogle
folk kunne være temmelig håndfaste, men sådan var forholdene også i land
på de tider.

114

Bramfri vil sige at tale frit, uden at pakke tingene ind.

29

Coastere En coaster er et mindre tørlastskib115, der kan sejle gods fra
havn til havn. En coaster er et skib, der kan medtage en godsmængde på
mellem 500 og 4000 tons. Søfart er i sagens natur et internationalt
erhverv, hvor det engelske sprog spiller en stor rolle. Derfor er mange
engelske ord gledet naturligt ind i dansk søfart længe før denne tendens
begyndte at gribe om sig generelt. Ordet ''coaster'' kommer således også
fra engelsk, hvor coasterbegrebet oprindeligt anvendtes om skibe i
kystfart116. I dansk søfart dækker coasterbegrebet et mindre skib, der
både sejler i de kystnære farvande og sejler i fart over længere
afstande -- for manges vedkommende på alle have -- world wide, som det
hedder inden for søfarten. Den type skibe, der nu til dags kaldes
coastere, blev kaldt for motorskibe fra starten, hvilket vil sige årene
efter Anden Verdenkrigs afslutning, altså i 1950'erne.
Coasterbetegnelsen slog først rigtigt igennem omkring 1960. Den gang
dækkede begrebet skibe, der kunne medtage fra ca. 200 tons og op til ca.
800 tons. Men som alt andet i denne verden voksede størrelsen igennem
årene, grænsen er flydende med opadgående tendens. Danske coastere var
talrige før i tiden, ikke mindst i 1970'erne. Dengang gik størrelsen op
til en 12-1400 tons. Man kunne møde dem overalt. Skibe med en lasteevne
på under 7- 800 tons, der tidligere var særdeles talrige i den danske
handelsflåde, er imidlertid i dag stort set sejlet helt ud af dansk
søfart. Ligeledes tynder det fortsat ud i rækkerne af coastere i den
næste kategori, der har en lasteevne til op omkring 2500 tons.
Coasterne/motorskibene sejlede i epokens begyndelse mest i fart på
europæiske og nordatlantiske havne, og for enkelte gik turen også ud på
den virkelige langfart. En stor del var beskæftiget med at transportere
alle mulige varer i nordeuropæiske farvande - også i danske. Men i
nærfarten transporterede lidt mindre skibe den største varemængde inden
for det, der kaldes småskibsfarten117, hvilket vil sige skibe under
coasterstørrelsen. Småskibsfartens skibe er nu væk, idet jernbaner og
lastbiler har udkonkurreret godstransport i nærfarten. En årsag til
dette er et omfattende byggeri af broer og veje samt en afgift på den
ellers miljøvenlige søtransport. Coastere besejler fortsat en hel del
danske havne, hvor de kommer for at losse eller laste gods eller for at
blive repareret. Mange havneafsnit er imidlertid i dag afspærret med
hegn, en såkaldt terror-sikring. Men der findes dog fortsat havne, hvor
man kan gå helt ned til det virkelige havneliv og klappe en coaster
eller et andet skib på skibssiden.

115

Tørlastskib er et skib, der sejler med gods, der ikke er flydende, i
modsætning til tankskibe. Kystfart vil sige skibe, der sejler i nærheden
af land, og således ikke kommer ud på lange rejser over store åbne have.
117 Småskibsfart er et udtryk for små skibe, der kun kan medbringe en
mindre godsmængde - op til ca. 200 tons - og sejler i kystfart. 116

30
