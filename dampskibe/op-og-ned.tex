\chapter{Op og ned}\label{op-og-ned}

Til søs er der et bestemt besætningsmønster, en rangorden med
skibsføreren som den øverste. Dette gælder i alle skibe. I sejlskibene
havde alle haft det samme udgangspunkt fra begynderjobbet som dæksdreng
over jungmand og videre til letmatros, matros og bådsmand Og for dem,
der havde ønsket, viljen og evnerne, lå jobbet som styrmand og
skibsfører også inden for rækkevidde. Dette mønster fortsatte i
dampskibenes periode.

I dampskibene havde dæksmandskabet dog andre
opgaver end i sejlskibene. Sejlskibssømanden havde ud over
vedligeholdelse af skibet en del sømandsarbejde, der bestod i arbejde
med tovværk og med at passe sejl, mens en meget stor del af
dampskibssømandens tid gik med at vedligeholde skibet.
Sejlskibssøfolkene kaldte nedladende dampskibsfolkene for
''rustbankere''. Til gengæld kunne dæksfolkene i dampskibene som regel
få deres \word{frivagt} i fred og ro. Derimod blev sejlskibssømændene ofte
purret ud, når der var sejlmanøvre, eller når vinden friskede, og der
skulle \wordd{mindskes sejl}{Mindske sejl vil sige, at når vinden frisker
op, er man nødt til at mindske sejlfladen på sejlskibene, for at skibet
ikke skal krænge for meget over til siden eller kæntre. Man bjærger nogle
af de øverste mindre sejl (det vil sige, at man tager dem ned), eller
gøre sejlfladen mindre på andre måder.}. Dæksfolkene i dampskibene
havde lignende
arbejdsopgaver som coastersøfolkene. 

I dampskibene kom der nye kategorier af folk om bord. Maskinen skulle
passes, og til det formål var der \wordd{kullempere}{Kullemper er en
ufaglært person, der er ansat til at bringe kul frem til fyrpladsen ved
maskinen, således at fyrbøderne har tilstrækkeligt med kul til at fyre
med. Han er en form for alt-mulig-mand, maskinens dæksdreng.},
fyrbødere,
\word{donkeymænd}, og
maskinassistenter samt maskinmestre, hvoraf 1. mesteren var den øverste.
Der gik megen tid med at manøvrere, reparere og overvåge maskinen -- og
med at fyre op under dens kedler.

Desuden havde de større dampere så stor en besætning, at der var en hel
afdeling inden for madlavning og servering med messedrenge, opvaskere,
\wordd{koksmather}{Koksmath er en kokkelærling.} og kokke med hovmesteren i spidsen. I hvert område på
skibet -- maskinen, kabyssen eller dækket -- arbejdede folk, der
udviklede hver sit speciale. Gruppen af officerer var sat sammen på
tværs af fag. De havde uniformer i modsætning til sejlskibsofficererne,
der gik i dagligt tøj om bord.

Richard Lauritzen, der var født 1919 i
Hvorup nord for Nørresundby, havde været en kort tid med småskibe i
nærfarten, og i 1934 ville han på langfart. Han havde heldet med sig hos
en \wordd{forhyrings}{Forhyret betyder ansat.}-agent i København. Han skulle mønstre som messedreng i
S/S ASTRID af København, der kunne medtage en last på 2600 tons og havde
en besætning på 20 mand. 

\begin{quote}

    Richardt beretter: ''Det er så i orden, sagde forhyringsagenten. Du
    kan begynde om bord i morgen klokken syv. Dit skib ligger ude i
    Sydhavnen, og du melder dig bare hos hovmesteren eller kokken, når du
    møder! Med køjesækken på nakken gik jeg forsigtigt hen under de
    svingende kraner til landgangen, vejen op til min nye spændende
    tilværelse. - Hvem er du? -- Det var vagtsmanden en lille vims
    matros, som jeg siden skulle kende som ''Småen''. Han godtog min
    redegørelse og viste mig rundt overalt om bord og bød mig på kaffe i
    kabyssen. Klokken 6 næste morgen blev jeg purret ud af vagtsmanden.
    Jeg var lysvågen med det samme.
    
    Så er det ud af bunken ''Petter''! kommanderede ''Småen'', påtaget
    barsk.  Du skal have kaffen på bordet i \wordd{messen}{Messe er
    spisestuen på et skib.} om en halv time, ellers får du troubles!
    Javel! sagde jeg og sprang ud på dørken. Det var den første dag. Jeg
    ville gerne være pæn, og mit tøj var også rent. Men jeg havde endnu
    intet vand i kammeret til at vaske fjæset med. Mit stride hår føjede
    sig dårligt for kammen, det skulle have vand for at arte sig. Jeg
    måtte klare det med lidt spyt. Og med et håndklæde gned jeg søvnen af
    øjnene. Ude i den lukkede gang gennem midtskibs kunne jeg trods de
    larmende kraner høre, at nogen rodede med grej henne i kabyssen. Den
    lå agterude midtskibs, ikke langt fra mit kammer. Der måtte jeg hen
    og bede om den vejledning, jeg savnede for mit arbejde.
    
    Det var tydeligvis kokken, der var begyndt i kabyssen. Han måtte være
    kommet lige fra land, for han gik i pænt jakkesæt og så forvirret ud,
    havde nok ikke sovet, tænkte jeg. Han ænsede ikke min nærværelse, før
    jeg stak hovedet ind over den åbne halvdør og højt sagde godmorgen.
    Nå, den travle mand sendte mig et hurtigt blik. Er det så vores nye
    ''Messepeter''? spurgte han. Ja, sagde jeg, og jeg vil godt spørge. -
    Du skal dække bord til seks i messen først, og så henter du brød og
    smør hos hovmesteren i \wordd{pantryet}{Pantry er et anretterrum,
    spisekammer.}. Når det er gjort, purrer du hele ''sværten'' ud, det
    er maskinmestrene og styrmændene. Og du skal kun banke på dørene og
    ikke bryde ind, for nogen af dem har besøg af kællingen, forstår du!
    -- Javel, sagde jeg spagt, men hvad så? - Se så at komme i sving dit
    skvadderhoved! 
    
    Kokken var ædru inden middag. Det havde forandret ham til det bedre,
    og han blev min flinke rådgiver. -- Vi to har næsten samme
    arbejdstid, forklarede han. Når vi ligger i land, bliver vi purret ud
    kl. seks og pukler forskelligt til omkring en time efter middag. Hvis
    vi når det hele til den tid, kan vi tage halvanden times hvilepause,
    inden kaffen skal brygges og serveres. Derefter har du i al fald nok
    at gøre til kl.  syv-halv-otte. Men hvordan er det så, når vi sejler?
    spurgte jeg. Er det anderledes? Næsten det samme, sagde kokken. Vi
    går jo ikke vagt, men vi skal på benene om natten fra klokken tre til
    fire-halv-fem og sørge for kaffe til skiftet mellem hundevagten og
    morgenvagten. Bagefter kan vi så lægge os til klokken seks. Jeg kom
    til at opleve, at mine vilkår just var sådan.''

\end{quote}

Richard kom en del forude i skibet hos dæksmandskabet, for det var hans
mål at komme til at sejle på dækket -- altså at gå matrosvejen. Tonen
hos dæksmandskabet var betydelig mere rå end blandt officererne. Herom
fortæller han: 

\begin{quote}

    ''Som messedreng, måtte jeg føle mig forkælet med eget
    lukaf midtskibs, men den tid var kort, og jeg fik siden mine årelange
    erfaringer som dæksmand i skibe med samme forhold forude som i ASTRID.
    Og da forstod jeg, hvorfor tonen \wordd{under bakken}{Bakken er det
    forhøjede dæk ude i forskibet. Under baksdækket var der indrettet
    beboelse for det menige mandskab. Det var ikke rart at bo ude under
    bakken, for her var man i høj grad udsat for skibets bevælgelser og
    larm i hård vejr. Ligeledes kunne vejen til og fra bakken i hårdt
    vejr være en farefuld og våd tur hen over dæk og luger.} kunne være
    så barsk og nærgående grov. Der var plads til olietøj, søstøvler og
    \wordd{wirehandsker}{Wirehandsker er arbejdshandsker} --- men ikke
    til englevinger.'' 

\end{quote}

Når tonen kunne være temmelig barsk i dampskibene, skyldes det bl.a. at
der var mønstret en del fyrbødere og kullempere ud i dampskibene fra
storbyens slum, og de var ikke Guds bedste børn alle sammen.
