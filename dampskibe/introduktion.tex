\part{Dampskibe}\label{part:dampskibe}

En damper - eller et dampskib - er et fragtskib, der bliver drevet frem
af en dampmaskine og er bygget af jern eller stål. I den allerførste tid
blev der også blev bygget dampskibe af træ. Dampfragtskibe begyndte at
gøre sig gældende fra omkring 1875 til 1900. Ganske vist var der før den
tid så småt begyndt at komme dampere i indenrigs- og nærfarten, men det
var hovedsagelig til passager- og postruter og faste fragtruter. I 1870
var der således kun 4 danske dampskibe i trampfart60 - altså i fri fart
mellem forskellige havne - i nordeuropæiske farvande, men i slutningen
af perioden begyndte dampskibene at gå på længere rejser.
Sejlskibsflåden var hjemmehørende i havne placeret rundt omkring i hele
kongeriget, specielt koncentreret i byerne omkring Det sydfynske Øhav.
Med dampskibstiden ændrede dette mønster sig. Dampskibsrederierne blev
centreret omkring hovedstaden, hvor næsten hele dampskibsflåden blev
hjemmehørende. Her møder man dog også undtagelser som eksempelvis
rederiet J. Lauritzen, der indtil 1914 havde hjemsted i Esbjerg,
hvorefter firmaet flyttede til København. Der var også forskellige
dampskibsrederier i Aabenraa og rederiet H. C. Christensen i Marstal.
Dampskibenes størrelse varierede meget. De mindste af slagsen fra
dampskibstidens begyndelse kunne medtage en last på et par hundrede
tons. De senere kunne medtage adskillige tusinde tons fragt. Tilsvarende
var der meget stor forskel på besætningens størrelse. Dampskibene
sejlede i epokens begyndelse i de nære farvande, men senere gik turen
også ud på langfart. En stor del af dampskibene var beskæftiget med at
transportere alle mulige varer i nordeuropæiske farvande. Der var typisk
tale om landbrugsprodukter til Storbritannien og kul tilbage til de
danske kakkelovne, der opvarmede hjemmene, før vi fik oliefyr og
fjernvarme. Røgen fra de sidste danske lastdampere er for længst
forsvundet i den blå luft. Men man kan stadigvæk komme på en tur med et
af de tilbageværende næsten lydløse dampskibe, nemlig passagerskibene
SKJELSKØR og HJEJLEN. De blev bygget henholdsvis i 1915 og 1861. HJEJLEN
sejler stadig på Silkeborgsøerne.
