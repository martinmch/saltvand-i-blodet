\begin{Abstract}
    \addcontentsline{toc}{chapter}{Abstract}

Bogen er opdelt i fire epoker, opdelt i syv kapitler. Hver epoke
fortæller om søfolkene, skibene og mødet med fremmede verdensdele på
hvert sit tidspunkt i historien. Bogen kan være et eventuelt tillæg til
historieundervisning, eller anden maritim undersøgelse og/eller
basisforståelse.

\section*{Sejlskibsepoken}

I sejlskibsepoken blev de fleste varer transporteret rundt i verdenen
med sejlskibe. Industrialiseringen betød, at der blev behov for at
transportere flere og flere varer mellem byerne og mellem
landdistrikterne og havne.

\section*{Dampskibsepoken}

Et dampskib er et fragtskib, der bliver drevet frem af en dampmaskine.
En stor del af epokens dampskibe transporterede varer rundt i de
nordeuropæiske havne, men turen gik dog også på langfart. Forholdene for
søfolkene forandrede sig under dampskibsepoken, og søfolkene begyndte
også at organisere sig i foreninger og forbund.

\section*{Coasterepoken}

En coaster er et mindre fragtskib, der drives frem af en motor. En
coaster sejler både i de kystnære farvande samt på langfart. I 1970'erne
var der mange danske coastere, mens der i dag er meget få tilbage.

\section*{Containerskibsepoken}

Mange af de varer som vi forbruger til dagligt er kommet hertil med
containerskibe. Globaliseringen har betydet, at mange varer produceres
billigt i Fjernøsten, hvorefter de transporteres hertil med
containerskibe. Udviklingen fra sejlskibe til containerskibe har betydet
mange forandringer for sømanden, der i dag skal kunne håndtere teknisk
udstyr frem for sømandsknob.
\end{Abstract}
