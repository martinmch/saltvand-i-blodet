\chapter{Lev vel}\label{lev-vel}

\begin{quote}
``Efterårsdage blev vi aldrig vasket i søen. Heller ikke om vinteren.
Det var alt for koldt at tage noget tøj af og stå der og vaske sig. Det
kunne man ikke. Sommerdage der på Østersøen kunne vi jo tage
udenbordsvand.''
\end{quote}

Hverdagen på et sejlskib kunne være hård. Arbejdet var krævende, og de
hygiejniske forhold ikke for gode. Man boede trangt, kunne ofte ikke
komme i bad, og kosten var sparsom. Madlavningen foregik i et lille
aflukke (kabyssen) placeret midt på dækket, hvor der lige var plads til
at vende sig. Maden blev lavet på et lille komfur, hvor røgen let kunne
slå ind og fylde rummet. Det var heller ikke altid nemt at lave mad, når
skibet gyngede kraftigt i en storm eller en forkert sø. Madlavningen på
de mindre sejlskibe blev som regel altid foretaget af yngste mand
ombord. På de mindste skibe kunne det være forbundet med hårdt arbejde.
Foruden madlavningen skulle kokken også hjælpe til med alt det andet om
bord. Om vinteren kunne det dog have sine fordele at være i
kabyssen\footnote{Kabyssen er køkkenet om bord på et skib.}.

\begin{quote}
``Kabyssen var det eneste sted der var varme. Der var ikke varme nogen
steder om vinteren undtagen kabyssen. Det var det eneste sted vi kunne
få tørret et par strømper.''
\end{quote}

Kosten var meget forskellig fra skib til skib. Meget afhang af, om
kokken kunne lave god mad ud af den proviant, man havde med i skibene.
På nogle skibe spiste man godt, mens der på andre blev sparet på kosten.
Når først skibet var afsejlet, kunne man ikke gøre så meget.

\begin{quote}
``Der var meget forskel på det. Ja --- uha. Nogle var hele
sultekasser\footnote{Sultekasser er skibe, hvor man fik for lidt at
  spise.} og andre var nogenlunde. De sparede på kosten for at give
rederiet\footnote{Rederi er et firma der ejer et eller flere skibe.}
mere. Det var jo om at få overskud på de skibe. Jo mere de sparede på
kosten jo mere tjente de jo for pokker. Vi kunne godt klage, men når vi
ikke havde ud-provianteret mere og lå i søen, så var man nødt til at
indordne sig.''
\end{quote}

Frisk kød var en sjældenhed. De store skibe, der krydsede oceanerne,
havde dog ofte levende grise om bord. Der var også mulighed for at
supplere kosten med fiskefangst. Fik man harpuneret en haj eller en
delfin, kunne man være heldig at få helt frisk kød. Dårlig eller for
lidt kost kunne føre til mangelsygdomme og svække immunforsvaret. Og det
kunne være farligt, når man kom til byer, hvor der var udbrudt epidemi.
Kolera\footnote{Kolera er en smitsom mave-tarm-sygdom, som ofte dukker
  op i forbindelse med urent drikkevand.}, dysenteri\footnote{Dysenteri
  er smitsom tarmbetændelse med blodig diarré. Opstår som følge af
  dårlig hygiejne i troperne.}, tyfus\footnote{Tyfus er en art
  blodforgiftning og den farligste af alle salmonellainfektionerne.
  Opstår som følge af forurenet mad og drikke i troperne.} og lignende
skavanker var kendte fænomener blandt søfolkene. Alle frygtede at blive
syge. Tænkte man sig ikke om, kunne det betyde, at man måske døde.

\begin{quote}
``I Petersborg stod vi og handede\footnote{Handede betyder ''rakte'',
  ''videregav''.} sten. Så havde vi godt nok fået at vide, at der var
kolera. Vi måtte endelig ikke gå i vandet. Vi havde jo jollen liggende i
vandet agterud, og når så vi havde stået og handet næsten op fra lasten
og op i land fra 6 morgen til 6 aften og svedt, og det var egentlig
hårdt arbejde, så kunne vi jo ikke nære os om aftenen, andet end at vi
skulle lige ned og skylle os en bitte, og der var ingen, der så det, når
vi gik ned i jollen og så blev vasket en bitte\footnote{Bitte betyder
  ''lille''.}. Men det resulterede alligevel i, at vi havde en tysker om
bord fra Königsberg, og han fik kolera og blev lagt i land og han døde
dèr. Vi skulle jo have holdt os væk, men vi andre, vi gik da fri.''
\end{quote}

En anden plage var søsygen\footnote{Søsyge opstår som følge af ubalance
  i balancesansen. Kendetegn er kvalme, depressiv tilstand, opkast og
  mavesmerter.}, som naturligvis var en del af sømandslivet. De fleste
søfolk havde på et eller andet tidspunkt oplevet søsyge. For nogle var
det en konstant plage, mens søsygen for andre var et engangsfænomen. En
slem søsyge kunne tage modet fra de fleste.

\begin{quote}
``Om sommeren lastede vi kridt til Kotka, og det blev dårligt vejr. Vi
lå underdrejede oppe i bugten ved Øland. Jeg mener vi lå der i halvandet
døgn, og drengen var så søsyg\footnote{Søsyge opstår som følge af
  ubalance i balancesansen. Kendetegn er kvalme, depressiv tilstand,
  opkast og mavesmerter.}, så ganske forfærdelig søsyg, og han lå og
rullede rundt i vandet. Det var frygteligt at se. Vi fik så meget vand
over, at vi næsten ikke kunne komme ned i forlogaret. Så sagde jeg til
skipperen: Nu lægger jeg ham ned i min køje\footnote{Køjer er senge om
  bord på et skib.}. Ja, han ville ikke have ham derned og
brække\ldots Jamen han har ikke noget at brække af, det kan du godt tro.
Han kom så med der. Så lå han dernede en dags tid. Han var aldrig søsyg
mere.''
\end{quote}

Bedre blev det heller ikke af, at toiletforholdene var meget primitive.

\begin{quote}
``Ja, se vi havde en lille tønde, f.eks. styrmand og skipper de havde jo
henne i hytten agter, der havde de jo en spand, og den skulle kokken jo
holde, sørge for at tømme hver anden dag. Men mandskabet forude, de
havde jo bare en rund tønde med to stropper i, og når vi så lå i havn,
så skulle de sørge for at krybe i læ et eller andet sted, særlig hvis
der gik damer på kajen. Der var i hvert fald tønde, og så fyldte vi en
halv spand vand i og så ud over siden med det. Men altså, vi havde ikke
andet end den åbne tønde.''
\end{quote}

Hvis sejlskibet var af ældre dato, kunne det være plaget af småkryb. Den
erfarne sømand vidste, at det var bedst at holde sig væk fra skibe med
skadedyr, hvis man kunne.

\begin{quote}
``Det var meget almindeligt dengang med sådan utøj om bord. I
barkentinen HAABET her af byen var der mange væggelus. Jeg var
mønstret\footnote{Mønstre er at gå om bord og få hyre på et skib; det er
  altså begyndelsen på jobbet.} i den og skulle med den ud at sejle. Så
var de andre kommet om bord, men jeg sov jo hjemme. Først lå de i
køjerne\footnote{Køjer er senge om bord på et skib.}, så lå de på deres
kistebænke\footnote{Kistebænke er kister, man kunne sidde på.}, og så
gik de i land. Så sagde jeg til kaptajn Andersen: Ja, nok er jeg
mønstret her, men jeg er nu ikke mønstret til at være sammen med sådan
en besætning som den der, så jeg vil i land. Åhr, det kunne jeg altså
ikke komme. Ja, det vil jeg altså, for jeg vil ikke sejle sammen med
alle de væggelus. Noterne mellem plankerne de var helt røde af bare lus.
Men der var masser af skibe, der havde væggelus dengang, og man kunne
gerne se det ved at lyset brændte på de skibe om natten for at holde
lusene lidt i ave. Så kom jeg ikke med den.''
\end{quote}

\begin{quote}
``Kakerlakker det var jo nogle bæster. Når vi skulle sove så rendte de
jo og kneb os, men det var ikke farligt på nogen måde. Men se der i
sejlskibene, selv om der var kakerlakker og væggelus, så blev det jo
ikke røget ud som f.eks. i damperne. De skulle have bevis. Det måtte
ikke blive over 6 mdr. gammelt, og hvis der var utøj om bord, så skulle
skibet svovles hver 6. måned. Så blev vi jaget i land nogle timer.''
\end{quote}
