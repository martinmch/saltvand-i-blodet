\chapter{Sjov og alvor}\label{sjov-og-alvor}

\begin{quote}
``På frivagterne\footnote{Frivagter er fritid om bord.} om bord i søen
da sov vi jo den meste tid. Somme tider spillede vi kort. Noget skulle
tiden jo gå med.''
\end{quote}

Fritid til søs var der ikke meget af i sejlskibene. I gennemsnit havde
man 12 timers arbejde og 12 timers fri i døgnet. Med et almindeligt
søvnbehov på 7--8 timer var der kun 4--5 timer i døgnet til at spise,
klæde sig på, drikke te eller kaffe, personlig hygiejne samt
toiletbesøg. For det meste blev fritiden sovet væk --- der var ikke tid
og kræfter til andet.

\begin{quote}
''Frivagterne? Puh-ha. Der røg vi lige på hovedet i køjen og så sov vi
lige som en sten.''
\end{quote}

Søfolkene var på mange områder afskåret fra omverdenen. Der eksisterede
ingen form for radio, ingen nyhedsinformation, ingen underholdning ---
ud over hvad de selv kunne finde på. Når skibet var i søen, betød det
mindre. For det meste nåede man kun lige at stikke hovederne sammen
ganske kort, inden der skulle soves. Og tit var der kun tid til at læse
eller genlæse breve hjemmefra. Egentlig fritid om bord kunne man stort
set kun opleve på de store fuldriggede sejlskibe, der med
passat-vindene\footnote{Passatvinde er jævne vinde, der blæser hen over
  verdenshavene.} i ryggen krydsede de store oceaner.

\begin{quote}
``Som regel der i de danske skonnerter når de havde frivagt, og de havde
været oppe og havde puklet i den tid, så var de sgu gerne så udmattede,
så de sov altså fra det hele. Så var det en anden ting i de store
sejlskibe i passaten, hvor det var så fint vejr, der var mange der
brugte deres frivagt til at lave skibsmodeller --- helmodeller eller
halvmodeller og andre sådan forskellige ting. Ligesådan flaskeskibe blev
der lavet. Jeg kan huske flere gange jeg var på Charlie Brown, en meget
kendt beværtning i London, hvor alle verdens søfolk kom når de havde
mønstret af, og der var alting deroppe --- der var krokodiller og
assagaier og alt mellem himmel og jord og skibsmodeller, og der var det
så tit, når søfolkene ikke havde mere, så tog de sgu deres skibsmodel
under armen og gik op og sagde, hvad kan jeg få for den? Jah, du kan få
ti shilling\footnote{Shilling er en engelsk møntenhed.}, og så kan du
indløse den. Og så fik de ti shilling og drak dem op, og den blev aldrig
indløst, så de fik jo nogle souvenirs -- folks arbejde der.''
\end{quote}

\begin{quote}
``I skonnerterne var der ikke noget fritid. Men i de store skibe. Jeg
kan huske engang vi lavede sådan en kludebold. Dagen før havde der været
åbent i slopkisten\footnote{Slopkisten er kaptajnens lille private butik
  om bord.}, og så var der en af mine kammerater, en matros der havde
købt et par sko, og dem havde han på dagen efter, da vi spillede
fodbold. Vi løb rundt på stordækket og det havde jo ikke meget med
fodbold at gøre. Det var nærmest sådan tjatteri. Og så sparkede han
kraftigt, så denne her sko røg af og op i rigningen og
udenbords\footnote{Udenbords vil sige vandet, der omgiver skibet.}
regnede han med. Der røg din sko udenbords, var der nogen der råbte. Ad
helvede til, råbte han, så blev han så gal, så tog han den anden af og
smed ud. Men så var der en der sagde --- jamen her sidder jo din sko ---
den sad fast i rigningen. Så blev han endnu mere tosset.''
\end{quote}

Det hændte, at der blev spillet musik og sunget, men som regel ventede
man til skibet kom i havn. Det var i havnene, sømændene kunne slå sig
løs, mens skibet ventede fragter eller skulle repareres. Når skibet
nærmede sig en havn efter mange dage til søs, begyndte landgangsfeberen.
Snakken tog til om alle de ting, man skulle nå, inden sejladsen blev
genoptaget. Nogle brugte straks opholdet i havnen til at få et
velkomment kosttilskud.

\begin{quote}
``Når vi kom til en dansk havn, så røg vi op og købte en hel lagkage.
Den kostede to kr. --- en flødeskumslagkage. Så åd vi den. I England
købte vi plumbudding, og i Spanien købte vi engang nogle fine kager, og
der var noget i, og det var sgu klipfisk, så vi åd dem altså ikke. Vi
havde lige losset klipfisk og så få en kage med klipfisk!''
\end{quote}

Sømændene var afskåret fra et almindeligt seksualliv. I uger og måneder
lå de ude til søs uden at have kontakt med kvinder. De unge var afskåret
fra at søge den naturlige kontakt til jævnaldrene unge piger, og de
voksne mænd måtte undvære deres kæresters kærlighed. I det mandssamfund,
som skibene udgjorde, blev kvinder for nogle en besættelse, og besøg i
havn kunne ikke råde bod på den manglende kontakt med kvinder. Disse
havnebesøg blev for det meste et spørgsmål om hurtigt at tilfredsstille
nogle seksuelle drifter --- det vil sige, at man købte sig til et
samleje med en totalt fremmed og uvedkommende pige, for hvem det kun
gjaldt om at få det overstået. Allerede mens skibet var på vej ind,
havde man om bord besluttet sig for, om man skulle på bordel eller
horekasse, som man også kaldte det. En ung sømand, som tøvede med at
slutte sig til selskabet, kunne blive til grin blandt de andre.

\begin{quote}
``Jah, når søfolkene de gik i land --- hvis der var bordeller, så gik de
ind på dem og ellers på havnerestaurationer, hvor de kunne træffe piger.
Det var gerne de ældre som tog føringen. Vil I med derhen? Skal vi det i
aften? Og sådan. Nogle ville jo med og andre ville ikke. Der var ingen
tvang, men sådan en ung mand som ikke ville med på bordel kunne de godt
lide at drille lidt --- åh --- sådan en barnerøv osv. Så hændte det jo,
at han til sidst tænkte\ldots det skal de skisme ikke tro\ldots og gik
med.''
\end{quote}

Frygten for kønssygdomme afholdt dog nogle fra at gå med.

\begin{quote}
``Der var en fisker med oppe fra Hanstholm\footnote{Hanstholm er en
  fiskerby i Nordvestjylland.}. Vi kom til Malaga\footnote{Malaga er en
  havneby i Sydspanien.} og skulle losse{[}\^{}losse{]} fisken, og så
var han gået op til nogle piger, og så gik han og ragede sig en gonoré
til, og jeg så hvordan han skabte sig og jamrede sig, og han blev lagt i
land på et hospital og kom slet ikke med skibet, vi måtte sejle fra ham.
Da tænkte jeg ved mig selv: For det første kom jeg fra et godt
barndomshjem, og for det andet, tænkte jeg, du skal aldrig komme til
sådan en pige, for der er alligevel risiko, når jeg havde set ham,
hvordan han led og så blev sejlet agterud dernede i Malaga, så tænkte
jeg ved mig selv: nej, det skal du holde dig fra!''
\end{quote}

Nogle prioriterede i stedet det at komme ud at se noget nyt. Det kunne
være tyrefægtning i Spanien, en tur i teatret i London eller måske noget
helt tredje.

\begin{quote}
``Jeg kan huske, da vi lå i Norge, da var vi ude at se på Holmenkollen
--- nej, det med værtshuse, det har jeg ikke haft lyst til sådan.''
\end{quote}
