\chapter{Høj og lav}\label{huxf8j-og-lav}

\begin{quote}
``Der var mange gange, at skipperen eller styrmanden sagde til
drengene:''Nu ka' I bare vente jer til vi kommer i søen, så skal vi nok
sørge for jer!" Se det var jo en trussel, at så skulle de have tæsk, og
så var det vel årsagen mange gange til, at de simpelthen bare stak af
for at blive fri for skibet."
\end{quote}

Livet på sejlskibene var underlagt en lang række love og regler, skrevne
såvel som uskrevne. Mange af disse regler var nok nødvendige, mens nogle
i vores øjne kan virke barske og urimelige. Et sejlskib langt fra havn
var som et afsondret lille samfund, og i tilfælde af krise måtte der
ikke herske tvivl om, hvem der gjorde hvad. Derfor havde skipperen eller
kaptajnen altid det endelige ord, og man skulle altid adlyde ham. Det
var også skipperen, der var øverste myndighed, altså en slags dommer om
bord. Det var ham, som kunne straffe folk ved at trække dem i løn eller
sætte dem i land i den nærmeste havn. Det gav skipperen et stort ansvar,
men også mulighed for at tyrannisere sine folk, når skibet var på havet,
hvis det var det, han ville. Omvendt kunne han også belønne mandskabet
med ekstra fritid i havn, ekstra kost eller måske en lønforhøjelse.

Når man stod til søs med et sejlskib, måtte man blot håbe, at skipperen
var en rimelig mand, og at han og besætningen behandlede én godt. Men
det var ofte ikke tilfældet. Selv om det var ulovligt, blev fysisk
afstraffelse som tæv eller spark ofte brugt over for skibsdrengene, der
jo var nye i faget og lavede fejl en gang imellem. Man kunne ikke stille
meget op, hvis man kom om bord på et \emph{uheldigt skib}. Man var
ifølge loven forpligtet til at sejle med i minimum to år, og der var
ikke mulighed for at sige jobbet op. Stak man af fra skibet, når det kom
i havn, blev man kaldt for \emph{rømningsmand}. Og hvis man var det,
måtte myndighederne i land arrestere en, og så måtte man betale en ret
stor bøde eller i værste fald gå i fængsel.

\begin{quote}
``Jeg mønstrede om bord i en galease, der lå i Frihavnen i København og
skulle være bedstemand. Det var et lille skib -- indenrigs. Da jeg havde
været der i to dage, så pakkede jeg sgu min sæk og så gik jeg i land.
Jeg havde fået 15 kr. i forskud til en ny halmmadras. Så lagde jeg
resten af pengene på madrassen og til ham den anden dreng, der var om
bord, sagde jeg:''Nu ka' du give ham dem der og sige jeg er gået!" Så
var jeg jo rømningsmand. --- Men så kom der jo bud fra Sø- og
Handelsretten\footnote{Sø- og handelsretten er domstol for sager, der
  har med skibe at gøre.}, at jeg var rømningsmand og jeg skulle
straffes. Og så fik jeg en bøde på 12 kroner, og ham det frække bæst der
i Holmensgade\footnote{Holmensgade er en gade i København.} 1.
udskrivningskreds han mente jo, jeg var et forfærdeligt menneske, at jeg
sådan kunne rømme fra skibet. Han skrev ned i min søfartsbog, at jeg
rømte. Så tænkte jeg: ``Tak skal du ha', nu får du aldrig nogen hyre
mere!'' Nej, jeg var en storforbryder!".
\end{quote}

Om bord på skibet var folk ordet efter rang. Officererne var
selvfølgelig øverst, men mellem resten af mandskabet var der nogle helt
klare, uskrevne regler for, hvordan man skulle omgås hinanden. Der var
ingen plads til sarte følelser eller fine fornemmelser. Omgangstonen var
rå og kontant, men sammenholdet kunne være stærkt.

De ældste matroser havde mest at skulle have sagt. Den ældste fik den
bedste køje og det behageligste arbejde, men måtte til gengæld gå
forrest, når det virkelig gjaldt. Det kunne være vanskeligt arbejde i
riggen\footnote{Riggen er alt det, der sidder på masterne.} eller måske
som søfolkenes talsmand over for kaptajnen, hvis der var noget, de var
utilfredse med. Systemet fortsatte hele vejen ned til skibsdrengen eller
kokkedrengen, der ikke havde ret meget at sige. Til gengæld var det
almindeligt, at man skulle hjælpe hinanden med arbejdet og hverdagens
problemer. Den mere erfarne skulle være parat til at hjælpe den mindre
erfarne med arbejdet.

\begin{quote}
``Ja, det er sådan, altså når vi er i søen, at hvis der er et job vi
mener vi kan klare bedre end en anden, at vi ta'r det. Det er somme
tider, det har jeg da set, at de {[}skibsdrengene{]} har stået og rystet
for at skulle gå til vejrs, når sejlet rigtig har slået til. Så har jeg
sagt somme tider:''Bliv du bare nede, det skal jeg nok klare selv." Det
har jeg gjort flere gange. Og jeg ved en jeg sejlede med, senere han
blev skipper, da jeg kom til at sejle med ham, så sagde han: ``Jeg kan
sgu huske da jeg var en dreng, du var flink ved mig og lod mig blive
nede.'' Men jeg vidste jo, hvor farligt det var for nybegyndere at komme
op i sådan et sejl."
\end{quote}

Folkene holdt øje med hinanden i dårligt vejr og var parat til at give
en hånd, når det kneb. Det var en helt naturlig sag, og der var ingen,
der sagde tak for hjælpen --- man forventede, at alle ville gøre det
samme. En selvfølge var det, at man deltes om provianten\footnote{Proviant
  er maden om bord.}. Der var ingen, der kunne tage noget til side til
sig selv uden at være en dårlig kammerat. Derfor købte man heller aldrig
ekstra proviant med. Man hjalp også hinanden i fritiden, og at låne lidt
penge til en kammerat var der ikke noget i vejen for.

\begin{quote}
``Hvis der var noget man kunne hjælpe med gjorde man det. Jeg har da
f.eks. vasket tøj for en og syet og stoppet for en anden -- en svensker,
og jeg blev selv engang hjulpet med nogle rubler oppe i Skt.
Petersborg\footnote{Skt. Petersborg er en by i Rusland.}, dengang kunne
jeg ikke få noget hos skipperen. Der var en der gav mig lidt. Han tænkte
det var synd, at jeg ikke skulle have en ting med hjem til mine
forældre. Man kunne købe de her æg med mange små inden i og tobaksdåser.
Det var russisk lakarbejde.''
\end{quote}

Gensidig tillid var forudsætningen for, at man kunne hjælpe og stole på
hinanden som skibskammerater. Hvis nogen brød denne tillid, reagerede de
andre. Tyveri var groft brud på kammeratskabet --- ja, selv mistanke om
tyveri kunne ophidse gemytterne. At låse sin skibskiste\footnote{Skibskiste
  er en kiste til opbevaring af private ejendele som tøj mm.} var det
samme som at beskylde kammeraterne for at være tyvagtige. Den slags
tolererede man ikke.

\begin{quote}
``Man måtte ikke låse sin skibskiste. Det var det samme som at vi
mistænkte kammeraterne for at stjæle. Jeg havde en lås i kisten, og jeg
kunne ikke tage nøglen ud uden at jeg skulle låse den, og den måtte ikke
sidde i for de stødte benet imod.''Hvad skal jeg gøre?" spurgte jeg.
``Ja, du kan brække låsen af!'' sagde de. Det måtte jeg så gøre."
\end{quote}

Bonde var et skældsord i et sejlskib. Et bondeskib var et skib, hvor
tingene blev grebet forkert an, og hvor kammeratskabet ikke fungerede. I
de fleste skibe fungerede kammeratskabet godt. Det betød ikke, at folk
ikke kunne blive uvenner, men konflikterne blev løst efter reglerne. En
af reglerne kunne være slagsmål, helst overvåget af andre af
besætningen. Når kampen var slut, skulle de to kamphaner være gode
venner igen. Der var ikke plads til langvarigt had på et sejlskib, fordi
man var meget afhængige af hinanden.
