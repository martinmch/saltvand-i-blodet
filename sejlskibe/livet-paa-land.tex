\chapter{Livet på land}\label{livet-puxe5-land}

\begin{quote}
``Vi legede med små træskibe og planker om sommeren og vadede rundt og
skubbede til dem, hvis det ikke gik stærkt nok.''
\end{quote}

På havnen var der altid noget at se, når de store sejlskibe kom ind.
Hestevogne hentede de mange varer, som skibene havde med. Og søfolkene
fortalte historier fra fremmede lande. Havnen var spændende og satte
fantasien i gang. Ville man ud at opleve verden, måtte man sejle af
sted, for der var ingen andre muligheder. Sejlskibene satte derfor gang
i drømmen om den store verden. Voksede børnene op i en søfartsby eller i
en havneby\footnote{Havneby er en by med en større havn.}, var det ikke
usædvanligt, at de legede på havnen, selv om det ikke altid var
ufarligt.

\begin{quote}
``Undertiden legede vi på tømmerstablerne på Faaborg Trælast. Vi
klatrede op og sprang fra den ene til den anden. Når så
havnefogeden\footnote{Havnefoged er mand, der er ansat til at styre
  havnen.} kom, fik vi øretæver, hvis han fik fat i nogen. Ellers var vi
meget på havnen. Jeg måtte ikke løbe til havnen uden at få lov. Men jeg
løb sgu derned alligevel og så vankede der stryg, hvis far opdagede det.
Det var alle skibene i havnen der trak, og vi kom i snak med
skibsdrengene, der kunne fortælle historier og vise uartige billeder.
Der var ikke ret mange drenge, der holdt sig dernede fra.''
\end{quote}

\begin{quote}
``Vi drillede havnefogeden. Det var gamle Noack. Når havnefogeden var
efter os, så kravlede vi ned mellem pælene og ind på glaciset\footnote{Glacis
  er en stenvæg, der går skråt ud i vandet.}, så kunne han sgutte få fat
i os. Vi lavede mange numre, og så kom Noack hen til den gamle og sagde:
De satans knægte!''
\end{quote}

Den daglige færden på havnen gav mulighed for at hjælpe mændene på
skibene. Hvis man var rigtig heldig, kunne man få lov at kravle op i
skibenes master.

\begin{quote}
``Der kom mange 3-mastere til Bagenkop\footnote{Bagenkop er et
  fiskerleje på sydspidsen af Langeland.} med kul. Så ville vi jo gerne
op at hjælpe til, sejlene skulle tørres osv. Vi fik lov på den
betingelse, at vi skulle hente hver 10 spande vand om dagen og komme
vandfad på dækket. Hvis vi så hentede 10 spande vand om dagen, så måtte
vi kravle op i riggen.''
\end{quote}

\begin{quote}
``Vi drenge vi lå jo og kravlede i rigningen på de skibe, der lagde op
om vinteren. Vi skulle se hvem der kunne komme hurtigst op og lægge sig
på maven oppe på fløjknappen. Ja Gud fri mig vel \ldots{} at der ikke
skete en ulykke! Vi var som aber dengang. Det var noget alle drenge
gjorde.''
\end{quote}

Man fik et naturligt forhold til at sejle, hvis man voksede op ved
vandet.

\begin{quote}
''Vi var dårligt nok kommet ned på havnen, så lå vi nede i en jolle og
roede jo. Ja der var jo altid nogen større med, som havde forstand på
det, indtil vi andre havde fået sat det rigtigt sammen, så kom det. Det
var både med sejljoller og rojoller."
\end{quote}

Nogle blev indført i jollelivet af nødvendighed. Ikke alle steder var
der råd til at lade børnene bruge tid i skolen.

\begin{quote}
``Vi boede i et fiskerleje nord for Assens\footnote{Assens er en by på
  Fyn.} -- Aborg Strand. Min far havde temmelig gode kundskaber, og han
søgte skolevæsenet om, at vi ikke kunne slippe for at gå i skole, for vi
var nødt til at være hjemme for at hjælpe dem, imod at han underviste
os, og vi mødte hvert år til eksamen sammen med de andre børn og kunne
klare os. Når vi lå derude og slæbte ålevåd om natten, så lå vi og drev
sådan som omstændighederne var i tre kvarter til en time --- så sad vi
rigtignok nede i logaret og terpede, så stak den gamle lige hovedet op
af kappen en gang imellem for at se om der var noget usædvanligt. For
ellers så sad han og terpede med os dernede. Om det så var tysk, så
kunne vi det.''
\end{quote}

I søfartssamfundene lå det ligefrem i luften, at søen ventede forude.

\begin{quote}
``Selvfølgelig færdedes vi børn ved havnen, så snart der var lejlighed
til det, og tumlede os i joller, og dengang var der jo en hel del
sejlskibe hjemmehørende her i Ærøskøbing\footnote{Ærøskøbing er en by på
  Ærø.} og det var jo en begivenhed at komme ned at se, hvis de kom hjem
om efteråret og havde været ude siden marts måned. Så kom de gerne hjem
med en ladning kul eller ballastet\footnote{Ballast er sten eller sand,
  som man puttede i bunden af et tomt skib, så det ikke væltede.} hjem
for at lægge op, og så var vi jo ved jollen, når de var fortøjet og
roede omkring i den. Og selvfølgelig fik vi lyst til søen. I
Marstal\footnote{Marstal er en by på Ærø.} var det jo sådan, at hvis en
dreng ikke ville til søs, når han var konfirmeret, så var det jo ikke
nogen rigtig dreng. De få, der kom i håndværkslære eller sådan noget, de
var ikke rigtig anset, og der var noget af det samme her, men ikke så
udtalt.''
\end{quote}

Nogle piger syntes også, at havnen var et spændende sted. Men
omgivelserne var ikke altid klar til at lade piger løbe rundt på havnen.

\begin{quote}
``Jeg har været en ti til tolv år, da jeg var sluppet ned på havnen og
ivrigt iagttog en skonnert blive losset. Jeg havde hørt i skolen, at der
var nogle tyrkere med om bord, og dem ville jeg gerne se, hvordan de så
ud. De drejede det tunge lossespil med de tunge kurve på. Jeg stod bare
i cirka 15 meters afstand og kiggede på dette. Det har ikke varet
længere end et kvarter eller tyve minutter. Længere tid havde jeg ikke
til min rådighed. Jeg havde en gammel dame at gå byærinder for efter
skoletid. Om aftenen mødte en tante op. Oh, ve, nogen havde bedt hende
gå til min mor og fortælle, at jeg løb på havnen og var ude efter
fremmede søfolk. Jeg havde en fornuftig mor, som bad mig om at blive
derfra, så hun kunne blive fri for sådanne sladderhistorier, men jeg kom
aldrig på havnen alene mere.''
\end{quote}
