\chapter{Pas på!}\label{pas-puxe5}

``\textsc{PRIMO} forsvandt i Nordatlanten{[}\^{}nordatlanten{]} lige
efter, at jeg var gået fra den. Det var en 3-mastet skonnert. Jeg kom i
land på Københavns red\footnote{Københavns red er det sted uden for
  havnen, hvor skibene lå for anker.}. Så rejsen efter, da forsvandt
skibet. Ja, de fandt den senere, men de fandt aldrig besætningen.''

Når man satte sine ben i et sejlskib, udsatte man sig også for en vis
risiko. Man blev ganske vist rig på oplevelser, men kunne også i værste
fald miste livet. Farerne lurede. Sygdom, lemlæstelse eller død var
noget, de fleste søfolk stiftede bekendtskab med på et eller andet
tidspunkt i deres karriere. De hørte om skibe, der forsvandt, mødte
kolleger, som døde af feber, eller oplevede måske selv en ulykke, som de
mirakuløst overlevede. Det var ikke ualmindeligt, at gamle matroser
havde oplevet indtil flere forlis\footnote{Forlis er, når et skib sank
  eller på anden måde gik i stykker.}, som de med gru kunne fortælle om
nede i logarets mørke. Grumme historier var en uundgåelig del af
sølivet, og det var ikke uden grund, at søfartssamfund havde en større
andel af enker end andre steder. Et sejlskib var ganske enkelt en af de
farligste arbejdspladser, der fandtes. Det frygtede råb: ''Mand
overbord!'' betød i dårligt vejr for det meste et farvel til en
skibskammerat, der blev overladt til den ensomme druknedød. Voldsomme
oplevelser kunne føre til syner eller en stærk tro på større magter.
Troen på Gud kunne være stærk, men afhang også af det hjem, som sømanden
kom fra. Nogle var mere troende end andre.

\begin{quote}
``Jeg stod henne ved skipperen, han stod ved roret og havde
sejsing\footnote{Sejsing er tovværk, man bandt om sejlene.} omkring
livet, og så siger han,''hold dig ved``, siger han,''for der kommer en
grim sø\footnote{Sø er en eller flere bølger.}", og så kiggede han ud
til siden, men vi skulle kigge opad, da kom den rullende oppe over os,
lige som sådan en elektrisk pæl der er i højden, og så faldt den jo i
øvrigt, og jeg tog fat under ruffet\footnote{Ruffet er dækshuset, der
  benyttes til kahytsrum.} sådan og jeg har sørme taget sådan, at mine
negle de var gået ind i træet. Men der var ikke noget at gøre, jeg blev
væltet ud, og jeg havde som ærlig talt troet, at jeg var druknet, det
troede jeg. Alt var jo mørkt, når man kom under sådan en masse vand, men
det har jo været mit held mange gange, at jeg har været udsat for det og
er sluppet godt fra det.

Ja, jeg blev faktisk smidt lige på hovedet ind på dækket, og så var jeg
på vej ud igen, men så tog skipperen mig så. Og da kunne jeg høre, det
er altså sådan noget, man får i øjeblikket, jeg hørte en koncert af den
skønneste musik, som du kan tænke dig, og i min underbevidsthed, dèr var
jeg klar over, at jeg tænkte, jeg vidste ikke rigtig, hvor jeg var
henne, jeg var slet ikke klar over, om jeg i det hele taget levede, men
jeg hørte denne her musik, og så lige pludselig hører jeg nogen, der
siger: ``Lever du, er du vågen'', og så var det skipperen, der stod og
ruskede i mig. Og så vågnede jeg op, og så siger han ``Fejler du
noget''. ``Nej, jeg gør ikke'', siger jeg så, da var jeg ved at være
rask. Så siger han, ``Så gå hen og hjælp til med at hale det sejl ned''.

Og det var det sejl, at søen havde splintret alle trådene der. Der var
jo tusinde tråde og det var det, at stormen fløjtede i, og det var det,
der lavede musikken jo. Og så står jeg derhenne sammen med mine
kammerater, og jeg så med det samme, vi mangler en mand, men jeg kunne
ikke sige, hvad han hed. Det kunne jeg ikke.

Men så gik jeg hen og sagde til skipperen ``Vi har mistet en mand'', og
med det samme han sagde ``Det er Hans'', da var jeg klar over det. Nej,
der var ikke noget at gøre ved det. Ja, så sagde skipperen, jeg skulle
kravle et stykke op i rigningen og kigge, og skønt der lå planker og
brædder her og der, kunne jeg ikke undgå at se han lå og fægtede med
armene, men --- ak gud fader, der var ikke noget at
gøre\ldots (bevæget). Nej, ok nej, --- vi var selv så hjælpeløse, selv
om vi havde villet, ku' vi ikke ha' drejet den, eller fået den til at
køre rundt. Det ku' vi ikke. Så havde vi skullet bruge flere kilometer.
Det kunne vi ikke. Det sagde han jo også til skipperen, han bad jo en
bøn for ham, men\ldots"
\end{quote}

Blandt besætningerne var sygdom altid frygtet. Noget af det værst
tænkelige for en sømand var, hvis han blev sat i land på grund af sin
sygdom på et hospital og derefter agterudsejlet -- det vil sige, at
skibet sejlede sin vej med den raske besætning. Skete det i troperne,
var man nærmest prisgivet. Risiko for følgesygdomme kombineret med
mangelfuld pleje gjorde, at særlig mange døde på de kanter. Nogle gange
var sømanden også afhængig af kontante midler. Ingen penge -- ingen
behandling. Om bord var der ikke meget, man kunne stille op. Skipperen
og styrmanden havde ganske vist lidt lægelig viden med fra
navigationsskolen, men når det kom til stykket, kunne de ikke stille
meget op. Småskavanker kunne dog klares med lidt opfindsomhed.

\begin{quote}
''Joh, jeg havde tandpine. Hold kæft! Jeg kunne springe overbord. Ja,
jeg havde det hele vejen over Atlanterhavet, jo. Jeg havde ikke mere
forstand på tandpine end som der var honning i en skruptudse. Hvergang
jeg kom op på dækket så var der aldrig noget i vejen. Når man skulle ind
og sove og mente, nu har jeg det dejligt, så lige så snart jeg lagde
hovedet på hovedpuden, så havde jeg tandpine med det samme. Jeg havde
det sådan, at jeg var ligeglad med hvad jeg gjorde! Til sidst så glødede
jeg en sejlnål\footnote{Sejlnål er en nål til at sy sejlene med.}!
Kokken han sagde: ``Må jeg stikke den ned?'' Gu' må du ej, den vil jeg
sgu selv hav lov at stikke ned." Jeg fik lige fat i den --- han skulle
kigge mig i munden om det var rigtigt --- så jagede jeg til. Men jeg
tror nok, jeg sprang helt ``op over fokkeråen''! Hold da kæft hvor
gjorde det ondt! Så var den tandpine væk. Jeg havde brændt nerven over.

Men da havde jeg så også gået med det i over en måned. Og jeg stoppede
papir i ørerne, og jeg var lige ved at stoppe Social
Demokraten\footnote{Social Demokraten er en avis.} op i r\ldots Ja, hvad
jeg ikke gjorde altså, Alt muligt prøvede jeg! Jeg stoppede sølvpapir
ind i ørerne og stoppede det alt for langt ind, så jeg næsten ikke kunne
få det ud igen. Hver nat kl. 12, så kom det. Så havde skipper fået fat i
en passer\footnote{Passer er et instrument til at måle afstande på et
  søkort.} og havde lavet sådan et par kroge og så snuppede han tanden
til sidst. Men han var mindst 8 dage om at få fat i den, den blev
siddende."
\end{quote}
