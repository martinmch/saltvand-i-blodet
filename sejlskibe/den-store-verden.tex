\chapter{Den Store Verden}\label{den-store-verden}

\begin{quote}
``Jeg havde mange sjældne konkylier\footnote{Konkylie er en art
  sneglehus dannet af rovsnegl, der lever i vandet og spiser ådsler,
  mindre fisk og muslinger.} nede fra Vestindien\footnote{Vestindien er
  en øgruppe i det Caribiske Hav mellem Syd- og Nordamerika.}. Og så
havde jeg et af de her jernspyd der var med modhage og pyntet sådan som
et tyrefægterne bruger. Jeg havde da også en harpun. Men det hele
forsvandt. Mine brødre havde gået og solgt det for at få penge.''.
\end{quote}

De søfolk der fik mulighed for at komme med de større skibe, som sejlede
til andre verdensdele, mødte folk fra helt andre kulturer end den
nordeuropæiske, som de selv var en del af. Dengang havde man et noget
andet menneskesyn end i dag, og fordomme om andre kulturer var
almindeligt. Dog kunne kortvarige bekendtskaber opstå spontant
alligevel, og hvis der var tale om længerevarende havneophold kunne der
knyttes kontakt til de lokale. Lettest var det, hvis der var tale om
andre nordeuropæiske eller nordamerikanske havne. Kom et sejlskib til en
lille havneby, kunne besætningen være så heldig at blive inviteret med
til fest.

\begin{quote}
''Vi kom op til New Foundland i december måned. Det var helt sjovt at
komme op og losse kul på sådan en lille plads\footnote{Plads er en
  mindre havn.}. Der boede 300 mennesker der. Første dag, vi var kommet
ind, så var der fest. De havde sådan et forsamlingshus\footnote{Forsamlingshus
  er et hus, hvor mindre lokalsamfund kan mødes til foredrag, fester
  eller lignende.} (\ldots) og når kulskibet kom den ene gang om året,
så var der fest, og det vil sige, at vi jo skulle op og spise til aften,
og der kom så de lokale folk og så spillede de jo. Vi stak helt af fra
selskabet\footnote{Vi stak af fra selskabet skal her have betydningen:
  Vi lignede ikke de andre.}, for vi havde jo vores sko på og lignede
slet ikke de andre. Men det gik godt og det var sjovt.''
\end{quote}

Sværere kunne det være for de danske søfolk at begå sig, når de sejlede
til fjerne egne. De folk, som sømændene mødte, var ofte blandt de
fattigste. Det kunne være havnearbejdere, prostituerede eller måske
tiggere, der var henvist til et kummerligt liv i rendestenen. Sprog- og
kulturbarrierer gjorde kommunikationen overordentlig vanskelig.
Alligevel hændte det, at man fik fælles oplevelser med den lokale
befolkning på godt og ondt.

\begin{quote}
''This, Charles, Hjulemand og jeg omme i floden, som løber ud et lille
stykke herfra, og vaske tøj og hente vand. Fint vand at vaske i. Vandet
har vi nemt ved at få fat i. Vi vender bare båden, så den bliver næsten
fuld af vand, for vi skal jo også være der. Jeg ved ikke hvor mange mil
oppe i landet floden kommer fra, men hele vejen ud bader folk i den og
vasker tøj og smider affald ud i den. Der er ikke meget vand i den nu.
Når vi ligger på maven, stikker ballerne ovenfor. De sorte piger her har
ikke badedragt på, så de må ligge på maven hele tiden. Vi maver rundt
mellem hinanden i al ærbarhed. Det er meget fint drikkevand. Det kan
næsten gå ombord selv, så fyldt af alskens ting er det. Selv hajerne kan
lide det. De kommer helt ind på det grunde vand, småhajerne altså, for
at få en god mundfuld af det dejlige vand. Jeg kan gætte mig til, hvor
sprælsk det vil blive, når vi har sejlet rundt i varmen med det i nogle
uger. Når vi så kommer ombord, hiver vi det op med pøse og hælder det i
vandfadene. Vi har to store stående foran halvdækket agter\footnote{Halvdækket
  agter er et delvist overdækket område bagerst i skibet.} samt et stort
træfad surret på dækket samt en stor beholder nede i forskibet.

Traf negeren fra Cape Vincent, vor gode ven. Han gav os kokosnødder. Dem
er der nok af her i træerne. Fik dem i frøkapslen, men han viste os,
hvordan vi skulle få dem ud af den. Vi drikker en del kokosmælk, som
skulle være sundt. Det er i hvert fald rent. Om aftenen var alle mand i
land til fest og fik limonade, men vi gik ret tidligt om bord igen. Det
var en fin måneskinsaften, så Aksel og jeg startede igen. Nu ligger der
en del negerhytter placeret rundt omkring i sandet på vejen op til byen.
I en ret stor, som der ikke var sider i, var der dækket et stort bord
med forskelligt. Der skulle nok være fest, men der var ingen mennesker,
så Aksel og jeg satte os ned og ventede. Der kom ingen, så vi smagte
lidt på varerne, så gik vi, for Sveske blev bange for, at vi skulle få
tærsk, hvis de opdagede os. Om aftenen til fest på pladsen, ja det er
altså en stor plads med træer og buske. I midten et monument og en åben
plads med stole, som vi kan leje, men det opdagede vi for sent. Det var
en aften, alle stolene var optaget, så rejste et par sig vel for at
strække benene lidt, og straks snuppede vi stolene. De kom tilbage og
blev ved at kredse om os, men sagde ingenting, så vi blev siddende.
Senere fortalte Vincent os, at vi skulle give penge for stolene, så det
var en flov historie. Hvad mon pigerne tænkte om os. Inde på denne lille
åbne plads går alle de hvide, altså noblessen, som vi altså mener vi
hører til, idet vi også opholder sig der. Uden om går så alle de sorte,
eller dem som ikke er hvide nok, og alleryderst går så skravlet."
\end{quote}

Indtrykkene fra den store verden tog mange med sig hjem i form af
forskellige souvenirs, der kunne foræres til familien. De hjembragte
sager gjorde som regel altid lykke.

\begin{quote}
``Nede sydpå havde de alle de sydfrugter, appelsiner og vindruer. Det
var luksus dengang. Jeg købte en stor trækasse med figner. Den havde jeg
med hjem til mine brødre og min søster. De kendte jo ikke andet end de
tørre. Den gjorde lykke den kasse der. Min søster var 1 år ældre end
mig, og hun kunne altid bruge parfume og sådan noget. Det var spændende
med udenlandske ting, især fra Frankrig. Der var ikke så meget ved ting
fra England.''
\end{quote}
