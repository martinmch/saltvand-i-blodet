\chapter{Far-Vel}\label{far-vel}

\begin{quote}
``Jeg var lige konfirmeret og så skulle jeg ud med den gamle Petersen,
det var vores nabo. Han boede ved siden af far og mor, og det var vist
aftalt længe før jeg blev konfirmeret.''
\end{quote}

I sejlskibstiden\footnote{Sejlskibstid er dengang, sejlskibene sejlede.}
havde man ikke mulighed for at vælge mellem uddannelser og beskæftigelse
som i dag. Drengene kom meget ofte til at lave det samme som faderen,
mens pigerne skulle lære at holde hus ved at tjene på gårdene eller i
større husholdninger. Her skulle de være, til de var
giftemodne\footnote{Giftemoden er, når man er gammel nok til at blive
  gift.}. Konfirmationen var et afgørende skel. Det omgivende samfund
anså herefter én for at være klar til arbejdslivet. Når man blev
konfirmeret, gik der ikke længe, før leg blev til alvor. Familierne var
børnerige og havde som regel ikke ret mange penge. Når barnet blev sendt
af sted, var der en mund mindre at mætte. Hvis man var dreng i et
søfartssamfund, eller hvis ens familie var beskæftiget i
søfartserhvervet\footnote{Søfartserhverv er de jobs, som har forbindelse
  med skibe.}, så var der gode chancer for, at man selv skulle til søs.
Tradition spillede en stor rolle, og mange gange var det allerede aftalt
på forhånd, hvilket skib den unge skulle sejle med. Nogle gange kunne
det gå hurtigt, og før konfirmanden vidste af det, kunne han være på vej
til skibet.

\begin{quote}
``Konfirmationsfesten skulle være om onsdagen, da vi skulle til alters,
og da var jo lagt stort i kakkelovnen\footnote{''Lagt i kakkelovnen'' er
  et udtryk for optræk til ballade}. De skulle jo være der alle sammen,
når vi kom hjem fra den kirkehistorie der, og skulle spise, og det gik
fint med det, og så der klokken fem om eftermiddagen, da kom der sgu et
telegram at jeg skulle rejse klokken fem. Jeg havde fået hyre\footnote{Hyre
  betyder løn.}.''
\end{quote}

Arbejdsaftaler med skipperen eller ejeren af skibet var tit indgået på
forhånd, men før man kunne komme om bord, skulle man lige ses an.

\begin{quote}
``Jeg gik med min bedstemor derned, og hun spurgte, om han havde brug
for en dreng. Ja, det har vi, sagde han, er det ham der? Ja, det var det
jo. Ja vi kan godt prøve at tage ham med. Og så skulle jeg have ti kr.
om måneden, og hvis jeg så duede til noget skulle jeg have femten kr.''
\end{quote}

I søfartssamfund blev konfirmationen afholdt tidligere, så drengene
kunne nå at sejle ud med skibene, lige så snart isen begyndte at smelte
i slutningen af marts eller i begyndelsen af april:

\begin{quote}
''Vi havde altid konfirmation fjorten dage før andre havde. Så sejlede
vi til Marstal\footnote{Marstal er søfartsby på øen Ærø, der ligger syd
  for Fyn.}. De skibe der havde været hjemme og overvintre og blev
repareret om vinteren, de skulle gerne ud senest i april og så skulle
konfirmationen være overstået først. Det var hyrebasser\footnote{Hyrebasser
  er folk, der ledte efter mænd og drenge, som kunne få job på skibene.},
det kaldte vi dem dengang, altså privatfolk, der forhyrede\footnote{Forhyrede
  er ansatte.} søfolk. De havde gerne sådan en lille
søudrustningsforretning ved siden af med olietøj og søstøvler og
forskellige knive, hvad en sømand skulle have, merlespir og sejlhandsker
og sådan.''
\end{quote}

Udrustningen blev der mange gange sørget for hjemmefra. Køjesækken og
skibskisten var vigtige. Skibskisten blev gerne pakket af moderen. Og
konfirmanden kunne regne med at få en god og fornuftig udrustning med
sig med tøj til alt slags vejr, hvis familien var vant til at beskæftige
sig med arbejdet på søen. Selve mønstringen om bord foregik på den måde,
at den nye dreng fik anvist en køje forude --- og så blev han straks
kastet ud i arbejdet. Allerede på det tidspunkt gik sølivets barske
realiteter op for den unge, og nogle af barndommens drømme er nok
hurtigt fordampet op i den blå luft.
