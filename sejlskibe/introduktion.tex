\part{Sejlskibe}\label{sejlskibe}

Sejlskibe udgjorde rygraden af den danske handelsflåde helt frem til
omkring 1900. Man brugte oftest søvejen, når varer skulle fra et sted
til et andet. Det var mange gange nemmere at sejle end at benytte
bumpede og mangelfulde landeveje, og så kunne et skib flytte langt mere
gods end en hestevogn. Det blev på den måde billigere at transportere
varer og mennesker over havet i stedet for over land. De fleste byer lå
da også ved en kyst med nem adgang til søtransport.

I 1870 var industrialiseringen så småt begyndt herhjemme. Jernbanenettet
blev udbygget, fabrikker blev opført, og befolkningen voksede meget
hurtigt på grund af bedre levevilkår. Der blev derfor et stadigt større
behov for at transportere varer rundt mellem byerne og mellem
landdistrikter og havne. Til at dække behovet søsatte man stadigt flere
skibe, og selv om dampskibet var opfundet, byggede man stadigvæk mange
sejlskibe. De var længe billigere både at bygge og at sejle med.
Ulemperne var dog flere. Sejlskibene var afhængige af vind og vejr.
Blæste der ingen vind, kunne skibet ikke sejle. Blæste der for meget
vind, kunne skibet være svært at styre. Og det risikerede at gå på grund
eller måske at blive knust mod en klippeside.

Sejlskibe var heller ikke helt så punktlige, og så kunne det tage lang
tid at laste\footnote{Laste er, når avrer eller gods tages om bord på et
  skib.} og losse\footnote{Losse er, når varer eller gods tages fra
  borde på et skib.} skibet. Det kunne også være temmelig risikabelt at
sejle med disse sejlskibe. Mange sømænd forsvandt sporløst, hvis et skib
for eksempel ramte et isbjerg eller kæntrede, når lasten forskubbede sig
i hård søgang. Alligevel kunne det betale sig at søsætte nye sejlskibe.
Hver gang skipperen eller kaptajnen afleverede en ladning sikkert i
havn, fik skibets ejere oftest et fint overskud, selvom priserne
naturligvis svingede, og nogle år var bedre end andre. Mange redere
brugte overskuddet til at bygge nye skibe. På den måde tjente de flere
penge --- og så kunne de købe endnu flere skibe. Nogle områder og byer i
Danmark specialiserede sig i sejlskibssøfarten. Den var ganske vist
risikabel, men gav også mange mennesker brød på bordet hver dag. Et af
disse områder lå i Det sydfynske Øhav.
