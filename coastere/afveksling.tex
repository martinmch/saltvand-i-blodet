\chapter{Afveksling}\label{afveksling}

Fritiden i forbindelse med søfart kan deles op, for der er jo både tale
om fritid, når man er til søs, og når man er i land. 

Dæksbesætningen på en coaster har fri efter \wordd{udskejning}{Udskejning
betyder fyraften -- at man har fri.} om eftermiddagen og i weekenderne -
fra fredag eftermiddag til mandag morgen - med mindre, der er
overtidsarbejde med vedligehold af skibet. Hvor langt omkring coasteren
sejler er meget forskelligt - nogle coastere sejlede tidligere i
\wordd{nærfarten}{Nærfarten er sejlads i nære farvande i Danmark og
områderne i nærheden af Danmark.} omkring Danmark og var derfor tit i
havn, måske hver weekend. Sejlads i de nære farvande forekommer fortsat i
begrænset omfang, eksempelvis i farten på jernværket i Frederiksværk.
Andre coastere sejlede og sejler på langfart, er i fart world wide, hvor
en måned eller mere i søen nemt kan forekomme.

Fritidssysler er derfor en nødvendighed, for at tiden ikke skal falde lang
på de længere rejser.  Hvad kan man så foretage sig? Man kan læse bøger
eller se film. Skibene er forsynet med et skibsbibliotek, og der er mange
videofilm, fordelt af Handelsflådens Velfærdsråd. Desuden er der
forskellige spil om bord, og søfolkene har selv elektroniske spil og
musik-CD'er med: 

\begin{quote}
    ``Ved afgang Havanna stævner vi lige ind i en weekend.  Lørdag/søndag
    er fridage for dæksbesætningen, med mindre, der er nogle
    ekstraordinære ting, der skal tjekkes og klares. Et par stykker af
    dæksbesætningen spiller Backgammon, nogle læser bøger og tidsskrifter
    og nyhederne hjemmefra.

    På mandage, onsdage og fredage bliver der fra Handelsflådens
    Velfærdsråd udsendt `\wordd{Navnyt}{``Navnyt'' er titlen på den
    elektroniske nyhedsavis, de elektroniske nyheder, der sendes ud til
    skibene.}', så at vi kan følge med i de vigtigste nyheder. På et
    aftalt tidspunkt bliver der vist videofilm.  Skibsbiblioteket bliver
    gennemgået, og der er gang i vaskemaskinen.  Lørdag aften sidder en
    stor del af besætningen på agterdækket i det smukke vejr og snakker og
    ser på de store krydstogtskibe, der sejler ud fra Miami med kurs mod
    Bahamas eller blot med sejlads ude i Florida Strædet\ldots{}
    
    Så var vi på havet igen efter 11 dage i land, i det venlige England.
    6-mandsludoen bliver omsider spillet færdig\ldots{} Under passage af
    De Kanariske øer får vi ludo-pandekager om aftenen, bagt af Anders og
    Jan, der tabte ludospillet, vi to vindere slapper af medens de
    midterste står for opvasken.'' 
\end{quote}

Naturen kan nu også levere underholdning: 

\begin{quote}
    ``Vi har dagligt besøg ved skibet. Vi ser spækhuggere, hvaler, og en
    aften en flok på flere hundrede delfiner rundt om skibet.  Der er
    såmænd mange oplevelser, vi er på dækket alle mand for at se
    sceneriet, også de mere garvede kommer med begejstrede udbrud.''
\end{quote}

Man kan også afholde forskellige festligheder om bord, f. eks. ved passage
af Ækvator: 

\begin{quote}
    ``Vi passerer Linien kl. halv fire eftermiddag og afholder Ækvatordåb.
    Tre af besætningen på ANNE BOYE har ikke været over Ækvator før, og
    fire af den øvrige besætning har glemt deres dåbsattest, så Kong
    Neptun forlanger også disse formastelige gendøbt og dermed godkendt
    til videre sejlads på De Syv Have. Oven på afvaskningen af
    landkrabbestøvet, barberingen og indtagelse af den ordinerede medicin
    er der vist ingen, der glemmer attesten en anden gang. Om aftenen er
    der spisning på brodækket, hvorunder skipperen uddeler dåbsattesterne.
    Dåben er en velkommen afbrydelse i den almindelige hverdag på en lang
    rejse.''
\end{quote}

`En sømand ser aldrig noget oppe i land nu til dags', hører man tit folk
påstå. De store containerskibe ligger oftest kun inde ved en
fjerntliggende containerterminal i få timer for losning og lastning,
således at søfolkene ikke kan komme i land og se på omgivelserne. Sådan
forholder det sig imidlertid ikke i den såkaldte mindre skibsfart. I
coasterfarten kommer man ind til de ''rigtige'' havne og ligger ofte flere
dage under land. Søfolkene kan så gå op efter udskejning og måske slå til
søren, hvad der godt kan være behov for efter måske en måned i søen: 

\begin{quote}
    ``Mange mennesker tror at vi søfolk har en pige i hver
    havn. Nu er det sådan, at vi ikke kan nå at komme rigtigt op i land i
    alle havne på grund af losning og lastning. Men specielt i
    sydamerikanske og caribiske havne, der ligger vi som regel nogle dage.
    Efter udskejning, fyraften, så går vi op i land om aftenen og får en øl,
    og går de steder hen, hvor der er søde piger. Og så hænder det vi får
    lidt på den spidse, ellers kan vi gå hen og blive lidt underlige på en 6
    måneders udmønstring.''
\end{quote}

\begin{quote}
    ``Efter udskejning, fyraften, går vi op i byen og drikker en øl og spiser
    en is, går på listige steder, hvor der er damer.''
\end{quote}

Sømandssange om bord er historie nu til dags og i coastertiden
som sådan. På en lørdag eftermiddags grillfest om bord bliver guitaren
hentet frem efter akkurat samme mønster som i land.
