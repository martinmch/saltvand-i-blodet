\chapter{Hips om haps}\label{hips-om-haps}

Til søs er der et bestemt besætningsmønster, en rangorden med skibsføreren
som den øverste, enerådende person på rangstigen.

Det kan måske virke lidt underligt på unge i vor tid, men et skibs
overlevelse afhænger i sin yderste konsekvens af handlekraft. Disse
forhold har altid været gældende til søs, og sådan er det fortsat.
Samtidigt må det understreges, at søfart altid har været den mest
demokratiske form for erhverv: Alle --- uanset hvor de kommer fra ---
startede/starter på samme niveau. Starten er stillingen som dæksdreng (nu:
ubefaren skibsassistent), uanset om faderen er baron eller matros. Alle
har lige mulighed for at stige i graderne og ende som skibsfører, hvis
ellers evnerne og viljen er der, naturligvis. 

Arbejdsopgaverne for en skibsassistent i en coaster har på ét punkt ændret
sig markant fra coasteralderens start og til i dag. Da
selvstyreren/autopiloten blev almindelig i skibene i første halvdel af
1960'erne, skulle dæksfolkene ikke længere have \wordd{rortørn}{Rortørn
vil sige at stå til rors, altså at styre skibet.}, hvor vagten
\wordd{tørnede om}{Tørne om betyder at skiftes til.} at stå til rors
døgnet rundt. 

Stillingsbetegnelserne om bord er ændret. Indtil besætningsloven fra 1985
var dæksfolkenes betegnelse: Dæksdreng, jungmand, letmatros og matros,
hvor man som regel havde ét års sejltid i hver stilling. Disse betegnelser
er afløst af betegnelserne `ubefaren skibsassistent' for de tre første
stillingers vedkommende, og befaren skibsassistent som betegnelse for
matrosen, der er den fuldbefarne sømand, altså den fuldt udlærte.
Skibsassistenterne deltager i ethvert job om bord, der omfatter
vedligehold, skibsarbejde, rengøring, klargøring til ankomst/afgang og
losning/lastning. De skal desuden hjælpe til i maskinen samt sørge for at
gange og messer samt eget kammer er i orden. Hvis kokken har
\wordd{forfald}{Forfald vil sige at udeblive fra arbejdet på grund af
sygdom.}, kan skibsassistenten udføre kokkens tjeneste. En erfaren matros
er ofte bådsmand om bord. Han er dæksfolkenes boss: 

\begin{quote}
    ``Anders er bådsmand om bord, han går op på broen hver morgen, når der
    er arbejdsdag før kl. 7 og snakker med skipperen om dagens arbejde.
    Kl. 7 går han ned og udstikker jobbene. Som bådsmand kan Anders sine
    ting, han er en dygtig sømand og en god kammerat, der ser til at ingen
    går alene og kører sur i det på frivagten eller i weekenderne. Han
    kører tingene på dækket problemfrit. De øvrige dæksfolk respekterer
    ham som sømand og som kammerat.''
\end{quote}


I et interview med bådsmand, ''Bådsen'', Anders Trustrup på dækket den
11.9.2002 på Atlanten på rejse Havanna-Cadiz, siger han:

``Jeg hedder
Anders, er 33 år. Jeg har boet på Grønland, i Spanien og Danmark. Nu er
det mest på Haiti, når jeg har fri. Jeg har sejlet siden 1987, da fik
jeg min første hyre. Jeg kom på søfartsskole 1988 og har så sejlet
siden, på nær en 4-årig periode, hvor jeg havde forskelligt arbejde i
land. Det at være coastersømand er et virkelig frit liv. Mange mennesker
tror, at vi er bundet meget fast til skibet. Det er vi selvfølgelig
også, når vi sejler, men når vi sådan kommer rundt i verden - vi kommer
mange spændende steder - så er det meget frit. Når vi har fri fra
arbejde kommer vi rundt og ser på de forskellige steder, vi er havnet i.
Selve arbejdet om bord er meget afvekslende, der er skibsarbejde og
rengøring, vedligeholdelse såsom rustbanken, maling, smøring af
bevægelige dele.''

I et interview den 3.9. 2002 på Atlanten på rejse
Havanna-Cadiz med dæksdrengen Jack, kaldet ''Dæks'' (forkortelse for
betegnelsen dæksdreng, der stadigvæk anvendes om bord), siger han: 

\begin{quote}
    ``Jeg hedder Jack Henriksen, jeg kommer fra Lolland, og bor i Nykøbing
    Falster nu. Jeg er 17 år og har været ude at sejle i 5 en halv måned.
    Jeg har måske ofte de lidt sure tjanser, men jeg er jo også yngste
    mand om bord.  Generelt så er der \wordd{bakstørn}{Bakstørn er opvask
    og klargøring i kabyssen.}, rengøring hver morgen fra 7 til 8, så
    bagefter hjælper jeg de andre ude på dækket med at banke rust, slibe
    og male, tjekke surringer afhængig af, hvad last vi har, og så smøre,
    hvad der skal smøres på skibet og hvad der ellers er. Vi vasker ned og
    skurer skibet, så det er flot, når vi kommer i havn. Vi \wordd{fersker
    ned}{Ferske ned vil sige at man spuler - eller vasker ned - med
    ferskvand for at rense skibet for de saltpartikler, som saltvandet
    efterlader. Det er normalt brobygningen der ferskes af.}, når der er
    meget salt på skibet. 
    
    Når vi skal i havn, så en halv time - tre kvarter før gør vi klar til
    at modtage lodsen. Vi rigger lods-\wordd{lejderen}{Lejder er en
    trappe.} an og jeg viser ham vej op på broen til skipperen. Så gør vi
    klar, lægger \wordd{trosserne}{Trosse er et reb til at fortøje et skib
    med.} parat og hvad vi ellers skal bruge. Når vi går til kaj er jeg
    ude på bakken, ude foran på skibet, hvor jeg gør kastelinen fast til
    trossen og kaster linen i land og er behjælpelig med at totte
    trosserne op, så at skibet ligger tæt ind til kajen. Når vi er
    fortøjet og har klaret op, gør vi lugerne klar til at åbne for losning
    eller lastning.
    
    Livet til søs er dejligt. I havn skal vi være klar til at dække
    lugerne over eller luge af. Når det er regnvejr er vi nødt til at
    lukke lugerne.  Det er hårdt arbejde. Vi skal kommunikere med
    havnearbejderne, det kan der sommetider komme nogle sjove samtaler ud
    af. Efter udskejning, fyraften, går vi op i byen og drikker en øl og
    spiser en is, går på listige steder, hvor der er damer. 
    
    Det er dejligt at være ude på søen, vi er et lille samfund. Vi er et
    lille samfund på 8 personer på sådan en coaster som ANNE BOYE. Vi
    sejler med alt muligt blandet last.'' 
\end{quote}


Om bord er der i en lidt større coaster --- ud over skibsføreren og
styrmanden --- også en maskinmester og en kok. I mindre coastere klarer
styrmanden maskineriet, og en af dæksfolkene klarer undertiden
kabystjansen. \wordd{Udmønstring}{Udmønstring er den periode, hvor man er ansat
(mønstret) om bord på et skib.} i en coaster af i dag er typisk 3 måneder for
officererne og 6 måneder for det menige mandskab. De to navigatører
skiftes til at have vagten på broen, hvor skipperen har vagt kl. 6 -- 12
og 18 -- 24, og styrmanden har de to øvrige. For at blive styrmand og
skipper skal man have navigationsskoleuddannelse.
