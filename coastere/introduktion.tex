\part{Coastere}\label{coastere}

En coaster er et mindre tørlastskib115, der kan sejle gods fra havn til
havn. En coaster er et skib, der kan medtage en godsmængde på mellem 500
og 4000 tons. Søfart er i sagens natur et internationalt erhverv, hvor
det engelske sprog spiller en stor rolle. Derfor er mange engelske ord
gledet naturligt ind i dansk søfart længe før denne tendens begyndte at
gribe om sig generelt. Ordet ''coaster'' kommer således også fra
engelsk, hvor coasterbegrebet oprindeligt anvendtes om skibe i
kystfart116. I dansk søfart dækker coasterbegrebet et mindre skib, der
både sejler i de kystnære farvande og sejler i fart over længere
afstande -- for manges vedkommende på alle have -- world wide, som det
hedder inden for søfarten. Den type skibe, der nu til dags kaldes
coastere, blev kaldt for motorskibe fra starten, hvilket vil sige årene
efter Anden Verdenkrigs afslutning, altså i 1950'erne.
Coasterbetegnelsen slog først rigtigt igennem omkring 1960. Den gang
dækkede begrebet skibe, der kunne medtage fra ca. 200 tons og op til ca.
800 tons. Men som alt andet i denne verden voksede størrelsen igennem
årene, grænsen er flydende med opadgående tendens. Danske coastere var
talrige før i tiden, ikke mindst i 1970'erne. Dengang gik størrelsen op
til en 12-1400 tons. Man kunne møde dem overalt. Skibe med en lasteevne
på under 7- 800 tons, der tidligere var særdeles talrige i den danske
handelsflåde, er imidlertid i dag stort set sejlet helt ud af dansk
søfart. Ligeledes tynder det fortsat ud i rækkerne af coastere i den
næste kategori, der har en lasteevne til op omkring 2500 tons.
Coasterne/motorskibene sejlede i epokens begyndelse mest i fart på
europæiske og nordatlantiske havne, og for enkelte gik turen også ud på
den virkelige langfart. En stor del var beskæftiget med at transportere
alle mulige varer i nordeuropæiske farvande - også i danske. Men i
nærfarten transporterede lidt mindre skibe den største varemængde inden
for det, der kaldes småskibsfarten117, hvilket vil sige skibe under
coasterstørrelsen. Småskibsfartens skibe er nu væk, idet jernbaner og
lastbiler har udkonkurreret godstransport i nærfarten. En årsag til
dette er et omfattende byggeri af broer og veje samt en afgift på den
ellers miljøvenlige søtransport. Coastere besejler fortsat en hel del
danske havne, hvor de kommer for at losse eller laste gods eller for at
blive repareret. Mange havneafsnit er imidlertid i dag afspærret med
hegn, en såkaldt terror-sikring. Men der findes dog fortsat havne, hvor
man kan gå helt ned til det virkelige havneliv og klappe en coaster
eller et andet skib på skibssiden.

Tørlastskib er et skib, der sejler med gods, der ikke er flydende, i
modsætning til tankskibe. Kystfart vil sige skibe, der sejler i nærheden
af land, og således ikke kommer ud på lange rejser over store åbne have.
117 Småskibsfart er et udtryk for små skibe, der kun kan medbringe en
mindre godsmængde - op til ca. 200 tons - og sejler i kystfart. 116
