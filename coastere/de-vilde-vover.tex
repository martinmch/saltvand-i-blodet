\chapter{De vilde vover}\label{de-vilde-vover}

Et skib er en speciel arbejdsplads, hvor man ikke skal gå at halvsove.
Dels er der mange specielle adgangsveje som snævre mandehuller145 og
stejle lejdere, dels bevæger skibet sig bl.a. under indflydelse af vind
og sø, så for at undgå arbejdsulykker skal man være årvågen. I vore dage
er der desuden flere piratoverfald, hvilket også går ud over danske
coastere. Sikkerheden om bord i nutidens skibe er imidlertid i højsædet.
De enkelte landes myndigheder tjekker løbende skibene ved besøg under
havneanløb (Port State Control) for at se, om alt er i orden. Om bord
indøver man procedurer, således at man er forberedt på forskellige
situationer som eksempelvis piratbesøg: ''Onsdag d. 27. november (2002),
kort efter Santos, da vi er ved middagsskafningen, kommer der opkald fra
styrmanden Allan: ''Skipper til broen!'', samtidig bliver roret lagt
hårdt bagbord over og kort efter bliver maskinen slået bak. Vi farer
småt gumlende ud på dækket og observerer fluks et fartøj, der for fuld
fart stævner hen imod os med halvnøgne folk, der viftede oprørt med
armene. Hvad var det monstro for noget det her? Vores maskine går
naturligvis i stå under manøvrerne, så at vi ikke kan komme af vejen i
en fart. Vi har kort forinden fra en landstation modtaget advarsler om
piratoverfald her på kysten, og har ydermere læst i de nys modtagne
aviser om overfaldet på Marstalcoasteren KIM i Cayenne. Og - ja vi tror
faktisk lige i øjeblikket, at der var tale om piratoverfald, og vi
begynder at lukke og skalke døre146, som tidligere afprøvet under
''Pirat Rulle''. Det viser sig imidlertid snart, at være fredelige og
venlige fiskere, hvis garn vi er sejlet ind i. Garnet bliver klaret fri,
fiskerne får nogle øl af skipperen, og vi får 2 guldmakreller, hvoraf
kokken tilbereder nogle velsmagende stykker til aftensmaden. Efter at
have vinket passende farvel til ''sørøverne'' steamer vi af sted igen
klar og modtagelige for nye indtryk.

145

Snævre mandehuller er adgangsveje til bundtanke og andre hulrum i
skibet, hvor der ikke er døre eller luger. Mandehullerne er små runde
huller, som en mand lige kan klemme sig igennem. De er normalt lukket
med en lem, der er boltet fast. 146 Skalke døre er at lukke lugerne
vandtæt.

39

Det tager nogen tid at sejle de 56 sømil op ad floden Demerara
(Guyana), hvor vi skal losse en del af lasten ved en plads, der hedder
Christiansburg ved byen Linden. Der er kommet en hel del personer om
bord på reden ved Georgetown og 11 af dem bliver på skibet på den videre
færd, 8 af disse er svært bevæbnede politisoldater - rart nok, hvis vi
skulle møde flodpirater eller andre skarnsfolk. Det er flodsejlads i
brunt vand, ganske spændende. Vi når imidlertid godt frem til
Christiansburg, og soldaterne flytter ind i nogle skure ved
anløbsbroen''. Om bord på et skib laver man med passende mellemrum
forskellige øvelser, således at man ved hvad man skal gøre, hvis ulykken
skulle ske. Ud over den ovenfor nævnte Pirat-rulle, altså øvelse i
forholdsregler, hvis der skulle ske piratoverfald, er der Brand-rulle og
bådøvelser: ''Onsdag bliver der sendt alarmsignal fra broen:
''Bådrulle''. Vi rubber os hvad vi kan, og møder op, hvor vi har besked
på at møde med redningsdragter, som vi ifører os. Efter øvelsen bliver
der snakket om tingene. Der bliver jævnligt holdt øvelser om bord også
inden for ''Brandrulle'' og ''Mand Over Bord Rulle'', således at man får
indøvet procedurerne under øvelserne.''. Coasterfart er et sikkert
erhverv i dag. Alligevel kan der forekomme søulykker og forlis. M/S ERIK
BOYE blev den 6.11. 1979 sejlet i sænk af den 24.000 tons store
kinesiskejede GOLDEN MIRANDA i Middelhavet. Fyns Amts Avis skrev den
10.11. herom: ''Besætningsmedlemmerne fra det forliste Marstalskib
''ERIK BOYE'' kom i går til Marstal. Føreren Harry Hansen havde sin
hustru Lonny og ægteparrets femårige søn Per med på den dramatiske tur.
Hele familien befandt sig, siger Lonny Hansen, i salonen, da ''ERIK
BOYE'' blev påsejlet næsten midtskibs, og ikke mange meter fra, hvor vi
befandt os. Det er vanskeligt bagefter at huske, hvad man egentlig
tænker i et sådant øjeblik. Der flyver så mange tanker igennem ens
hoved. Min mand tog straks Per under armen og løb ud af salonen og råbte
til mig, at jeg skulle følge med. Det er nok ikke ret meget, man
egentlig sanser i sådanne øjeblikke, men som en refleksbevægelse rakte
jeg ud over bordet for at tage mine briller. Det var det eneste, vi fik
reddet, ellers intet ud over det tøj, vi gik og stod i. Der var hurtigt
vand i gangene ved salonen. Vi kom hurtigt op på broen ved styrehuset og
alle mand fik redningsveste på. Per havde sin på i forvejen -- det havde
han hele tiden om bord. Lonny Hansen roser besætningen for den omsorg,
de viste sønnen i disse hektiske og farlige minutter. De råbte til mig,
at jeg blot skulle sørge for mig selv, så skulle de nok tage sig af
drengen, og det gjorde de på en enestående måde''. Søfolkene fik
redningsbåden sat i vandet, men det hele gik så hurtigt, at fem af dem
måtte springe i vandet og kom derfra op i båden. ERIK BOYE's besætning
blev samlet op af det kinesiske skib efter halvanden times forløb i det
tågede vejr, og de søfolk, der havde været en tur i vandet, fik tørt tøj
på. De bjærgede søfolk blev efter fem timer landsat i Algier.

40
