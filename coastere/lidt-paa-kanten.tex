\chapter{Lidt på kanten}\label{lidt-puxe5-kanten}

Søfolk møder i sagens natur folk fra andre nationer. Om man blander sig
med folk, eller man holder sig langt fra de lokale, fremmede folk,
afhænger af en selv og de påvirkninger, man har modtaget i opvæksten og
de første udmønstringer. Nogle får en racistisk holdning på grundlag af
ubehagelige oplevelser, andre er åbne og modtagelige for nye indtryk, og
den kategori af søfolkene oplever en masse.

Nogle søfolks landgang strækker sig til de nærmeste værtshuse, mens
andre får set en del af byen og landskabet: ''Skibets 3 J'er: Jack, Jan
og Jens har naturligvis været på opdagelse oppe i land, de er altid åbne
for nye oplevelser og kontakter -- for det meste har de stiftet
bekendtskab med en lokal pige, og det er nok ikke det værste man kan
blive udsat for. De unge søfolk kan vi absolut godt være bekendt med at
have som repræsentanter for skibet og det flag vi fører agter''. ''På
flodbredden (Demerara River, Guyana) ved siden af skibet, lå en af de
første dage en kano med pagaj - fristelsen var for stor, den måtte vi
naturligvis på ekspedition i. Så vi, Inge-Lise og jeg, padlede rundt i
nabolaget og betragtede hytter, palmer og jungle. Kort efter at vi havde
gjort fast igen, dukkede ejeren op, en ældre gråhåret kvinde. Hun fik et
håndtryk og lidt gaver af os som tak for lån af kanoen. Hun og hendes
familie boede på den anden side af floden, hvor nevøen havde lidt
havelandbrug på et ryddet stykke jungle. Frugterne sejlede han hver dag
ind til et marked i byen Linden en times kanosejlads længere oppe ad
floden. Inge-Lise og også senere undertegnede blev inviteret over i
deres hytte. Det var en stor oplevelse at få lov til at komme i deres
beskedne men uhyre gæstfrie hjem. Den slags oplevelser er der mange af,
hvis man er modtagelig. Familien kom senere om bord på genvisit og
hilste på besætningen. Da vi afgik, stod hele familien og vinkede farvel
fra flodbredden, skønt det var på den årle morgenkvist''. ''Vi ligger
gerne nogle dage måske 3 dage eller en uge under land. F. eks. på
Falklandsøerne, der er der fint at komme i land, folk er meget gæstfrie
der''. ''Tidligt om morgenen gik jeg en tur i byens (Samsun, Tyrkiet)
forretningskvarter for at iagttage byen vågne. Dagen startede roligt med
kun lidt trafik, gradvist kommer der flere og flere mennesker på gaden,
gående, på cykel, i bil eller bus, alle travlt på vej til arbejde.
Varevogne kommer til slagteren, bageren eller supermarkedet med deres
daglige leveringer. Når man ikke selv har travlt er det spændende at se
hvordan hjulene begynder at dreje i den daglige rutine. I denne by mødte
jeg mange venlige mennesker, smilende og behjælpelige, når jeg spurgte
om vej rundt i byen, de fleste talte godt engelsk. Det var en positiv
oplevelse. Efter Dardanellerne sejlede vi til en lille by, der hedder
Masta på den græske ø Chios, hvor der var en cementbro med brohoved
halvt så langt som skibet. Her fortøjede vi. Vi lå der natten over, så
jeg gik op på den lokale Taverne og talte med fiskerne, der sad der om
aftenen og nød noget græsk musik. De taler ikke så meget engelsk og jeg
taler ikke så meget græsk, men vi gjorde vort bedste og fik talt sammen,
det var rare mennesker''. ''Houston ligger ca. 6 timers sejlads op af
floden fra Galveston i Mexican Gulf, og hele området er fyldt med
olietanke og raffinaderier placeret side om side. Hele vejen hørmer af
olie og gas, Texakanerne siger ''det stinker af dollars'', og penge er
der nok af i det område. Alle over 15 år har mindst en eller to store
biler, og nogle fjernsyn, ''ellers kan man bare ikke leve''. Du kan
opleve dem stige ud af deres store bil, talende i mobilen (som de altid
gør), med en stor cowboyhat på hovedet (altid), og resten af tøjet
ligner noget fra skralde-spanden, eller du kan finde dem på en
restaurant ordre en firepersoners middag til een person og spise det
hele plus et par pandekager til kaffe. Men de er

41

alle venlige og hjælpsomme, hvis du beder om hjælp vil de gøre alt for
at hjælpe dig\ldots{}. Fra skibet til et handelscenter kørte vi normalt
i taxa, men vor kok var lidt mere smart, han spurgte en fyr der holdt
ved havnen i sin bil om vej til centeret, fyren svarede ''hop ind knægt
så kører jeg dig derhen'', vor kok sparede 5 \$ til taxaen, sådan er
folkene i Texas. Tampico (Mexico) er en mellemstor by fra starten bygget
i typisk spansk stil, senere omgivet af nyere fire og fem etage
bygninger, men bygget i en stil så de ikke spolerer idyllen i den gamle
bydel. Jeg gik op for at handle lidt, men udbudet af varer var meget
dyrt, og af en tvivlsom kvalitet, så jeg forlod de handlende med deres
varer. I stedet hyggede jeg mig med at studere menneskemylderet i
bykernen. Langt de fleste indbyggere er af mixet spansk-indiansk blod,
men i modsætning til de sydamerikanske lande er det spanske dominerende
her. Jeg fandt ingen der kunne tale engelsk her, så jeg måtte nøjes med
det spanske jeg kan, men det gik. Tampico var sidste havn i denne rejse,
der startede i Dammam den 10. august, derefter havnene Bahrain, Shariah,
Samsun, Istanbul, Mestas, Algeciras, Newport News, Houston og sidst
Tampico den 7. december, slutningen på en rejse og begyndelsen på en ny.
Den næste havn blev Dos Bocas, Mexico, hvor vi lastede stålrør til
Genova i Italien. Vi afsejlede den 12. december og sejler nu i Caribean
Sea. Vejret er dejligt solrigt med 30 grader uden for. Det er ikke let
at komme i julestemning i denne varme. Vi vil formentlig holde julen i
Nordatlanten, vi har et lille juletræ og noget julepynt til messerne, og
jeg købte to kalkuner og en pattegris i Mexico til aftensmaden og
julefrokosten''. I en sømandsby vil man ofte opleve samtaler og snak,
der handler om den store verden og verdens havnebyer. I en del hjem
findes der eksotiske ting, der signalerer, at her bor en familie, hvor
en fra husstanden arbejder på De syv Have. Sømanden er flink til at
hjembringe sager fra fjerne lande.

Containerskibe Tidligere blev mange varer transporteret rundt i verden
med skibe. Og selv om mange måske ikke tror det, så er det stadigvæk
sådan. I dag er skibene dog blevet meget større og mere avancerede. De
skibe, der i dag transporterer varer rundt i verden, hedder
containerskibe. De transporterer faktisk omkring 90 \% af alt
internationalt gods147. Et containerskib sejler med store containere
fyldt med alt mellem himmel og jord såsom biler, frugter, kød, tøj,
legetøj, fjernsyn, råstoffer og meget mere. Containerskibe sejler i
faste ruter over hele verdenen. Man siger også, at de sejler i
liniefart, hvilket er et udtryk for regelmæssig trafik på bestemte
havne. En virksomhed, som ejer og sejler med skibe, kalder man for et
rederi. Der arbejder ca. 30.000 mennesker for de danske rederier, heraf
næsten 13.000 i handelsflåden. Hele det maritime erhverv, kaldet Det Blå
Danmark, har ca. 100.000 ansatte. Kun ca. 65 procent af de ansatte i
handelsflåden er danskere, da skibsassistener og dækspersonale ofte
hyres i udlandet. Containerskibsfarten bliver større for hver dag, der
går, da produkter, som tidligere blev transporteret på paller eller i
løs vægt, i stadigt højere grad transporteres i containere. Herudover
produceres flere og flere varer i Fjernøsten, for så at blive
transporteret med containerskibe til den vestlige verden.

147

Gods er en massebetegnelse for varer eller materialer, der er undervejs
mellem en leverandør og en modtager.

42

Containerskibet er egentlig ikke nogen ny opfindelse, for allerede i
1955 kom amerikaneren Malcolm P. Mclean, der ejede et stort
vognmandsfirma, på en revolutionerende tanke: Han begyndte at pakke alt
gods fra sine kunder ned i store kasser af samme størrelse, der blev
kørt til nærmeste afskibningshavn og sat ombord i et skib, der sejlede
det til modtagerhavnen, hvor en anden lastbil hentede det og kørte det
ud til kunden. Mclean havde købt nogle tankskibe fra 2. Verdenskrig,
hvoraf Ideal X i 1955 blev bygget om til verdens første containerskib.
Herefter ændrede hele skibsfarten sig meget, da skibene blev større og
større og kunne sejle hurtigere, og der var heller ikke længere brug for
så mange søfolk til at sejle skibene. Rederierne begyndte også mere og
mere at arbejde i land og startede et tæt samarbejde med havne, pakhuse
og vognmandsforretninger. Derfor kan rederierne i dag tilbyde deres
kunder at transportere varer lige fra producenten og til forbrugeren.
Det første danske firma, der fik et containerskib, var Østasiatisk
Kompagni (ØK). Det skete i årene 1971-72, hvor Nakskov Skibsværft
leverede skibene Falstria og Meonia, mens værftet B\&W leverede de mere
end dobbelt så store Selandia og Jutlandia. Herefter kom virksomheden A.
P. Møller- Mærsk på banen, da den i 1974 fik leveret sit første
containerskib, kaldet Svendborg Mærsk. I dag er A. P. MøllerMærsk
verdens største containerrederi og ejer mere end 500 containerskibe og
omkring 1.400.000 containere verden over. A. P. Møller - Mærsk ejer også
verdens største containerskibe.
