\chapter{Tørre tæer}\label{tuxf8rre-tuxe6er}

Ifølge det gammelkendte mønster fra sejlskibstiden, var det sønner af
søfolk, der stod til søs, men denne tradition ændrede sig så småt, da der
kom flere dampskibe i de danske havne. Efterhånden som coasterne overtog
skibsfarten, ændrede mønstret sig igen: Søfolkene kom nu fra alle steder i
Danmark og fra næsten alle samfundslag. Også utilpassede unge tog ud at
sejle. 

I coasterperiodens begyndelse, altså i 1950'erne og 60'erne, var livet på
land stadigvæk ret traditionelt, set i forhold til søfarten og de grupper
af unge, som valgte at stå til søs. Sådan er det fortsat, selv om den
danske \wordd{coasterflåde}{Coasterflåde er fragtskibe i handelsflåden,
der går ind under betegnelsen coaster.} bliver formindsket, som tiden går.
En stor del af byens drenges fritid var henlagt til havnen.  En havn var
altid en tillokkende legeplads med skibsdele, \wordd{bomme}{Bommen er
monteret på et skibs mast og kan fungere som en kran, som man kan håndtere
gods med. På sejlskibe bruges bomme til at spile sejlene ud med.},
\wordd{spil}{Spil anvendes til at hejse tunge ting op med --- eller til at
hale skibets ankre om bord på skibet med.}, \wordd{ankre}{Ankre vil sige,
at man lader skibets ankre glide ud i vandet og ned på havbunden, så at de
kan holde skibet fast til havbunden. På en båd siger man, at man kaster
anker. Det kan man også med et lille anker, men ankrene på et stort skib
er alt for store til, at man kan kaste dem.}, skibe på \wordd{værft}{Værft
er et værksted eller en fabrik, hvor man bygger og reparerer skibe.} og
måske oven i købet oplagte skibe, der kunne bruges til uofficielle
legepladser. De fleste børn havde adgang til en \wordd{jolle}{Jolle er en
lille båd.} eller et andet fartøj. På den måde blev de vant til at
håndtere en åre og et sejl i forskellige situationer.

\begin{quote}
    ``Hvis det var regnvejr, entrede drengene ad en svær trosse op om bord på
    et større skib, der var sat op ved højvande. Her kravlede de ned i
    lastrummet og snittede bogstaver og skar skibe ind i træskotterne. Eller
    også sneg de sig om bord på nogle alderstegne galeaser, der ikke kunne
    leve op til sødygtighedskravene mere og lå pensioneret i havnen, oplagt
    mellem de sorte jagtpæle. Herude lå også en overgang et par temmelig store
    skonnerter oplagt, de havde tidligere været på \wordd{langfarten}{Langfart
    er skibe, der sejler på lange strækninger til fjerne lande.}. Men tiden
    havde agterudsejlet den form for søfart. Eller også så satte han og de
    andre drenge sejl på deres joller. Disse var af vidt forskellig art, en
    del var aflagte skibsjoller, der var rigget med det obligatoriske
    firkantede \wordd{smakkesejl}{Et smakkesejl sidder på en smakkejolle, som
    er en lille jolle, som blev brugt til at sejle mellem øerne med. Man
    transporterede kvæg mellem gårdene, folk til fester på naboøen eller måske
    varer fra den nærmeste by. Smakkejollen kunne fås i flere forskellige
    størrelser og blev gerne lavet af en bådebygger eller en dygtig
    håndværker. Jollen var let at kende med sine firkantede sejl.} og en
    trekantet \wordd{stagfok}{Stagfok er et trekantet forsejl, anbragt i
    forenden af sejlskibet, se under faktaboks.} samt et topsejl til lettere
    vejr. Så lå de der og sejlede frem og tilbage gennem havnen hele aftenen
    også efter det var blevet mørkt. De tunge joller var blevet til
    velsejlende tremastskonnerter eller til store barker, der kæmpede mod
    forbandet vejr ved Kap Horn.''
\end{quote}

Søfartssamfundets knægte havde ofte adgang til at komme om bord på en af
byens coastere, der lå i hjemhavnen for at for at komme på værft eller for
at holde jul og andre højtider. Mange havde søfolk i familien --- det kunne
være ens far, bror, onkel --- eller det kunne være en person inden for
vennekredsen. Beretninger fra søen var også hverdagskost, ligesom
sommerferieture om bord på farens skib var et kendt fænomen.  Denne nære
kontakt gjorde, at de fleste af de vordende søfolk fra disse samfund havde
et forhåndskendskab til erhvervet, før de havde deres første udmønstring. 

\begin{quote}
    ``Sådan noget hørte han jo om, når den store familie var samlet til
    spisninger. Men der var også andre beretninger, altid var de fra søens
    verden. Der var en onkel, hvis beretninger fængslede tilhøreren, han var
    selv forlist et par gange, men han var lidt fanden i voldsk og havde en
    ret kæk måde at berette om tingene på.''
\end{quote}

\begin{quote}
    ``En tur på sommertogt i skoleferien ud i den forjættede verden med et
    rigtigt skib, en coaster tilhørende familien, blev en stærk oplevelse.
    Nordatlantens ubarmhjertige søer havde tæsket maden ud af maven på ham og
    over bord, men det var glemt så snart skibet var inde i
    \wordd{smulten}{Smult er det rolige vand i læsiden af en kyst og/eller
    mellem øer og grunde. Dette udtryk bruges som så mange andre af søens
    udtryk i overført betydning i land.} bag nordnorske øer og skær.''
\end{quote}

Tilgangen til søfarten, rekrutteringen, i søfartsbyerne foregik således
naturligt, som den altid havde gjort. Det gav også byer med søfartsskole
og/eller navigationsskole en god kontakt med søens verden. 

Også unge, som ikke boede i søfartsbyer, tog ud at sejle, fordi de havde
lyst til at opleve den store verden, fordi de havde hørt spændende og
dramatiske beretninger fra livet på havet --- eller fordi de var
skoletrætte: 

\begin{quote}
    ``Da jeg var ca. midt i 10. klasse blev jeg skoletræt, og så var der en
    ven af mig, der foreslog at jeg tog ud at sejle. Jeg var så heldig at få
    hyre på en coaster. Og der på den coaster besluttede jeg mig for at tage
    på søfartsskole. Sølivet, det var noget for mig, og jeg skulle fortsætte
    til søs.''
\end{quote}

\begin{quote}
    ``Ja, hvordan kom jeg ud at sejle. Jeg kender en del stykker på en 19 år,
    der har været ude at sejle, der var delte meninger, om hvordan det var. Så
    fik jeg chancen for at komme ud og prøve det, og tog den, og det har været
    godt nok indtil videre, og det tror jeg også det bliver ved med at være.
    Jeg har gået og leget lidt med tanken om at komme ud at sejle, og havde
    også set en del om sømandslivet på fjernsynet og har snakket med folk, der
    har fortalt spændende historier rundt omkring ude fra verden.''
\end{quote}

