\chapter{Afskeden}

Før i tiden, dengang da sejlads var forbundet med stor risiko for
forlis, lød hilsenen altid fra de gamle søfolk, der stod og så skibene
glide ud fra Marstal Havn: ``Ja så farvel da, og lykke på rejsen!''

Og lykke kunne godt behøves i sejlskibstiden, for da forliste flere skibe.
Hvert år blev skibe med hele \wordd{besætninger}{Besætning er udtryk for
de søfolk, der er ansat om bord til at betjene skibet, således at det kan
sejle.} væk derude. Nu er det imidlertid helt andre tider. Sikkerheden på
skibene er blevet meget bedre i coasterens tid, men der er fortsat brug
for lykke på rejsen! En af ulemperne til søs er det stigende antal
piratoverfald.

I et søfartssamfund var og er kendskab til sølivet inden for rækkevidde,
og mange af de nye søfolk havde en hel del forhåndsviden. Men
forkundskaber indhentet på en søfartsskole blev efterhånden mere
almindeligt, og søfartsuddannelse er i dag et krav, før man står til søs.
I starten af coasterperioden var der dog fortsat mange, der
\wordd{mønstrede}{Mønstre vil sige at blive ansat på skibet. Når
ansættelsen ophører, afmønstrer man fra skibet igen.} direkte ud uden
søfartsskoleuddannelse. I søfartsbyerne foregik det ofte i et kendt skib
med velkendte folk om bord blandt besætningen.

\begin{quote}
    ``Skibet, der var
    bygget på det lokale H.C. Christensens Stålskibsværft, gled ud af den
    hjemlige havn bestemt for dansk havn, videre til engelske pladser og
    over Biscayen frem mod Middelhavet. Ja, når sandt skal siges, følte han
    sig i det øjeblik faktisk som fuldbefaren, skønt der kun var få måneder
    på (søfarts)bogen. Det var der nu ikke så mange der kunne se på ham på
    det tidspunkt. I den ubehagelige Biscayabugt var han måske sin stilling
    som dæksdreng mere bevidst. Men i Gibraltarstrædet, hvor han så
    klipperne i virkeligheden og anløb havne, han havde hørt søfolkene
    berette om, ja da vendte følelsen af sømand tilbage dybt inde i den
    søgrønne sjæl. Snart var der delfiner for boven og flyvefisk på den
    morgenvåde dæk -- hejsa hvor det gik. Inde i det badevandsblå Middelhav,
    der slikkede op forbi koøjerne, følte han sig i pagt med slægten og de
    gamle søfolk på havnebænken derhjemme, måske på en lidt mere ydmyg og
    taknemmelig måde end før. Nu kunne han blande sig i koret blandt
    kammeraterne, når han kom hjem. Jo, det var sand for dyden livet det
    her.''
\end{quote}

Der er selvfølgelig mange forventninger, når man står til søs. Der er
mange oplevelser, der lever op til dette. Men omvendt er der rigtig mange
sure oplevelser; specielt søsyge kan være et overordentligt ubehageligt
bekendtskab. De færreste slipper for at stifte bekendtskab hermed, nogle
får det overstået med én gang, andre må døje med det hele livet, og en del
må lige turen igennem, når de mønstrer ud igen efter ferie i land.
Erindringerne om de ubehagelige oplevelser fordufter imidlertid ganske
forunderligt hurtigt, når skibet kommer i smult vande eller søsygen er
klaret --- så er det hele glemt for en stund. Man glemmer også hurtigt de
undertiden ubehagelige møder med andre folk om bord. Det er ikke alle
officerer, der er lige venlige, og på havet har de stor magt. Og det
afhænger i høj grad af dem, om skibet er `et godt skib' eller en
`møgkasse', som man forbander langt væk.  Skibskammeraternes væremåde og
opførsel spiller i sagens natur også en stor rolle. I et skib er man
sammen i døgnets 24 timer --- ofte med lange sørejser. Man kan ikke cykle
hjem efter udskejning (fyraften) eller tage på weekendophold. Omvendt kan
man i coastere være heldig at komme i land på eksotiske steder. 

\begin{quote}
    ``Han (skipperen) har de egenskaber, der gør ham
    til en vellidt og samtidig respekteret mand. Han kan sidde og få
    eftermiddagskaffen sammen med den øvrige besætning på frivagten på
    agterdækket og sludre frit, og samtidig have den fornødne autoritet i
    kraft af sin kunnen og sit væsen. Han er ikke humørsyg og hans ord er
    det samme den ene dag som den anden. Man kan med andre ord føle sig tryg
    med ham som skipper, og det er vigtigt ikke mindst, når en ung knægt
    skal ud og prøve om sølivet er noget for ham. Skipper forstår lige at
    give folk en opmuntring, hvis tingene hober sig op. Takket være ham,
    samt selvfølgelig en god besætning i øvrigt, er ANNE BOYE et godt skib
    at være i på denne del af rejsen. Vi har en fin rejse til Havanna og
    `over Dammen', Atlanten.''
\end{quote}


Alt i alt må man lære at klare sig på et skib --- man bliver udsat for
nogle begivenheder, der uundgåeligt sætter skub i udviklingen fra dreng
til mand.
