\chapter{Levnedsomstændigheder}\label{levnedsomstuxe6ndigheder}

Kosten om bord i danske skibe er god og veltillavet i nyere tid. Der er
rigelig og meget varieret kost. En normal husholdning kan slet ikke være
med. Her fortæller skibskokken Inge-Lise Kromann: ''Jeg har nu været
påmønstret i 5 måneder. Det er et hårdt job at være kok og skal lave mad
til 8-9 mennesker. Jeg har tidligere prøvet det for en kortere periode i
1978 om bord på MARIE BOYE i nærfarten, der var vi ikke så mange og vi
var tit i havn og kunne supplere provianten. Her er det mere hårdt. Til
at begynde med måtte jeg arbejde 9-10 timer dagligt, det var
selvfølgelig fordi det var uvant, og det hele slingrede frem og tilbage
og jeg måtte stå at holde fast samtidig med at jeg skulle lave mad. Jeg
har brændt mig et utal af gange på ovnen. Men der er kommet mad på
bordet hver dag, og folk kommer og siger, at de har taget på, så det må
jo være tegn på at det er gået godt nok. Råmaterialerne har været meget
fine, meget flottere end i et hjem. Da jeg så
tremåneders-provianteringen, der kom om bord i Shoreham fra den danske
skibshandler Wrist, da tænkte jeg at der var nok til en hel hær i et år.
Der var ud over det almindelige røget laks og ål og andre delikatesser.
Men nu er vi på vej til Panama og det er tre en halv måned siden vi
provianterede, så der skal suppleres op med nogle af varerne, af andre
er der fortsat rigeligt. For en del varegrupper passer det godt med nye
forsyningerne nu, specielt da skibet skal ud på en lang rejse til
Australien.''. Spiseplanen for en almindelig dag på en dansk coaster ser
således ud: Kl. 6.30: Kaffe Kl. 8.00 - 8.30: Morgenmad med æg og bacon
Kl. 10.00 -- 10.30: Formiddagskaffe Kl. 11.30: Middag bestående af to
varme retter til styrmanden, der skal på vagt kl. 12.00 Kl. 12.00 --
13.00: Middagsmad for den øvrige besætning Kl. 15.00 -15.30:
Eftermiddagskaffe med nybagt kage Kl. 17.30: Aftensmad til skipperen,
der går på vagt på broen kl. 18.00 Kl. 18.00: Aftenskafning138 for
resten af besætningen. Aftensmaden består af brød og mange slags pålæg
samt en lun ret, fremstillet af rester fra tidligere dages varme retter.
Hygiejnen til søs er generelt høj: ''Når danskere i land hører om danske
søfolk, har de ofte en urealistisk forestilling om den race. Søfolk er
generelt ordentlige folk. Her om bord møder dæksfolkene ikke op i messen
i arbejdstøj, nej de er omklædt i rent tøj til måltiderne.
Renligholdelse af skibet og den personlige hygiejne er stor. At den er
større end kollegernes fra land, fik jeg ved selvsyn konstateret i min
tid i marinen, da jeg lå i kompagni med befarne søfolk og folk fra land.
Søfolkene skilte sig klart ud, de er rutinerede i renligholdelse''.
Søfolkene i coasterne har i dag enmandskamre, hvor det i begyndelsen af
coasterperioden, 1950'erne, 60'erne og 70'erne, var almindeligt med
tomandskamre. Apteringen139 i coasterne er udmærket, men selvfølgelig
ikke af samme standard og udstyr som på de store enheder i den danske
handelsflåde.

138 139

Aftenskafning betyder aftensmad. Aptering er indretningen, altså
kahytter m.v. i et skib.

36

Søfolkene går klædt lige som alle andre folk i land. Coastersøfolk har
ikke uniformer, som man har i de store rederier, dog undertiden en
''kedelsut'' med rederiets logo bag på. Men når søfolkene går i land, er
de ikke til at skelne fra andre folk. Tro og overtro er ligeledes, som
på land - blandt ''landkrabber''. Dog er der enkelte søfolk, der
stadigvæk forbyder eller påtaler fløjtetoner om bord. Den gamle overtro,
at man påkalder uvejr ved at fløjte, er ikke helt forsvundet. I
coasteralderens første tid var der en del folk, som ikke ville starte
årets første rejse på en fredag eller mandag, det var dog mest folk med
eget skib, de såkaldte selvejere. ''I sejlskibstiden måtte sømanden ikke
engang vise den afslappethed at fløjte. Dengang fik man en god tæt
kindhest140, hvis man tillod sig den afslappethed at fløjte en kæk
melodi. Man mente, at det hidkaldte blæsevejr. Sandheden var jo nok den,
at almindelig adspredelse som fløjten kunne aflede sømandens
opmærksomhed fra vinden og sejlene, og det kunne være farligt at slække
for årvågenheden. Den overtro eller erfaring, fra de hvide og grå sejls
dage, har holdt sig helt op til nutiden. Da knægten var med på et af
sine sommertogter, var han kommet for skade at fløjte en lystig melodi.
Den ældre maskinmester stillede sig bombastisk op foran den formastelige
letsindige og udslyngede: Dit ansigt piver sømand!!! - en bemærkning,
der ætsede sig mere fast end en ørefigen141 ville have kunne
præsteret''.
